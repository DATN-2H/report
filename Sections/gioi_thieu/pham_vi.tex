\subsection{Phạm vi đề tài}

Để đảm bảo dự án khả thi và tập trung vào việc giải quyết các mục tiêu cốt lõi trong khung thời gian và nguồn lực hạn chế, phạm vi của đề tài được xác định rõ ràng như sau:

\subsubsection{Phạm vi Chức năng}

Hệ thống sẽ tập trung vào các chức năng thiết yếu cho quy trình đặt món và quản lý cơ bản tại một nhà hàng điển hình. Các chức năng chính bao gồm:
        \begin{itemize}
            \item Quản lý Thực đơn: Thêm, sửa, xóa, ẩn/hiện món ăn, danh mục món ăn, quản lý giá bán và mô tả cơ bản.
            \item Quản lý Bàn/Khu vực: Thiết lập sơ đồ bàn, quản lý trạng thái bàn (trống, đang phục vụ, đã đặt, cần dọn).
            \item Quy trình Đặt món tại bàn: Nhân viên phục vụ nhận đơn qua thiết bị di động (máy tính bảng/điện thoại), gửi đơn tự động đến bếp/bar. Bao gồm khả năng chọn món, tùy chỉnh số lượng, ghi chú đơn giản.
            \item Hiển thị Đơn hàng tại Bếp (Kitchen Display System - KDS): Hiển thị danh sách các món cần chế biến, trạng thái món, ... lên màn hình.
            \item Quản lý Đơn hàng cơ bản: Xem danh sách đơn hàng, trạng thái đơn hàng.
            \item Quy trình Thanh toán: Hỗ trợ tạo hóa đơn tạm, áp dụng giảm giá đơn giản (theo phần trăm hoặc số tiền cố định), ghi nhận thanh toán bằng tiền mặt và tích hợp thanh toán cơ bản qua ví điện tử.
            \item Báo cáo cơ bản: Báo cáo doanh thu theo ngày, báo cáo các món bán chạy trong ngày.
            \item Quản lý Tài khoản người dùng: Tạo và quản lý tài khoản cho các vai trò (Quản lý, Phục vụ, Bếp, Thu ngân, Quản lý) với phân quyền truy cập chức năng tương ứng.
        \end{itemize}

\subsubsection{Phạm vi Kỹ thuật}

Các giới hạn và lựa chọn về mặt công nghệ được xác định như sau:

\begin{itemize}
    \item \textbf{Kiến trúc hệ thống:} Dự kiến xây dựng theo kiến trúc ứng dụng kiến trúc Client-Server cơ bản để đơn giản hóa việc phát triển và triển khai trong giai đoạn đầu.
    \item \textbf{Công nghệ phát triển (Dự kiến):}
        \begin{itemize}
            \item \textit{Frontend (Giao diện người dùng):} Sử dụng một framework JavaScript hiện đại như React cùng với ShadcnUI và TailwindCSS để xây dựng giao diện tương tác và responsive.
            \item \textit{Backend (Xử lý logic):} Sử dụng một nền tảng phía máy chủ phổ biến Java (với Spring Boot) dựa trên kinh nghiệm của thành viên trong nhóm và các đặc điểm của nền tảng.
            \item \textit{Cơ sở dữ liệu:} Sử dụng hệ quản trị cơ sở dữ liệu quan hệ (SQL) PostgreSQL để đảm bảo tính nhất quán dữ liệu.
            \item \textit{Giao tiếp Real-time (nếu cần cho KDS/cập nhật trạng thái):} Có thể sử dụng WebSockets hoặc các kỹ thuật tương tự.
        \end{itemize}
    \item \textbf{Nền tảng triển khai:} Hệ thống chủ yếu hoạt động trên trình duyệt Web trên các thiết bị như máy tính, máy POS, máy tính bảng. Chưa bao gồm việc phát triển ứng dụng di động gốc (Native Mobile App) cho iOS hay Android trong phạm vi này.
    \item \textbf{Bảo mật:} Áp dụng các biện pháp bảo mật cơ bản như mã hóa mật khẩu, bảo vệ chống lại các lỗ hổng web phổ biến, phân quyền dựa trên vai trò. Không bao gồm các biện pháp kiểm thử xâm nhập (penetration testing) chuyên sâu.
\end{itemize}