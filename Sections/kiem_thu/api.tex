\subsection{Kiểm thử API}
Kiểm thử API là quá trình kiểm tra yêu cầu và phản hồi trong giao tiếp giữa client và server trong ứng dụng API. Quá trình này tập trung vào kiểm tra tính chính xác và tuân thủ các quy tắc và giao thức của API, với sự tập trung vào lớp business logic của phần mềm mà không liên quan đến giao diện người dùng. \\

Nhóm đã chọn công cụ sử dụng là Postman để thực hiện kiểm thử API. Postman cung cấp một môi trường để gửi yêu cầu API và kiểm tra phản hồi từ hệ thống. Bằng cách sử dụng Postman, nhóm có thể thực hiện kiểm thử API một cách hiệu quả và đáng tin cậy cho hệ thống. \\

Để đảm bảo tính đầy đủ và chính xác của API, nhóm tập trung chú ý vào việc:

\begin{itemize}
    \item Kiểm tra tính đúng đắn của dữ liệu đầu vào: Đảm bảo rằng API xử lý đúng các dạng dữ liệu đầu vào và kiểm tra xử lý hợp lệ của các trường dữ liệu. Bao gồm kiểm tra các trường hợp dữ liệu hợp lệ, không hợp lệ và ranh giới.

    \item Kiểm tra trạng thái và phản hồi của API: Xác minh rằng API trả về trạng thái và mã phản hồi chính xác cho các yêu cầu. Đảm bảo rằng các mã phản hồi như 200 OK, 400 Bad Request, 401 Unauthorized, 500 Internal Server Error được xử lý đúng và phù hợp.

    \item Kiểm tra các chức năng và hoạt động của API: Đảm bảo rằng các chức năng và hoạt động của API hoạt động chính xác và đáp ứng yêu cầu. 
\end{itemize}

Quá trình thực hiện kiểm thử API:

\begin{itemize}
    \item Sử dụng Postman tạo collection để tổ chức các API đã triển khai. Collection sẽ chứa các request tương ứng với từng API.
    \item Mô tả các API đã triển khai: Trong mỗi request trong collection, nhóm xác định thông tin về API đã triển khai, bao gồm URL của endpoint, phương thức HTTP (GET, POST, PUT, DELETE), các tham số yêu cầu, tiêu đề và dữ liệu yêu cầu.
    \item  Thực hiện việc thiết lập Environment trong Postman
    để dễ dàng sử dụng các biến môi trường cho các request. (URL, thông số, ...)
    \item Kiểm thử API: Thực hiện các request trong Postman bằng cách gửi yêu cầu tới endpoint của API đã triển khai. Postman hiển thị phản hồi từ server, bao gồm mã phản hồi, dữ liệu trả về và bất kỳ lỗi nào có thể xảy ra.
    \item Để minh họa cách sử dụng API một cách rõ ràng, nhóm có Example cho mỗi request. Example hiển thị một ví dụ cụ thể về dữ liệu yêu cầu và phản hồi mà nhóm đã thực hiện.

\end{itemize}

Kết quả kiểm thử: 


Đường dẫn truy cập API Document: 
