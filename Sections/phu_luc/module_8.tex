\subsection{Module MD-08: Tích hợp Bếp (Kitchen Integration)}
\subsubsection{Use Case UC-MD08-01: Xem Đơn hàng/Món ăn Mới trên KDS hoặc Nhận Phiếu in Bếp}
\begin{longtable}{|m{4cm}|p{11cm}|}
\caption{Đặc tả Use Case UC-MD08-01: Xem Đơn hàng/Món ăn Mới trên KDS hoặc Nhận Phiếu in Bếp} \label{tab:uc_md08_01_final_v2} \\
\hline
\multicolumn{2}{|c|}{\textbf{2.1. Tóm tắt (Summary)}} \\
\hline
\textbf{Mục} & \textbf{Nội dung} \\
\hline
\endhead % Header cho các trang tiếp theo
\hline
\endfoot % Footer cho bảng
\hline
\endlastfoot % Footer cho trang cuối cùng
Use Case Name & Xem Đơn hàng/Món ăn Mới trên KDS hoặc Nhận Phiếu in Bếp \\
\hline
Use Case ID & UC-MD08-01 \\
\hline
Use Case Description & Cho phép Nhân viên bếp (US-04) tiếp nhận thông tin về các đơn hàng hoặc các món ăn mới cần chuẩn bị, bằng cách quan sát chúng xuất hiện trên Màn hình Hiển thị Bếp (KDS) hoặc nhận phiếu in từ máy in bếp. \\
\hline
Actor & US-04 (Nhân viên bếp) \\
\hline
Priority & Must Have \\
\hline
Trigger & - Một đơn hàng mới (hoặc các món mới trong một đơn hàng đang xử lý) được Nhân viên POS gửi yêu cầu chuẩn bị (kết quả của UC-MD05-08, UC-MD06-06, UC-MD07-06). \\
\hline
Pre-Condition & - Nhân viên bếp đang có mặt tại khu vực làm việc. \newline - \textit{Đối với KDS:} Màn hình KDS đang hoạt động, kết nối mạng và đã được cấu hình để nhận đơn hàng. \newline - \textit{Đối với Máy in bếp:} Máy in bếp đang hoạt động, có giấy, mực và kết nối mạng. \\
\hline
Post-Condition & - Nhân viên bếp nhận biết được có yêu cầu chuẩn bị món ăn mới. \newline - Thông tin tóm tắt ban đầu của đơn hàng/món ăn mới được hiển thị/in ra. \newline - Nhân viên bếp sẵn sàng để xem chi tiết yêu cầu (UC-MD08-02). \\
\hline
\multicolumn{2}{|c|}{\textbf{2.2. Luồng thực thi (Flow)}} \\
\hline
\textbf{Mục} & \textbf{Nội dung} \\
\hline
Basic Flow (Sử dụng KDS) & 1. Nhân viên bếp (US-04) đang theo dõi Màn hình Hiển thị Bếp (KDS). \newline 2. Khi có yêu cầu chuẩn bị món mới được gửi từ POS, một đơn hàng/phiếu (ticket) mới xuất hiện trên giao diện KDS. \newline 3. Đơn hàng/phiếu mới này thường được làm nổi bật (ví dụ: màu sắc khác, âm thanh thông báo ngắn, nhấp nháy) để thu hút sự chú ý của US-04 (BR-UC8.1-1). \newline 4. US-04 quan sát và nhận biết có đơn hàng/phiếu mới cần xử lý. Thông tin ban đầu hiển thị có thể bao gồm mã đơn/số bàn, loại đơn (Eat-in, Takeout, Delivery), thời gian gửi. \\
\hline
Alternative Flow & \textbf{Basic Flow (Sử dụng Máy in bếp):} \newline    1. Máy in bếp tự động in ra một phiếu yêu cầu chuẩn bị món mới (kitchen order ticket - KOT) khi có yêu cầu từ POS. \newline    2. Nhân viên bếp (US-04) nghe thấy tiếng máy in hoạt động hoặc kiểm tra định kỳ khay ra giấy của máy in. \newline    3. US-04 lấy phiếu in mới. \newline    4. US-04 đọc các thông tin ban đầu trên phiếu in. \\
\hline
Exception Flow & \textbf{KDS - 2a. Lỗi hiển thị đơn mới trên KDS:} \newline    1. Do lỗi kết nối mạng giữa KDS và máy chủ, hoặc lỗi phần mềm KDS, đơn hàng mới không xuất hiện hoặc xuất hiện chậm trễ. \newline    2. Nhân viên bếp không nhận được yêu cầu. Cần có quy trình dự phòng (ví dụ: nhân viên POS thông báo miệng) hoặc khắc phục sự cố kỹ thuật KDS. \newline \textbf{Máy in bếp - 1a. Lỗi máy in (hết giấy, kẹt giấy, hết mực...):} \newline    1. Máy in không thể in ra phiếu yêu cầu. \newline    2. Nhân viên bếp không nhận được yêu cầu. Nhân viên POS có thể nhận được thông báo lỗi từ hệ thống (nếu IoT Box/dịch vụ in có phản hồi) hoặc cần kiểm tra thủ công. Cần khắc phục sự cố máy in và có thể yêu cầu POS in lại. \\
\hline
\multicolumn{2}{|c|}{\textbf{2.3. Thông tin bổ sung (Additional Information)}} \\
\hline
\textbf{Mục} & \textbf{Nội dung} \\
\hline
Business Rule & - \textbf{BR-UC8.1-1 (KDS):} KDS nên có cơ chế thông báo rõ ràng (âm thanh, hình ảnh) khi có đơn hàng/phiếu mới đến để đảm bảo nhân viên bếp không bỏ sót. \newline - \textbf{BR-UC8.1-2 (System):} Việc định tuyến các món ăn đến đúng KDS hoặc máy in bếp (nếu có nhiều thiết bị cho các khu vực khác nhau như bếp chính, quầy bar, trạm salad) được thực hiện tự động bởi hệ thống dựa trên cấu hình Danh mục Sản phẩm POS (FR-MD02-20). \newline - \textbf{BR-UC8.1-3:} Thứ tự xuất hiện của các đơn hàng/phiếu mới trên KDS hoặc thứ tự in ra của phiếu nên tuân theo thời gian yêu cầu được gửi từ POS (first-in, first-out là mặc định). \\
\hline
Non-Functional Requirement & - \textbf{NFR-UC8.1-1 (Timeliness - KDS):} Đơn hàng/phiếu mới phải xuất hiện trên KDS gần như ngay lập tức (trong vòng vài giây) sau khi được gửi từ POS. \newline - \textbf{NFR-UC8.1-2 (Reliability - KDS/Printer):} Cả KDS và máy in bếp phải hoạt động ổn định, đảm bảo không bỏ sót bất kỳ yêu cầu chuẩn bị món nào. \newline - \textbf{NFR-UC8.1-3 (Usability - KDS):} Thông báo đơn hàng mới trên KDS phải dễ nhận biết. \newline - \textbf{NFR-UC8.1-4 (Clarity - Printer):} Phiếu in bếp phải rõ ràng, dễ đọc. \\
\hline
\end{longtable}

\subsubsection{Use Case UC-MD08-02: Xem Chi tiết Yêu cầu Món ăn trên KDS/Phiếu in}
\begin{longtable}{|m{4cm}|p{11cm}|}
\caption{Đặc tả Use Case UC-MD08-02: Xem Chi tiết Yêu cầu Món ăn trên KDS/Phiếu in} \label{tab:uc_md08_02_final_v2} \\
\hline
\multicolumn{2}{|c|}{\textbf{2.1. Tóm tắt (Summary)}} \\
\hline
\textbf{Mục} & \textbf{Nội dung} \\
\hline
\endhead % Header cho các trang tiếp theo
\hline
\endfoot % Footer cho bảng
\hline
\endlastfoot % Footer cho trang cuối cùng
Use Case Name & Xem Chi tiết Yêu cầu Món ăn trên KDS/Phiếu in \\
\hline
Use Case ID & UC-MD08-02 \\
\hline
Use Case Description & Cho phép Nhân viên bếp (US-04) đọc và hiểu rõ các thông tin chi tiết liên quan đến từng món ăn hoặc đồ uống cụ thể cần được chuẩn bị, bao gồm tên món, số lượng, các tùy chọn biến thể đã được khách hàng chọn, và bất kỳ ghi chú đặc biệt nào từ khách hàng hoặc nhân viên phục vụ. Thông tin này được xem trên KDS hoặc trên phiếu in bếp. \\
\hline
Actor & US-04 (Nhân viên bếp) \\
\hline
Priority & Must Have \\
\hline
Trigger & - Sau khi Nhân viên bếp nhận biết có đơn hàng/phiếu mới (UC-MD08-01). \newline - Trước hoặc trong khi bắt đầu quá trình chế biến một món ăn. \\
\hline
Pre-Condition & - Một đơn hàng/phiếu yêu cầu chuẩn bị món đã được hiển thị trên KDS hoặc đã được in ra từ máy in bếp. \\
\hline
Post-Condition & - Nhân viên bếp nắm được đầy đủ và chính xác các yêu cầu cụ thể cho việc chế biến từng món ăn. \newline - Nhân viên bếp có đủ thông tin để tiến hành chế biến món ăn đúng theo yêu cầu. \\
\hline
\multicolumn{2}{|c|}{\textbf{2.2. Luồng thực thi (Flow)}} \\
\hline
\textbf{Mục} & \textbf{Nội dung} \\
\hline
Basic Flow (Sử dụng KDS) & 1. Nhân viên bếp (US-04) đang xem một đơn hàng/phiếu cụ thể trên màn hình KDS (đã chọn từ danh sách ở UC-MD08-01 hoặc KDS tự động hiển thị đơn tiếp theo). \newline 2. Đối với mỗi dòng món ăn (item) trong đơn hàng/phiếu đó, giao diện KDS hiển thị các thông tin chi tiết sau: \newline    - \textbf{Số lượng cần chuẩn bị} (Quantity). \newline    - \textbf{Tên món ăn/đồ uống} (Product Name). \newline    - \textbf{Các tùy chọn Biến thể} (Product Variants) đã được khách hàng hoặc nhân viên POS chọn (ví dụ: "Size: Lớn", "Độ chín: Tái vừa", "Không đường", "Ít đá"). \newline    - \textbf{Các Ghi chú đặc biệt} (Kitchen Notes / Special Requests) do nhân viên POS nhập (ví dụ: "Không hành", "Dị ứng đậu phộng", "Món này làm cho trẻ em, không gia vị cay"). \newline 3. US-04 đọc kỹ và phân tích tất cả các thông tin chi tiết này cho từng món ăn để hiểu rõ yêu cầu chế biến. \\
\hline
Alternative Flow & \textbf{Basic Flow (Sử dụng Phiếu in Bếp):} \newline    1. Nhân viên bếp (US-04) đang cầm trên tay phiếu in bếp (KOT) đã nhận được (từ UC-MD08-01). \newline    2. US-04 đọc kỹ các thông tin được in trên phiếu cho từng dòng món ăn, bao gồm các thông tin tương tự như được liệt kê ở bước 2 của Basic Flow (Sử dụng KDS). \newline \textbf{2a. Nhấp/Chạm vào món ăn trên KDS để xem thêm chi tiết (nếu có):} \newline    1. Nếu giao diện KDS ban đầu chỉ hiển thị thông tin tóm tắt cho mỗi món. \newline    2. US-04 có thể nhấp hoặc chạm vào một dòng món ăn cụ thể. \newline    3. Hệ thống KDS hiển thị một cửa sổ popup hoặc một khu vực chi tiết hơn, trình bày đầy đủ tất cả các biến thể và ghi chú dài (nếu có) cho món ăn đó. \\
\hline
Exception Flow & \textbf{2a. Thông tin món ăn bị thiếu, không rõ ràng hoặc mâu thuẫn trên KDS/Phiếu in:} \newline    1. Do lỗi truyền dữ liệu từ POS, lỗi định dạng mẫu in, hoặc lỗi hiển thị của KDS, một số thông tin quan trọng (ví dụ: biến thể, ghi chú dị ứng) bị thiếu, hiển thị không chính xác, hoặc có vẻ mâu thuẫn. \newline    2. Nhân viên bếp không có đủ thông tin hoặc không chắc chắn về cách chế biến. \newline    3. US-04 phải tạm dừng việc chuẩn bị món đó và liên hệ ngay với Nhân viên phục vụ đã gửi đơn (hoặc Quản lý) để làm rõ yêu cầu trước khi tiếp tục. Việc này nhằm tránh chế biến sai, gây lãng phí hoặc ảnh hưởng đến sức khỏe khách hàng. \\
\hline
\multicolumn{2}{|c|}{\textbf{2.3. Thông tin bổ sung (Additional Information)}} \\
\hline
\textbf{Mục} & \textbf{Nội dung} \\
\hline
Business Rule & - \textbf{BR-UC8.2-1 (System):} Hệ thống (POS và KDS/Máy in) phải đảm bảo truyền tải đầy đủ và chính xác tất cả các thông tin chi tiết liên quan đến việc chế biến món ăn: tên món, số lượng, tất cả các tùy chọn biến thể đã chọn, và toàn bộ nội dung các ghi chú đặc biệt. \newline - \textbf{BR-UC8.2-2:} Các thông tin đặc biệt quan trọng như ghi chú về dị ứng thực phẩm hoặc các yêu cầu sức khỏe nghiêm ngặt phải được làm nổi bật một cách rõ ràng trên KDS hoặc phiếu in (ví dụ: bằng màu sắc khác, font chữ lớn hơn, biểu tượng cảnh báo) để Nhân viên bếp không thể bỏ qua. \newline - \textbf{BR-UC8.2-3:} Thứ tự các món ăn trong một đơn hàng/phiếu trên KDS hoặc phiếu in nên phản ánh thứ tự khách gọi hoặc thứ tự ưu tiên phục vụ (ví dụ: khai vị trước, món chính sau), nếu thông tin này được POS gửi sang. \\
\hline
Non-Functional Requirement & - \textbf{NFR-UC8.2-1 (Clarity \& Readability):} Thông tin chi tiết món ăn trên KDS hoặc phiếu in phải được trình bày cực kỳ rõ ràng, dễ đọc, dễ hiểu. Font chữ phải đủ lớn, bố cục phải logic, tránh gây nhầm lẫn cho nhân viên bếp trong môi trường làm việc bận rộn. \newline - \textbf{NFR-UC8.2-2 (Accuracy):} Dữ liệu chi tiết món ăn hiển thị/in ra phải khớp 100\% với dữ liệu đã được Nhân viên POS nhập và gửi đi. Không được có sai lệch hay mất mát thông tin. \\
\hline
\end{longtable}

\subsubsection{Use Case UC-MD08-03: Cập nhật Trạng thái Chế biến Món ăn trên KDS}
\begin{longtable}{|m{4cm}|p{11cm}|}
\caption{Đặc tả Use Case UC-MD08-03: Cập nhật Trạng thái Chế biến Món ăn trên KDS} \label{tab:uc_md08_03_final_v2} \\
\hline
\multicolumn{2}{|c|}{\textbf{2.1. Tóm tắt (Summary)}} \\
\hline
\textbf{Mục} & \textbf{Nội dung} \\
\hline
\endhead % Header cho các trang tiếp theo
\hline
\endfoot % Footer cho bảng
\hline
\endlastfoot % Footer cho trang cuối cùng
Use Case Name & Cập nhật Trạng thái Chế biến Món ăn trên KDS \\
\hline
Use Case ID & UC-MD08-03 \\
\hline
Use Case Description & Cho phép Nhân viên bếp (US-04) tương tác trực tiếp với màn hình KDS (thường là màn hình cảm ứng) để đánh dấu và cập nhật trạng thái chế biến của từng món ăn cụ thể trong một đơn hàng/phiếu, ví dụ: chuyển từ "Đang chờ" sang "Đang làm" (In Progress/Cooking) hoặc từ "Đang làm" sang "Đã xong" (Completed/Ready). \\
\hline
Actor & US-04 (Nhân viên bếp) \\
\hline
Priority & Must Have (Nếu sử dụng KDS và cần theo dõi tiến độ từng món) \\
\hline
Trigger & - Nhân viên bếp bắt đầu thực hiện chế biến một món ăn cụ thể. \newline - Nhân viên bếp vừa hoàn thành việc chế biến một món ăn cụ thể. \\
\hline
Pre-Condition & - Nhân viên bếp đang xem chi tiết một đơn hàng/phiếu trên KDS (UC-MD08-02). \newline - Món ăn cần cập nhật trạng thái đang ở một trạng thái cho phép chuyển đổi (ví dụ: không thể đánh dấu "Đã xong" nếu chưa "Bắt đầu làm", tùy quy trình định nghĩa). \newline - Giao diện KDS hỗ trợ các hành động tương tác để thay đổi trạng thái món ăn. \\
\hline
Post-Condition & - Trạng thái của món ăn được chọn trên giao diện KDS được cập nhật thành trạng thái mới (ví dụ: đổi màu, có biểu tượng mới, hoặc di chuyển sang một cột/khu vực khác tương ứng với trạng thái mới). \newline - (Tùy chọn, nếu FR-MD08-06 được triển khai) Thông tin về trạng thái mới của món ăn ("Đã xong") có thể được hệ thống KDS gửi cập nhật về lại hệ thống POS để Nhân viên phục vụ biết. \newline - Giúp các nhân viên khác trong bếp (và có thể cả quản lý) theo dõi được tiến độ chế biến của từng món. \\
\hline
\multicolumn{2}{|c|}{\textbf{2.2. Luồng thực thi (Flow)}} \\
\hline
\textbf{Mục} & \textbf{Nội dung} \\
\hline
Basic Flow (Đánh dấu món "Đang làm") & 1. Nhân viên bếp (US-04) đang xem một đơn hàng/phiếu trên KDS và quyết định bắt đầu chế biến một món ăn cụ thể (ví dụ: "Pizza Hải Sản") trong đơn hàng/phiếu đó. \newline 2. US-04 chạm vào dòng món "Pizza Hải Sản" trên màn hình KDS. \newline 3. Giao diện KDS hiển thị các tùy chọn hành động cho món ăn đó, bao gồm nút "Bắt đầu làm" / "Start Cooking" / "In Progress". \newline 4. US-04 nhấp/chạm vào nút "Bắt đầu làm". \newline 5. Hệ thống KDS cập nhật trạng thái của món "Pizza Hải Sản" thành "Đang làm". Điều này có thể được biểu thị bằng cách: \newline    - Thay đổi màu nền của dòng món đó. \newline    - Hiển thị một biểu tượng "đang nấu". \newline    - Hoặc di chuyển dòng món đó sang một cột/khu vực "Đang làm" trên KDS (nếu KDS có nhiều cột trạng thái). \\
\hline
Alternative Flow & \textbf{Basic Flow (Đánh dấu món "Đã xong"):} \newline    1. Sau khi Nhân viên bếp (US-04) đã hoàn thành việc chế biến món "Pizza Hải Sản" (đang ở trạng thái "Đang làm"). \newline    2. US-04 chạm lại vào dòng món "Pizza Hải Sản" trên KDS. \newline    3. Giao diện KDS hiển thị các tùy chọn hành động, bao gồm nút "Hoàn thành" / "Done" / "Ready". \newline    4. US-04 nhấp/chạm vào nút "Hoàn thành". \newline    5. Hệ thống KDS cập nhật trạng thái của món "Pizza Hải Sản" thành "Đã xong". Điều này có thể được biểu thị bằng cách: \newline       - Thay đổi màu nền (ví dụ: sang màu xanh lá). \newline       - Gạch ngang tên món hoặc hiển thị biểu tượng "hoàn thành". \newline       - Di chuyển dòng món đó sang cột/khu vực "Đã xong" hoặc "Chờ phục vụ". \newline    6. (Nếu FR-MD08-06 được triển khai) Hệ thống KDS có thể tự động gửi thông báo về POS rằng món "Pizza Hải Sản" đã sẵn sàng. \newline \textbf{4a. Hoàn tác trạng thái đã cập nhật:} \newline    1. Nếu Nhân viên bếp vô tình đánh dấu sai trạng thái cho một món ăn (ví dụ: đánh dấu "Đã xong" nhưng thực tế chưa xong). \newline    2. US-04 có thể chạm lại vào món đó và chọn một tùy chọn "Hoàn tác" / "Undo" / "Chuyển về Đang làm" để đưa món ăn về lại trạng thái trước đó. \\
\hline
Exception Flow & \textbf{4a. Lỗi hệ thống KDS khi cập nhật trạng thái món ăn:} \newline    1. Hệ thống KDS gặp lỗi kỹ thuật (ví dụ: lỗi phần mềm, lỗi lưu trữ cục bộ nếu KDS có cơ chế offline) khi Nhân viên bếp cố gắng cập nhật trạng thái. \newline    2. KDS có thể hiển thị thông báo lỗi hoặc trạng thái của món ăn không được cập nhật đúng cách trên màn hình. \newline    3. Nhân viên bếp có thể cần thử lại hoặc thông báo cho bộ phận kỹ thuật/quản lý. \newline \textbf{Alternative Flow 6a. Lỗi hệ thống KDS khi gửi thông báo về POS:} \newline    1. KDS gặp lỗi khi cố gắng gửi thông tin cập nhật trạng thái "Đã xong" về hệ thống backend hoặc POS client. \newline    2. Trạng thái trên POS có thể không được cập nhật, mặc dù món đã xong trên KDS. \\
\hline
\multicolumn{2}{|c|}{\textbf{2.3. Thông tin bổ sung (Additional Information)}} \\
\hline
\textbf{Mục} & \textbf{Nội dung} \\
\hline
Business Rule & - \textbf{BR-UC8.3-1:} Hệ thống KDS phải cung cấp một cách thức rõ ràng và dễ dàng để Nhân viên bếp có thể cập nhật trạng thái chế biến cho từng món ăn (hoặc cho cả đơn hàng/phiếu). \newline - \textbf{BR-UC8.3-2:} Các trạng thái chế biến khả dụng (ví dụ: "Chờ xử lý", "Đang làm", "Đã xong", "Gặp vấn đề") và luồng chuyển đổi giữa các trạng thái này cần được định nghĩa và cấu hình cho phù hợp với quy trình làm việc thực tế của từng khu vực bếp/bar. \newline - \textbf{BR-UC8.3-3:} Việc cập nhật trạng thái trên KDS phải phản ánh chính xác tiến độ chế biến thực tế của món ăn để các nhân viên khác trong bếp và (nếu có FR-MD08-06) nhân viên phục vụ có thông tin đúng. \\
\hline
Non-Functional Requirement & - \textbf{NFR-UC8.3-1 (Usability):} Các thao tác tương tác trên màn hình KDS (chạm, vuốt nếu có) để thay đổi trạng thái món ăn phải nhạy, chính xác và dễ thực hiện, ngay cả trong môi trường bếp bận rộn và nhân viên có thể đang đeo găng tay (nếu màn hình cảm ứng hỗ trợ). \newline - \textbf{NFR-UC8.3-2 (Performance):} Phản hồi của hệ thống KDS khi Nhân viên bếp chạm để cập nhật trạng thái và việc thay đổi hiển thị trên màn hình phải diễn ra tức thời, không có độ trễ. \newline - \textbf{NFR-UC8.3-3 (Reliability):} Việc cập nhật và lưu trữ trạng thái món ăn trên KDS (và đồng bộ về backend nếu có) phải đáng tin cậy, không bị mất dữ liệu trạng thái khi có sự cố mạng tạm thời (KDS nên có khả năng hoạt động offline ở mức độ nhất định và đồng bộ lại sau). \\
\hline
\end{longtable}

\subsubsection{Use Case UC-MD08-04: Đánh dấu Hoàn thành Toàn bộ Đơn hàng/Phiếu trên KDS}
\begin{longtable}{|m{4cm}|p{11cm}|}
\caption{Đặc tả Use Case UC-MD08-04: Đánh dấu Hoàn thành Toàn bộ Đơn hàng/Phiếu trên KDS} \label{tab:uc_md08_04_final_v2} \\
\hline
\multicolumn{2}{|c|}{\textbf{2.1. Tóm tắt (Summary)}} \\
\hline
\textbf{Mục} & \textbf{Nội dung} \\
\hline
\endhead % Header cho các trang tiếp theo
\hline
\endfoot % Footer cho bảng
\hline
\endlastfoot % Footer cho trang cuối cùng
Use Case Name & Đánh dấu Hoàn thành Toàn bộ Đơn hàng/Phiếu trên KDS \\
\hline
Use Case ID & UC-MD08-04 \\
\hline
Use Case Description & Cho phép Nhân viên bếp (US-04) đánh dấu rằng tất cả các món ăn trong một đơn hàng/phiếu cụ thể hiển thị trên KDS đã được chuẩn bị xong và sẵn sàng để được phục vụ hoặc đóng gói. \\
\hline
Actor & US-04 (Nhân viên bếp) \\
\hline
Priority & Must Have (Nếu sử dụng KDS và quản lý theo đơn/phiếu) \\
\hline
Trigger & Nhân viên bếp đã hoàn thành việc chế biến tất cả các món ăn thuộc một đơn hàng/phiếu yêu cầu trên KDS. \\
\hline
Pre-Condition & - Nhân viên bếp đang xem đơn hàng/phiếu trên KDS (UC-MD08-02). \newline - Tất cả các món ăn riêng lẻ trong đơn hàng/phiếu đó đã được hoàn thành (có thể đã được đánh dấu "Đã xong" ở UC-MD08-03, hoặc nhân viên xác nhận hoàn thành tất cả cùng lúc). \newline - Giao diện KDS hỗ trợ hành động hoàn thành toàn bộ đơn hàng/phiếu. \\
\hline
Post-Condition & - Toàn bộ đơn hàng/phiếu trên KDS được cập nhật sang trạng thái "Đã hoàn thành" / "Sẵn sàng". \newline - Đơn hàng/phiếu đó có thể biến mất khỏi danh sách các đơn hàng đang chờ xử lý hoặc di chuyển sang một khu vực/cột "Đã xong" trên KDS. \newline - (Tùy chọn, nếu FR-MD08-06 được triển khai) Thông báo về việc toàn bộ đơn hàng đã sẵn sàng có thể được gửi đến hệ thống POS. \\
\hline
\multicolumn{2}{|c|}{\textbf{2.2. Luồng thực thi (Flow)}} \\
\hline
\textbf{Mục} & \textbf{Nội dung} \\
\hline
Basic Flow & 1. Nhân viên bếp (US-04) đã xác nhận rằng tất cả các món ăn trong một đơn hàng/phiếu (ví dụ: Phiếu Bếp \#123) hiển thị trên KDS đã được chuẩn bị xong. \newline 2. US-04 chạm vào đơn hàng/phiếu Bếp \#123 trên màn hình KDS. \newline 3. Giao diện KDS hiển thị các tùy chọn hành động cho toàn bộ đơn hàng/phiếu đó, bao gồm một nút nổi bật "Hoàn thành Đơn" / "Mark Order as Ready" / "Done". \newline 4. US-04 nhấp/chạm vào nút "Hoàn thành Đơn". \newline 5. Hệ thống KDS cập nhật trạng thái của toàn bộ đơn hàng/phiếu Bếp \#123 thành "Đã hoàn thành". \newline 6. Đơn hàng/phiếu Bếp \#123 biến mất khỏi danh sách các đơn hàng đang chờ xử lý chính hoặc được di chuyển sang một khu vực "Đã hoàn thành" trên KDS. \newline 7. (Nếu FR-MD08-06 được triển khai) Hệ thống KDS có thể tự động gửi một thông báo tổng thể đến POS rằng tất cả các món của đơn hàng/bàn tương ứng đã sẵn sàng. \\
\hline
Alternative Flow & \textbf{1a. Tự động hoàn thành đơn khi tất cả món đã xong:} \newline    1. Nếu KDS được cấu hình để quản lý trạng thái từng món riêng lẻ (UC-MD08-03). \newline    2. Khi Nhân viên bếp đánh dấu món ăn cuối cùng trong một đơn hàng/phiếu là "Đã xong". \newline    3. Hệ thống KDS có thể tự động nhận diện rằng tất cả các món của đơn hàng/phiếu đó đã hoàn thành và tự động thực hiện các bước 5, 6, 7 của Basic Flow mà không cần Nhân viên bếp phải nhấn nút "Hoàn thành Đơn" riêng. \\
\hline
Exception Flow & \textbf{4a. Lỗi hệ thống KDS khi cập nhật trạng thái đơn hàng/phiếu:} \newline    1. Hệ thống KDS gặp lỗi kỹ thuật khi Nhân viên bếp cố gắng đánh dấu hoàn thành toàn bộ đơn hàng/phiếu. \newline    2. KDS có thể hiển thị thông báo lỗi hoặc trạng thái của đơn hàng/phiếu không được cập nhật đúng cách. \newline    3. Nhân viên bếp có thể cần thử lại hoặc thông báo cho bộ phận kỹ thuật/quản lý. \\
\hline
\multicolumn{2}{|c|}{\textbf{2.3. Thông tin bổ sung (Additional Information)}} \\
\hline
\textbf{Mục} & \textbf{Nội dung} \\
\hline
Business Rule & - \textbf{BR-UC8.4-1 (V3):} Hành động đánh dấu hoàn thành toàn bộ đơn hàng/phiếu chỉ nên được thực hiện khi tất cả các món ăn trong đó đã thực sự được chuẩn bị xong. \newline - \textbf{BR-UC8.4-2 (V3):} Sau khi một đơn hàng/phiếu được đánh dấu hoàn thành trên KDS, nó không nên xuất hiện trong danh sách các công việc cần làm chính nữa để tránh nhầm lẫn. Nó có thể được lưu vào một danh sách "Đã hoàn thành" để tra cứu nếu cần. \\
\hline
Non-Functional Requirement & - \textbf{NFR-UC8.4-1 (V3) (Usability):} Nút hoặc hành động để đánh dấu hoàn thành toàn bộ đơn hàng/phiếu phải rõ ràng và dễ thao tác. \newline - \textbf{NFR-UC8.4-2 (V3) (Performance):} Phản hồi của KDS khi đánh dấu hoàn thành phải nhanh chóng. \\
\hline
\end{longtable}

\subsubsection{Use Case UC-MD08-05: (Tùy chọn) Sắp xếp/Đánh dấu Ưu tiên Đơn hàng trên KDS}
\begin{longtable}{|m{4cm}|p{11cm}|}
\caption{Đặc tả Use Case UC-MD08-05: (Tùy chọn) Sắp xếp/Đánh dấu Ưu tiên Đơn hàng trên KDS} \label{tab:uc_md08_05_final_v2} \\
\hline
\multicolumn{2}{|c|}{\textbf{2.1. Tóm tắt (Summary)}} \\
\hline
\textbf{Mục} & \textbf{Nội dung} \\
\hline
\endhead % Header cho các trang tiếp theo
\hline
\endfoot % Footer cho bảng
\hline
\endlastfoot % Footer cho trang cuối cùng
Use Case Name & (Tùy chọn) Sắp xếp/Đánh dấu Ưu tiên Đơn hàng trên KDS \\
\hline
Use Case ID & UC-MD08-05 \\
\hline
Use Case Description & Cung cấp khả năng cho Nhân viên bếp (US-04) sắp xếp lại thứ tự hiển thị của các đơn hàng/phiếu trên KDS hoặc đánh dấu một số đơn hàng/phiếu là ưu tiên cần xử lý trước, dựa trên các yếu tố như thời gian chờ của khách, yêu cầu đặc biệt hoặc chỉ đạo của quản lý/bếp trưởng. \\
\hline
Actor & US-04 (Nhân viên bếp) \\
\hline
Priority & Low / Nice to Have \\
\hline
Trigger & - Bếp nhận được nhiều đơn hàng gần như cùng lúc và cần sắp xếp lại thứ tự làm việc để tối ưu hóa hoặc đáp ứng yêu cầu. \newline - Có một đơn hàng cụ thể cần được ưu tiên xử lý trước các đơn khác (ví dụ: khách VIP, khách phàn nàn đợi lâu, món cần thời gian chuẩn bị đặc biệt). \\
\hline
Pre-Condition & - Nhân viên bếp đang xem danh sách các đơn hàng/phiếu đang chờ xử lý trên KDS (UC-MD08-01). \newline - Giao diện KDS được thiết kế để hỗ trợ chức năng sắp xếp thủ công hoặc đánh dấu ưu tiên cho đơn hàng/phiếu. \\
\hline
Post-Condition & - Thứ tự hiển thị của các đơn hàng/phiếu trên KDS được thay đổi theo ý muốn của nhân viên. \newline - Các đơn hàng/phiếu được đánh dấu ưu tiên có một dấu hiệu nhận biết trực quan rõ ràng (ví dụ: màu sắc khác, biểu tượng đặc biệt, vị trí nổi bật hơn). \\
\hline
\multicolumn{2}{|c|}{\textbf{2.2. Luồng thực thi (Flow)}} \\
\hline
\textbf{Mục} & \textbf{Nội dung} \\
\hline
Basic Flow (Đánh dấu ưu tiên một đơn hàng/phiếu) & 1. Nhân viên bếp (US-04) xác định một đơn hàng/phiếu cụ thể trên KDS cần được ưu tiên xử lý. \newline 2. US-04 chạm vào đơn hàng/phiếu đó trên màn hình KDS. \newline 3. Giao diện KDS hiển thị các tùy chọn hành động, trong đó có nút hoặc tùy chọn "Đánh dấu Ưu tiên" (Mark as Priority / Prioritize). \newline 4. US-04 nhấp/chạm vào tùy chọn "Đánh dấu Ưu tiên". \newline 5. Hệ thống KDS cập nhật trạng thái ưu tiên cho đơn hàng/phiếu đó. Đơn hàng/phiếu này giờ đây được hiển thị với một dấu hiệu trực quan cho biết nó là ưu tiên (ví dụ: đổi sang màu đỏ, có biểu tượng cờ, hoặc tự động di chuyển lên vị trí đầu danh sách các đơn hàng chờ). \\
\hline
Alternative Flow & \textbf{1a. Sắp xếp lại thứ tự hiển thị của các đơn hàng/phiếu (ví dụ, bằng cách kéo thả):} \newline    1. Nếu giao diện KDS hỗ trợ, US-04 có thể nhấn giữ vào một đơn hàng/phiếu. \newline    2. US-04 kéo đơn hàng/phiếu đó đến một vị trí mới mong muốn trong danh sách các đơn hàng đang chờ. \newline    3. US-04 thả tay ra. \newline    4. Hệ thống KDS cập nhật lại thứ tự hiển thị của các đơn hàng/phiếu trên màn hình. \newline \textbf{1b. Sắp xếp danh sách theo một tiêu chí có sẵn:} \newline    1. Giao diện KDS có thể cung cấp các nút/tùy chọn để tự động sắp xếp toàn bộ danh sách đơn hàng/phiếu theo các tiêu chí khác nhau (ví dụ: "Sắp xếp theo Thời gian chờ lâu nhất", "Sắp xếp theo Thời gian gửi đơn mới nhất", "Sắp xếp theo Loại đơn hàng"...). \newline    2. US-04 chọn một tiêu chí sắp xếp. \newline    3. Hệ thống KDS tự động sắp xếp lại toàn bộ danh sách theo tiêu chí đó. \\
\hline
Exception Flow & \textbf{4a (Basic) / 3a (Alt 1a) / 2a (Alt 1b). Lỗi hệ thống KDS khi cố gắng sắp xếp hoặc đánh dấu ưu tiên:} \newline    1. Hệ thống KDS gặp lỗi kỹ thuật khi Nhân viên bếp thực hiện hành động sắp xếp hoặc đánh dấu ưu tiên. \newline    2. Thao tác có thể không thành công, hoặc thứ tự/trạng thái ưu tiên không được cập nhật đúng cách trên màn hình. KDS có thể hiển thị thông báo lỗi. \\
\hline
\multicolumn{2}{|c|}{\textbf{2.3. Thông tin bổ sung (Additional Information)}} \\
\hline
\textbf{Mục} & \textbf{Nội dung} \\
\hline
Business Rule & - \textbf{BR-UC8.5-1:} Chức năng sắp xếp hoặc đánh dấu ưu tiên đơn hàng/phiếu trên KDS là một tính năng tùy chọn, không bắt buộc phải có trong mọi triển khai KDS. \newline - \textbf{BR-UC8.5-2:} Nếu có chức năng đánh dấu ưu tiên, các đơn hàng/phiếu được ưu tiên phải có một biểu hiện trực quan rõ ràng để tất cả các nhân viên bếp khác có thể dễ dàng nhận biết và xử lý phù hợp. \newline - \textbf{BR-UC8.5-3:} Việc sắp xếp lại thứ tự hoặc đánh dấu ưu tiên trên KDS là do nhân viên bếp chủ động thực hiện và chịu trách nhiệm, không nhất thiết phải đồng bộ ngược lại trạng thái ưu tiên này về phía POS (trừ khi có yêu cầu nghiệp vụ đặc biệt). \\
\hline
Non-Functional Requirement & - \textbf{NFR-UC8.5-1 (Usability):} Các thao tác sắp xếp (ví dụ: kéo thả) hoặc đánh dấu ưu tiên phải dễ dàng và trực quan để thực hiện trên màn hình cảm ứng của KDS. \newline - \textbf{NFR-UC8.5-2 (Performance):} Việc sắp xếp lại danh sách hoặc cập nhật trạng thái ưu tiên của một đơn hàng/phiếu trên KDS phải có hiệu lực và được phản ánh trên màn hình ngay lập tức. \\
\hline
\end{longtable}

\subsubsection{Use Case UC-MD08-06: (Tùy chọn) Xem Cập nhật Trạng thái Món ăn từ Bếp trên POS}
\begin{longtable}{|m{4cm}|p{11cm}|}
\caption{Đặc tả Use Case UC-MD08-06: (Tùy chọn) Xem Cập nhật Trạng thái Món ăn từ Bếp trên POS} \label{tab:uc_md08_06_final_v2} \\
\hline
\multicolumn{2}{|c|}{\textbf{2.1. Tóm tắt (Summary)}} \\
\hline
\textbf{Mục} & \textbf{Nội dung} \\
\hline
\endhead % Header cho các trang tiếp theo
\hline
\endfoot % Footer cho bảng
\hline
\endlastfoot % Footer cho trang cuối cùng
Use Case Name & (Tùy chọn) Xem Cập nhật Trạng thái Món ăn từ Bếp trên POS \\
\hline
Use Case ID & UC-MD08-06 \\
\hline
Use Case Description & Cho phép Nhân viên Phục vụ (US-02) hoặc Thu ngân (US-05) xem được thông tin cập nhật về trạng thái chế biến của các món ăn (ví dụ: món nào đã được bếp chuẩn bị xong và sẵn sàng để mang ra phục vụ khách) trực tiếp trên giao diện đơn hàng POS. Thông tin này được đồng bộ từ KDS (sau khi Nhân viên bếp thực hiện UC-MD08-03 hoặc UC-MD08-04). \\
\hline
Actor & US-02 (Nhân viên phục vụ), US-05 (Nhân viên thu ngân) \\
\hline
Priority & Nice to Have \\
\hline
Trigger & - Nhân viên phục vụ muốn kiểm tra xem các món ăn của một bàn cụ thể đã được bếp chuẩn bị xong chưa để mang ra cho khách. \newline - Hệ thống KDS gửi tín hiệu cập nhật trạng thái món ăn về POS. \\
\hline
Pre-Condition & - Nhà hàng sử dụng hệ thống KDS (thay vì chỉ máy in bếp). \newline - Chức năng đồng bộ trạng thái món ăn hai chiều giữa KDS và POS đã được kích hoạt và cấu hình trong hệ thống. \newline - Nhân viên phục vụ/thu ngân đang xem chi tiết một đơn hàng trên POS mà đơn hàng đó có các món đã được gửi xuống KDS. \\
\hline
Post-Condition & - Giao diện đơn hàng trên POS hiển thị chỉ báo trực quan về trạng thái mới nhất của từng món ăn (ví dụ: "Đang làm", "Đã xong", "Sẵn sàng phục vụ"). \newline - Nhân viên phục vụ có thể dựa vào thông tin này để chủ động lấy món từ bếp và phục vụ khách kịp thời. \\
\hline
\multicolumn{2}{|c|}{\textbf{2.2. Luồng thực thi (Flow)}} \\
\hline
\textbf{Mục} & \textbf{Nội dung} \\
\hline
Basic Flow (Nhân viên chủ động xem) & 1. Nhân viên phục vụ (US-02) hoặc Thu ngân (US-05) đang xem chi tiết một đơn hàng trên giao diện POS. \newline 2. Đối với mỗi dòng món ăn trong đơn hàng đã được gửi xuống bếp/KDS, hệ thống POS hiển thị một chỉ báo về trạng thái chế biến hiện tại của món đó (ví dụ: một biểu tượng nhỏ, một dòng chữ trạng thái, hoặc màu sắc khác nhau của dòng món). \newline 3. Các trạng thái có thể bao gồm: "Đã gửi bếp/Chờ xử lý", "Bếp đang làm", "Đã xong/Sẵn sàng". \newline 4. Nhân viên xem các chỉ báo này để biết món nào đã có thể mang ra phục vụ. \\
\hline
Alternative Flow & \textbf{2a. Nhận thông báo tự động trên POS khi món sẵn sàng:} \newline    1. Khi một món ăn được Nhân viên bếp đánh dấu là "Đã xong" trên KDS (UC-MD08-03/UC-MD08-04) và hệ thống KDS gửi cập nhật về. \newline    2. Giao diện POS client của nhân viên đang phụ trách đơn hàng đó (hoặc tất cả các POS client, tùy cấu hình) nhận được tín hiệu cập nhật. \newline    3. Giao diện POS tự động cập nhật trạng thái của dòng món ăn tương ứng thành "Đã xong/Sẵn sàng". \newline    4. (Tùy chọn) Hệ thống POS có thể phát ra một âm thanh thông báo ngắn hoặc hiển thị một popup/thông báo nhỏ trên màn hình (ví dụ: "Món [Tên món] của Bàn [Số bàn] đã sẵn sàng!") để thu hút sự chú ý của nhân viên. \\
\hline
Exception Flow & \textbf{2a. Lỗi đồng bộ trạng thái từ KDS về POS:} \newline    1. Do lỗi kết nối mạng giữa KDS và máy chủ, hoặc giữa máy chủ và POS client, hoặc lỗi xử lý logic cập nhật. \newline    2. Trạng thái món ăn hiển thị trên POS không được cập nhật hoặc cập nhật chậm trễ so với trạng thái thực tế trên KDS. \newline    3. Nhân viên phục vụ có thể không biết món đã xong để lấy. Cần có quy trình dự phòng (ví dụ: bếp thông báo miệng, hoặc nhân viên phục vụ chủ động hỏi bếp). \\
\hline
\multicolumn{2}{|c|}{\textbf{2.3. Thông tin bổ sung (Additional Information)}} \\
\hline
\textbf{Mục} & \textbf{Nội dung} \\
\hline
Business Rule & - \textbf{BR-UC8.6-1 (System):} Chức năng đồng bộ trạng thái món ăn từ KDS về POS là một tính năng tùy chọn và cần được kích hoạt, cấu hình đúng cách trong hệ thống. \newline - \textbf{BR-UC8.6-2:} Chỉ những thay đổi trạng thái quan trọng từ KDS (đặc biệt là trạng thái "Đã xong" hoặc "Sẵn sàng phục vụ") mới nên được hệ thống chủ động thông báo hoặc cập nhật nổi bật trên POS để tránh làm quá tải thông tin cho nhân viên phục vụ. \newline - \textbf{BR-UC8.6-3:} Cách thức hiển thị trạng thái món ăn và cơ chế thông báo trên POS cho nhân viên phục vụ cần được thiết kế sao cho rõ ràng, dễ nhận biết nhưng không quá gây xao nhãng hoặc làm chậm thao tác trên POS. \\
\hline
Non-Functional Requirement & - \textbf{NFR-UC8.6-1 (Performance \& Near Real-time):} Việc cập nhật trạng thái món ăn từ KDS về giao diện POS nên diễn ra gần như trong thời gian thực (độ trễ thấp, ví dụ: vài giây) để nhân viên phục vụ có thông tin kịp thời nhất. \newline - \textbf{NFR-UC8.6-2 (Reliability):} Cơ chế giao tiếp và đồng bộ trạng thái giữa KDS, backend, và các POS client phải đảm bảo tính đáng tin cậy, giảm thiểu việc mất mát hoặc sai lệch thông tin cập nhật. \\
\hline
\end{longtable}

