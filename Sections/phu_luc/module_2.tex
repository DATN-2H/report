\subsection{Module MD-02: Quản lý Thực đơn \& Sản phẩm}

\subsubsection{Use Case UC-MD02-01: Tạo mới Sản phẩm}

\begin{longtable}{|m{4cm}|p{11cm}|}
\caption{Đặc tả Use Case UC-MD02-01: Tạo mới Sản phẩm} \label{tab:uc_md02_01_revised} \\
\hline
\multicolumn{2}{|c|}{\textbf{2.1. Tóm tắt (Summary)}} \\
\hline
\textbf{Mục} & \textbf{Nội dung} \\
\hline
\endhead % Header cho các trang tiếp theo
\hline
\endfoot % Footer cho bảng
\hline
\endlastfoot % Footer cho trang cuối cùng
Use Case Name & Tạo mới Sản phẩm \\
\hline
Use Case ID & UC-MD02-01 \\
\hline
Use Case Description & Cho phép Quản lý nhà hàng (US-01) thêm một món ăn, đồ uống, hoặc dịch vụ mới vào hệ thống với các chi tiết cần thiết ban đầu như tên, giá bán, và loại sản phẩm. \\
\hline
Actor & US-01 (Quản lý nhà hàng) \\
\hline
Priority & Must Have \\
\hline
Trigger & Nhà hàng cần bổ sung một món mới vào thực đơn, hoặc quản lý một dịch vụ/mặt hàng mới. \\
\hline
Pre-Condition & - US-01 đã đăng nhập vào hệ thống với quyền quản lý sản phẩm. \\
\hline
Post-Condition & - Một bản ghi Sản phẩm Gốc (Product Template) mới được tạo và lưu thành công trong cơ sở dữ liệu. \newline - Sản phẩm này sẵn sàng cho các cấu hình chi tiết hơn (biến thể, hiển thị POS, hình ảnh...). \newline - Hệ thống ghi nhận hoạt động. \\
\hline
\multicolumn{2}{|c|}{\textbf{2.2. Luồng thực thi (Flow)}} \\
\hline
\textbf{Mục} & \textbf{Nội dung} \\
\hline
Basic Flow & 1. US-01 truy cập vào chức năng quản lý Sản phẩm (ví dụ: Inventory > Products > Products hoặc Sales > Products > Products). \newline 2. US-01 chọn hành động "Tạo mới" (Create). \newline 3. Hệ thống hiển thị một form sản phẩm trống. \newline 4. US-01 nhập Tên sản phẩm (Product Name) (Bắt buộc - BR-UC2.1-1). \newline 5. US-01 chọn Loại sản phẩm (Product Type) từ danh sách thả xuống (Consumable, Stockable, Service) (Bắt buộc - BR-UC2.1-3). Đối với hầu hết món ăn/đồ uống nhà hàng không cần theo dõi tồn kho chi tiết, chọn "Consumable". \newline 6. US-01 nhập Giá bán (Sales Price) (Bắt buộc - BR-UC2.1-2). \newline 7. (Tùy chọn) US-01 nhập Giá vốn (Cost). \newline 8. (Tùy chọn) US-01 chọn Danh mục sản phẩm nội bộ (Internal Product Category). \newline 9. (Tùy chọn) US-01 nhập Mã nội bộ (Internal Reference). \newline 10. US-01 đảm bảo các tùy chọn mặc định như "Có thể Bán" (Can be Sold) và "Có thể Mua" (Can be Purchased - nếu áp dụng) được chọn đúng. \newline 11. US-01 chọn hành động "Lưu" (Save). \newline 12. Hệ thống kiểm tra các trường bắt buộc đã được điền và dữ liệu hợp lệ. \newline 13. Hệ thống lưu bản ghi sản phẩm mới. \newline 14. Hệ thống hiển thị form sản phẩm ở chế độ xem với thông tin vừa lưu và thông báo "Sản phẩm đã được tạo". \\
\hline
Alternative Flow & \textbf{11a. Lưu và Tạo mới (Save \& New):} \newline    1. US-01 chọn "Lưu và Tạo mới". \newline    2. Hệ thống thực hiện bước 12, 13. \newline    3. Hệ thống hiển thị lại form sản phẩm trống để tiếp tục nhập. \\
\hline
Exception Flow & \textbf{12a. Lỗi Xác thực Dữ liệu:} \newline    1. Hệ thống phát hiện thiếu Tên sản phẩm, Giá bán, hoặc Loại SP; hoặc Giá bán không phải số. \newline    2. Hệ thống hiển thị thông báo lỗi, chỉ rõ trường lỗi. \newline    3. Hệ thống không lưu, giữ nguyên form để US-01 sửa. Use Case quay lại bước 4. \newline \textbf{13a. Lỗi Hệ thống khi Lưu:} \newline    1. Hệ thống gặp sự cố khi lưu. \newline    2. Hệ thống hiển thị thông báo lỗi chung. Use Case kết thúc không thành công. \\
\hline
\multicolumn{2}{|c|}{\textbf{2.3. Thông tin bổ sung (Additional Information)}} \\
\hline
\textbf{Mục} & \textbf{Nội dung} \\
\hline
Business Rule & - \textbf{BR-UC2.1-1:} Tên sản phẩm là bắt buộc. \newline - \textbf{BR-UC2.1-2:} Giá bán là bắt buộc và phải là số không âm. \newline - \textbf{BR-UC2.1-3:} Loại sản phẩm là bắt buộc. "Consumable" cho món ăn/đồ uống không quản lý tồn kho chi tiết. "Stockable" cho hàng hóa cần theo dõi tồn kho (rượu chai, bia lon). "Service" cho phí dịch vụ. \newline - \textbf{BR-UC2.1-4:} Sản phẩm mới tạo mặc định "Active" và "Can be Sold". \\
\hline
Non-Functional Requirement & - \textbf{NFR-UC2.1-1 (Usability):} Form tạo sản phẩm rõ ràng, trường bắt buộc dễ nhận biết. \newline - \textbf{NFR-UC2.1-2 (Performance):} Lưu sản phẩm mới dưới 3 giây. \newline - \textbf{NFR-UC2.1-3 (Data Integrity):} Dữ liệu nhập phải được lưu chính xác. \\
\hline
\end{longtable}

\subsubsection{Use Case UC-MD02-02: Xem Danh sách Sản phẩm}

\begin{longtable}{|m{4cm}|p{11cm}|}
\caption{Đặc tả Use Case UC-MD02-02: Xem Danh sách Sản phẩm} \label{tab:uc_md02_02_revised} \\
\hline
\multicolumn{2}{|c|}{\textbf{2.1. Tóm tắt (Summary)}} \\
\hline
\textbf{Mục} & \textbf{Nội dung} \\
\hline
\endhead % Header cho các trang tiếp theo
\hline
\endfoot % Footer cho bảng
\hline
\endlastfoot % Footer cho trang cuối cùng
Use Case Name & Xem Danh sách Sản phẩm \\
\hline
Use Case ID & UC-MD02-02 \\
\hline
Use Case Description & Cho phép Quản lý nhà hàng (US-01) xem danh sách tất cả các sản phẩm (món ăn, đồ uống, dịch vụ) đã được tạo trong hệ thống, với các thông tin tóm tắt. \\
\hline
Actor & US-01 (Quản lý nhà hàng) \\
\hline
Priority & Must Have \\
\hline
Trigger & Quản lý nhà hàng cần kiểm tra các sản phẩm hiện có, tìm kiếm một sản phẩm cụ thể, hoặc chuẩn bị cho các thao tác quản lý khác (sửa, xóa, lưu trữ). \\
\hline
Pre-Condition & - US-01 đã đăng nhập vào hệ thống với quyền xem sản phẩm. \\
\hline
Post-Condition & - Danh sách các sản phẩm (mặc định là các sản phẩm đang hoạt động - Active) được hiển thị. \newline - US-01 có thể xem các thông tin cơ bản của từng sản phẩm và thực hiện các hành động tiếp theo. \\
\hline
\multicolumn{2}{|c|}{\textbf{2.2. Luồng thực thi (Flow)}} \\
\hline
\textbf{Mục} & \textbf{Nội dung} \\
\hline
Basic Flow & 1. US-01 truy cập vào chức năng quản lý Sản phẩm (ví dụ: Inventory > Products > Products). \newline 2. Hệ thống truy vấn và hiển thị danh sách các sản phẩm (Product Templates) đang hoạt động (Active=True). \newline 3. Với mỗi sản phẩm trong danh sách (List View), hệ thống hiển thị các thông tin cơ bản như: Hình ảnh (nếu có), Tên sản phẩm, Mã nội bộ (nếu có), Giá bán, Giá vốn (nếu có), Số lượng tồn kho dự báo (nếu là Stockable). \newline 4. US-01 xem xét danh sách. \\
\hline
Alternative Flow & \textbf{4a. Tìm kiếm sản phẩm:} \newline    1. US-01 sử dụng ô tìm kiếm, nhập tên, mã nội bộ hoặc một phần thông tin sản phẩm. \newline    2. Hệ thống lọc và hiển thị các sản phẩm khớp với từ khóa. \newline \textbf{4b. Lọc sản phẩm:} \newline    1. US-01 sử dụng các bộ lọc có sẵn (Filters) như "Có thể Bán", "Có thể Mua", "Loại sản phẩm (Consumable, Stockable, Service)", "Danh mục sản phẩm nội bộ", "Đã lưu trữ (Archived)". \newline    2. Hệ thống áp dụng bộ lọc và hiển thị kết quả. \newline \textbf{4c. Nhóm sản phẩm (Group By):} \newline    1. US-01 sử dụng chức năng nhóm (ví dụ: nhóm theo Loại sản phẩm, theo Danh mục sản phẩm nội bộ). \newline    2. Hệ thống hiển thị danh sách được nhóm lại. \newline \textbf{4d. Chuyển đổi dạng xem:} \newline    1. US-01 có thể chuyển sang dạng xem Kanban (thẻ sản phẩm với hình ảnh) hoặc các dạng xem khác nếu có. \\
\hline
Exception Flow & \textbf{2a. Lỗi tải danh sách:} \newline    1. Hệ thống gặp lỗi khi truy vấn dữ liệu sản phẩm. \newline    2. Hệ thống hiển thị thông báo lỗi. \newline \textbf{2b. Không có sản phẩm nào:} \newline    1. Nếu chưa có sản phẩm nào được tạo (hoặc không có sản phẩm nào khớp bộ lọc). \newline    2. Hệ thống hiển thị danh sách trống hoặc thông báo "Chưa có sản phẩm nào." \\
\hline
\multicolumn{2}{|c|}{\textbf{2.3. Thông tin bổ sung (Additional Information)}} \\
\hline
\textbf{Mục} & \textbf{Nội dung} \\
\hline
Business Rule & - \textbf{BR-UC2.2-1:} Mặc định, danh sách chỉ hiển thị các sản phẩm đang hoạt động (Active=True). Người dùng cần bỏ bộ lọc "Archived=False" để xem các sản phẩm đã lưu trữ. \\
\hline
Non-Functional Requirement & - \textbf{NFR-UC2.2-1 (Usability):} Giao diện danh sách phải rõ ràng, dễ đọc. Chức năng tìm kiếm, lọc, nhóm phải hiệu quả. \newline - \textbf{NFR-UC2.2-2 (Performance):} Thời gian tải danh sách sản phẩm (ngay cả với hàng trăm sản phẩm) phải nhanh chóng. \\
\hline
\end{longtable}

\subsubsection{Use Case UC-MD02-03: Xem Chi tiết Sản phẩm}

\begin{longtable}{|m{4cm}|p{11cm}|}
\caption{Đặc tả Use Case UC-MD02-03: Xem Chi tiết Sản phẩm} \label{tab:uc_md02_03_revised} \\
\hline
\multicolumn{2}{|c|}{\textbf{2.1. Tóm tắt (Summary)}} \\
\hline
\textbf{Mục} & \textbf{Nội dung} \\
\hline
\endhead % Header cho các trang tiếp theo
\hline
\endfoot % Footer cho bảng
\hline
\endlastfoot % Footer cho trang cuối cùng
Use Case Name & Xem Chi tiết Sản phẩm \\
\hline
Use Case ID & UC-MD02-03 \\
\hline
Use Case Description & Cho phép Quản lý nhà hàng (US-01) xem thông tin chi tiết đầy đủ của một sản phẩm cụ thể đã được chọn từ danh sách, bao gồm tất cả các tab thông tin (Chung, Biến thể, Bán hàng, Mua hàng, Tồn kho, Kế toán, POS...). \\
\hline
Actor & US-01 (Quản lý nhà hàng) \\
\hline
Priority & Must Have \\
\hline
Trigger & Quản lý nhà hàng nhấp vào một sản phẩm từ danh sách (UC-MD02-02) để xem hoặc chuẩn bị sửa đổi thông tin. \\
\hline
Pre-Condition & - US-01 đang xem danh sách sản phẩm (UC-MD02-02 thành công). \newline - US-01 có quyền xem chi tiết sản phẩm. \\
\hline
Post-Condition & - Form chi tiết (Product Template Form View) của sản phẩm được chọn được hiển thị. \newline - US-01 nắm được mọi thông tin đã cấu hình cho sản phẩm đó. \\
\hline
\multicolumn{2}{|c|}{\textbf{2.2. Luồng thực thi (Flow)}} \\
\hline
\textbf{Mục} & \textbf{Nội dung} \\
\hline
Basic Flow & 1. US-01 đang xem danh sách sản phẩm (UC-MD02-02). \newline 2. US-01 nhấp vào tên hoặc một vùng có thể nhấp được của dòng sản phẩm muốn xem chi tiết. \newline 3. Hệ thống truy xuất toàn bộ thông tin của Sản phẩm Gốc (Product Template) và các Biến thể (Product Variants) liên quan (nếu có). \newline 4. Hệ thống hiển thị Form chi tiết sản phẩm, thường được tổ chức thành nhiều tab: \newline    - \textbf{General Information (Thông tin chung):} Tên, Loại sản phẩm, Danh mục nội bộ, Mã nội bộ, Giá bán, Giá vốn, Thuế, Đơn vị tính... \newline    - \textbf{Variants (Biến thể):} Danh sách các Thuộc tính và Giá trị đã gán, nút để xem/quản lý các biến thể cụ thể. \newline    - \textbf{Sales (Bán hàng):} Cấu hình cho đơn bán hàng, chính sách hóa đơn, mô tả bán hàng. \newline    - \textbf{Point of Sale (Điểm bán hàng):} Cấu hình hiển thị trên POS (Available in POS, POS Category). \newline    - \textbf{Purchase (Mua hàng):} Cấu hình cho đơn mua hàng (nếu sản phẩm có thể mua). \newline    - \textbf{Inventory (Tồn kho):} Cấu hình liên quan đến quản lý kho (Lộ trình, Quy tắc Tồn kho, Trọng lượng, Thể tích...). \newline    - \textbf{Accounting (Kế toán):} Cấu hình tài khoản doanh thu, chi phí. \newline 5. US-01 xem xét các thông tin chi tiết. \\
\hline
Alternative Flow & \textbf{5a. Nhấp vào các nút thông minh (Smart Buttons):} \newline    1. Trên form sản phẩm, có các nút thông minh hiển thị số lượng liên quan (ví dụ: "X Variants", "Y On Hand", "Z Sold"). \newline    2. US-01 nhấp vào một nút thông minh để xem danh sách các bản ghi liên quan (ví dụ: danh sách biến thể, bút toán kho, đơn hàng đã bán). \\
\hline
Exception Flow & \textbf{3a. Lỗi tải chi tiết sản phẩm:} \newline    1. Hệ thống gặp lỗi khi lấy thông tin chi tiết của sản phẩm. \newline    2. Hệ thống báo lỗi. \newline \textbf{3b. Sản phẩm không tồn tại/không có quyền xem:} \newline    1. Do lỗi đồng bộ hoặc vấn đề phân quyền. \newline    2. Hệ thống báo lỗi "Không tìm thấy sản phẩm" hoặc "Không có quyền truy cập". \\
\hline
\multicolumn{2}{|c|}{\textbf{2.3. Thông tin bổ sung (Additional Information)}} \\
\hline
\textbf{Mục} & \textbf{Nội dung} \\
\hline
Business Rule & - \textbf{BR-UC2.3-1:} Form chi tiết phải hiển thị tất cả các trường thông tin đã được cấu hình cho sản phẩm một cách đầy đủ và chính xác. \\
\hline
Non-Functional Requirement & - \textbf{NFR-UC2.3-1 (Usability):} Thông tin trên form phải được tổ chức logic theo các tab, dễ tìm kiếm. \newline - \textbf{NFR-UC2.3-2 (Performance):} Thời gian tải form chi tiết sản phẩm (kể cả sản phẩm có nhiều biến thể) phải nhanh. \\
\hline
\end{longtable}

\subsubsection{Use Case UC-MD02-04: Sửa Thông tin Cơ bản của Sản phẩm}

\begin{longtable}{|m{4cm}|p{11cm}|}
\caption{Đặc tả Use Case UC-MD02-04: Sửa Thông tin Cơ bản của Sản phẩm} \label{tab:uc_md02_04_revised} \\
\hline
\multicolumn{2}{|c|}{\textbf{2.1. Tóm tắt (Summary)}} \\
\hline
\textbf{Mục} & \textbf{Nội dung} \\
\hline
\endhead % Header cho các trang tiếp theo
\hline
\endfoot % Footer cho bảng
\hline
\endlastfoot % Footer cho trang cuối cùng
Use Case Name & Sửa Thông tin Cơ bản của Sản phẩm \\
\hline
Use Case ID & UC-MD02-04 \\
\hline
Use Case Description & Cho phép Quản lý nhà hàng (US-01) cập nhật các thông tin chung và cơ bản của một sản phẩm đã tồn tại, như Tên sản phẩm, Loại sản phẩm, Giá bán, Giá vốn, Mã nội bộ, Danh mục sản phẩm nội bộ, Đơn vị tính, và các tùy chọn "Có thể Bán"/"Có thể Mua". \\
\hline
Actor & US-01 (Quản lý nhà hàng) \\
\hline
Priority & Must Have \\
\hline
Trigger & Thông tin cơ bản của một sản phẩm cần được cập nhật (ví dụ: đổi tên, thay đổi giá, phân loại lại). \\
\hline
Pre-Condition & - US-01 đã đăng nhập với quyền quản lý sản phẩm. \newline - Sản phẩm cần sửa đã tồn tại và US-01 đang xem form chi tiết của sản phẩm đó (UC-MD02-03). \\
\hline
Post-Condition & - Các thông tin cơ bản của sản phẩm được cập nhật thành công trong cơ sở dữ liệu. \newline - Các thay đổi (ví dụ: giá bán, tên) sẽ được phản ánh trên các giao diện liên quan. \\
\hline
\multicolumn{2}{|c|}{\textbf{2.2. Luồng thực thi (Flow)}} \\
\hline
\textbf{Mục} & \textbf{Nội dung} \\
\hline
Basic Flow & 1. US-01 đang xem form chi tiết sản phẩm (UC-MD02-03). \newline 2. US-01 chọn hành động "Sửa" (Edit). \newline 3. Hệ thống cho phép chỉnh sửa các trường trên tab "General Information" (hoặc tương đương). \newline 4. US-01 thực hiện các thay đổi mong muốn: \newline    - Sửa Tên sản phẩm (Product Name). \newline    - Thay đổi Loại sản phẩm (Product Type) (Lưu ý BR-UC2.4-2). \newline    - Cập nhật Giá bán (Sales Price), Giá vốn (Cost). \newline    - Sửa Danh mục sản phẩm nội bộ (Internal Product Category). \newline    - Sửa Mã nội bộ (Internal Reference). \newline    - Thay đổi Đơn vị tính (Unit of Measure), Đơn vị tính mua hàng (Purchase UoM). \newline    - Tick/Bỏ tick các ô "Can be Sold", "Can be Purchased". \newline    - (Tùy chọn) Sửa Mô tả nội bộ (Internal Notes). \newline 5. US-01 chọn hành động "Lưu" (Save). \newline 6. Hệ thống kiểm tra tính hợp lệ của các dữ liệu đã thay đổi (Tên không trống, Giá là số...). \newline 7. Hệ thống lưu các thay đổi vào bản ghi sản phẩm. \newline 8. Hệ thống chuyển form về chế độ xem với thông tin đã cập nhật và báo thành công. \\
\hline
Alternative Flow & \textbf{4a. Sửa các thông tin ở tab khác:} \newline    1. US-01 có thể chuyển sang các tab khác (Sales, Purchase, Inventory, Accounting) để sửa các thông tin chuyên sâu hơn liên quan đến các module đó (ví dụ: chính sách hóa đơn, lộ trình tồn kho, tài khoản kế toán). Các Use Case này có thể được coi là mở rộng hoặc thuộc về các module chuyên biệt đó. \\
\hline
Exception Flow & \textbf{6a. Lỗi Xác thực Dữ liệu:} \newline    1. Hệ thống phát hiện lỗi (ví dụ: Giá bán không phải số, Tên trống). \newline    2. Hệ thống báo lỗi, không cho lưu, giữ nguyên form để sửa. \newline \textbf{7a. Lỗi Hệ thống khi Cập nhật:} \newline    1. Hệ thống gặp lỗi khi lưu. \newline    2. Hệ thống báo lỗi chung. \\
\hline
\multicolumn{2}{|c|}{\textbf{2.3. Thông tin bổ sung (Additional Information)}} \\
\hline
\textbf{Mục} & \textbf{Nội dung} \\
\hline
Business Rule & - \textbf{BR-UC2.4-1:} Tên sản phẩm và Giá bán sau khi sửa không được để trống và phải hợp lệ. \newline - \textbf{BR-UC2.4-2:} Việc thay đổi Loại sản phẩm (đặc biệt từ/sang Stockable) của sản phẩm đã có giao dịch tồn kho có thể bị hạn chế hoặc yêu cầu các bước xử lý bổ sung (tham khảo BR-UC2.7-4 của UC tạo mới). \\
\hline
Non-Functional Requirement & - \textbf{NFR-UC2.4-1 (Usability):} Form sửa sản phẩm phải dễ dàng định vị và thay đổi các trường thông tin. \newline - \textbf{NFR-UC2.4-2 (Performance):} Thời gian lưu thay đổi phải nhanh. \\
\hline
\end{longtable}

\subsubsection{Use Case UC-MD02-05: Lưu trữ Sản phẩm}

\begin{longtable}{|m{4cm}|p{11cm}|}
\caption{Đặc tả Use Case UC-MD02-05: Lưu trữ Sản phẩm} \label{tab:uc_md02_05_revised} \\
\hline
\multicolumn{2}{|c|}{\textbf{2.1. Tóm tắt (Summary)}} \\
\hline
\textbf{Mục} & \textbf{Nội dung} \\
\hline
\endhead % Header cho các trang tiếp theo
\hline
\endfoot % Footer cho bảng
\hline
\endlastfoot % Footer cho trang cuối cùng
Use Case Name & Lưu trữ Sản phẩm \\
\hline
Use Case ID & UC-MD02-05 \\
\hline
Use Case Description & Cho phép Quản lý nhà hàng (US-01) tạm thời hoặc vĩnh viễn ẩn một sản phẩm khỏi các giao diện hoạt động (POS, danh sách chọn sản phẩm bán hàng...) bằng cách đặt sản phẩm đó vào trạng thái "Lưu trữ" (Archived). Dữ liệu lịch sử của sản phẩm vẫn được giữ lại. \\
\hline
Actor & US-01 (Quản lý nhà hàng) \\
\hline
Priority & Should Have \\
\hline
Trigger & Một sản phẩm không còn được bán/sử dụng nữa (ví dụ: món theo mùa đã hết, ngừng kinh doanh mặt hàng). \\
\hline
Pre-Condition & - US-01 đã đăng nhập với quyền quản lý sản phẩm. \newline - Sản phẩm cần lưu trữ đang ở trạng thái hoạt động (Active=True). \\
\hline
Post-Condition & - Trường 'Active' của sản phẩm được đặt thành False. \newline - Sản phẩm không còn hiển thị trong danh sách sản phẩm mặc định và không thể chọn trong các giao dịch mới. \newline - Dữ liệu lịch sử của sản phẩm không bị ảnh hưởng. \\
\hline
\multicolumn{2}{|c|}{\textbf{2.2. Luồng thực thi (Flow)}} \\
\hline
\textbf{Mục} & \textbf{Nội dung} \\
\hline
Basic Flow (Từ Form View) & 1. US-01 đang xem form chi tiết của sản phẩm muốn lưu trữ (UC-MD02-03). \newline 2. US-01 chọn menu "Hành động" (Action). \newline 3. US-01 chọn tùy chọn "Lưu trữ" (Archive). \newline 4. Hệ thống (có thể) hiển thị hộp thoại xác nhận. US-01 xác nhận. \newline 5. Hệ thống cập nhật trường `active` của sản phẩm thành `False`. \newline 6. Hệ thống có thể tự động điều hướng người dùng quay lại danh sách sản phẩm (nơi sản phẩm vừa lưu trữ sẽ không còn hiển thị theo bộ lọc mặc định). \newline 7. Hệ thống hiển thị thông báo "Sản phẩm đã được lưu trữ." \\
\hline
Alternative Flow & \textbf{Basic Flow (Từ List View):} \newline    1. US-01 đang xem danh sách sản phẩm (UC-MD02-02). \newline    2. US-01 chọn (tick vào ô vuông) một hoặc nhiều sản phẩm muốn lưu trữ. \newline    3. US-01 chọn menu "Hành động" (Action) của danh sách. \newline    4. US-01 chọn tùy chọn "Lưu trữ" (Archive). \newline    5. Hệ thống (có thể) yêu cầu xác nhận. US-01 xác nhận. \newline    6. Hệ thống cập nhật `active = False` cho tất cả các sản phẩm đã chọn. \newline    7. Hệ thống làm mới danh sách, các sản phẩm vừa lưu trữ biến mất. \newline    8. Hệ thống báo thành công. \\
\hline
Exception Flow & \textbf{5a. Lỗi hệ thống khi cập nhật trạng thái:} \newline    1. Hệ thống gặp lỗi khi cố gắng cập nhật trường `active`. \newline    2. Hệ thống báo lỗi chung. Trạng thái sản phẩm có thể không thay đổi. \\
\hline
\multicolumn{2}{|c|}{\textbf{2.3. Thông tin bổ sung (Additional Information)}} \\
\hline
\textbf{Mục} & \textbf{Nội dung} \\
\hline
Business Rule & - \textbf{BR-UC2.5-1:} Sản phẩm bị Lưu trữ sẽ không xuất hiện trong các lựa chọn sản phẩm mặc định trên POS, đơn bán hàng, v.v. \newline - \textbf{BR-UC2.5-2:} Lưu trữ không xóa dữ liệu lịch sử. \\
\hline
Non-Functional Requirement & - \textbf{NFR-UC2.5-1 (Usability):} Hành động Lưu trữ phải dễ dàng truy cập. \newline - \textbf{NFR-UC2.5-2 (Performance):} Cập nhật trạng thái phải nhanh. \\
\hline
\end{longtable}

\subsubsection{Use Case UC-MD02-06: Hủy Lưu trữ Sản phẩm}

\begin{longtable}{|m{4cm}|p{11cm}|}
\caption{Đặc tả Use Case UC-MD02-06: Hủy Lưu trữ Sản phẩm} \label{tab:uc_md02_06_revised} \\
\hline
\multicolumn{2}{|c|}{\textbf{2.1. Tóm tắt (Summary)}} \\
\hline
\textbf{Mục} & \textbf{Nội dung} \\
\hline
\endhead % Header cho các trang tiếp theo
\hline
\endfoot % Footer cho bảng
\hline
\endlastfoot % Footer cho trang cuối cùng
Use Case Name & Hủy Lưu trữ Sản phẩm \\
\hline
Use Case ID & UC-MD02-06 \\
\hline
Use Case Description & Cho phép Quản lý nhà hàng (US-01) kích hoạt lại một sản phẩm đã bị đặt vào trạng thái "Lưu trữ" (Archived), làm cho sản phẩm đó hoạt động trở lại và có thể được sử dụng trong các giao dịch. \\
\hline
Actor & US-01 (Quản lý nhà hàng) \\
\hline
Priority & Should Have \\
\hline
Trigger & Cần bán lại hoặc sử dụng lại một sản phẩm đã từng bị ẩn đi. \\
\hline
Pre-Condition & - US-01 đã đăng nhập với quyền quản lý sản phẩm. \newline - Sản phẩm cần hủy lưu trữ đang ở trạng thái "Lưu trữ" (Active=False). \\
\hline
Post-Condition & - Trường 'Active' của sản phẩm được đặt thành True. \newline - Sản phẩm xuất hiện trở lại trong danh sách sản phẩm mặc định và có thể được sử dụng trong các giao dịch. \\
\hline
\multicolumn{2}{|c|}{\textbf{2.2. Luồng thực thi (Flow)}} \\
\hline
\textbf{Mục} & \textbf{Nội dung} \\
\hline
Basic Flow (Từ Form View) & 1. US-01 truy cập danh sách sản phẩm và bỏ bộ lọc "Archived=False" (hoặc áp dụng bộ lọc "Archived=True") để tìm sản phẩm đã lưu trữ. \newline 2. US-01 chọn sản phẩm muốn hủy lưu trữ để mở form chi tiết. \newline 3. US-01 chọn menu "Hành động" (Action). \newline 4. US-01 chọn tùy chọn "Hủy lưu trữ" (Unarchive). \newline 5. Hệ thống (có thể) yêu cầu xác nhận. US-01 xác nhận. \newline 6. Hệ thống cập nhật trường `active` của sản phẩm thành `True`. \newline 7. Hệ thống hiển thị thông báo "Sản phẩm đã được hủy lưu trữ." Sản phẩm giờ sẽ hiển thị trong danh sách mặc định. \\
\hline
Alternative Flow & \textbf{Basic Flow (Từ List View):} \newline    1. US-01 truy cập danh sách sản phẩm và lọc để hiển thị các sản phẩm đã lưu trữ. \newline    2. US-01 chọn (tick) một hoặc nhiều sản phẩm muốn hủy lưu trữ. \newline    3. US-01 chọn menu "Hành động" (Action) của danh sách. \newline    4. US-01 chọn tùy chọn "Hủy lưu trữ" (Unarchive). \newline    5. Hệ thống (có thể) yêu cầu xác nhận. US-01 xác nhận. \newline    6. Hệ thống cập nhật `active = True` cho các sản phẩm đã chọn. \newline    7. Hệ thống làm mới danh sách. Nếu đang lọc "Archived", các sản phẩm này biến mất. Nếu về bộ lọc mặc định, chúng sẽ xuất hiện. \newline    8. Hệ thống báo thành công. \\
\hline
Exception Flow & \textbf{6a. Lỗi hệ thống khi cập nhật trạng thái:} \newline    1. Hệ thống gặp lỗi khi cố gắng cập nhật trường `active`. \newline    2. Hệ thống báo lỗi chung. \\
\hline
\multicolumn{2}{|c|}{\textbf{2.3. Thông tin bổ sung (Additional Information)}} \\
\hline
\textbf{Mục} & \textbf{Nội dung} \\
\hline
Business Rule & - \textbf{BR-UC2.6-1:} Chỉ những sản phẩm đang ở trạng thái "Lưu trữ" mới có thể được Hủy lưu trữ. \\
\hline
Non-Functional Requirement & - \textbf{NFR-UC2.6-1 (Usability):} Việc tìm và hủy lưu trữ sản phẩm phải dễ dàng. \newline - \textbf{NFR-UC2.6-2 (Performance):} Cập nhật trạng thái phải nhanh. \\
\hline
\end{longtable}

\subsubsection{Use Case UC-MD02-07: Tạo mới Danh mục POS}

\begin{longtable}{|m{4cm}|p{11cm}|}
\caption{Đặc tả Use Case UC-MD02-07: Tạo mới Danh mục POS} \label{tab:uc_md02_07_revised} \\
\hline
\multicolumn{2}{|c|}{\textbf{2.1. Tóm tắt (Summary)}} \\
\hline
\textbf{Mục} & \textbf{Nội dung} \\
\hline
\endhead % Header cho các trang tiếp theo
\hline
\endfoot % Footer cho bảng
\hline
\endlastfoot % Footer cho trang cuối cùng
Use Case Name & Tạo mới Danh mục POS \\
\hline
Use Case ID & UC-MD02-07 \\
\hline
Use Case Description & Cho phép Quản lý nhà hàng (US-01) tạo một danh mục mới (ví dụ: "Khai vị", "Món chính", "Đồ uống đặc biệt") để sử dụng cho việc phân loại và hiển thị sản phẩm trên giao diện Point of Sale (POS). \\
\hline
Actor & US-01 (Quản lý nhà hàng) \\
\hline
Priority & Must Have \\
\hline
Trigger & Cần một nhóm mới để tổ chức các món ăn/đồ uống trên màn hình POS. \\
\hline
Pre-Condition & - US-01 đã đăng nhập với quyền quản trị cấu hình Point of Sale. \\
\hline
Post-Condition & - Một bản ghi Danh mục POS mới được tạo và lưu. \newline - Danh mục mới này sẵn sàng để được gán sản phẩm vào và hiển thị trên POS. \\
\hline
\multicolumn{2}{|c|}{\textbf{2.2. Luồng thực thi (Flow)}} \\
\hline
\textbf{Mục} & \textbf{Nội dung} \\
\hline
Basic Flow & 1. US-01 truy cập vào phần cấu hình của Point of Sale. \newline 2. US-01 chọn mục quản lý "Danh mục Sản phẩm POS" (POS Product Categories). \newline 3. Hệ thống hiển thị danh sách các danh mục POS hiện có (UC-MD02-08). \newline 4. US-01 chọn hành động "Tạo mới" (Create). \newline 5. Hệ thống hiển thị form để nhập thông tin danh mục mới. \newline 6. US-01 nhập Tên Danh mục (Category Name) (bắt buộc - BR-UC2.7-1). \newline 7. (Tùy chọn) US-01 chọn Danh mục Cha (Parent Category) nếu muốn tạo cấu trúc phân cấp. \newline 8. (Tùy chọn) US-01 nhập Số thứ tự (Sequence) để kiểm soát vị trí hiển thị. \newline 9. (Tùy chọn) US-01 chọn hình ảnh đại diện cho danh mục (nếu POS theme hỗ trợ). \newline 10. US-01 chọn lệnh "Lưu" (Save). \newline 11. Hệ thống kiểm tra Tên Danh mục không trống. \newline 12. Hệ thống lưu bản ghi danh mục POS mới. \newline 13. Hệ thống cập nhật danh sách, hiển thị danh mục mới. \newline 14. Hệ thống hiển thị thông báo tạo thành công. \\
\hline
Alternative Flow & Tương tự UC-MD01-01 (Lưu và Tạo mới). \\
\hline
Exception Flow & \textbf{11a. Lỗi Xác thực Dữ liệu:} Tên danh mục trống. \newline \textbf{12a. Lỗi Hệ thống khi Lưu:} Hệ thống gặp sự cố khi lưu. \\
\hline
\multicolumn{2}{|c|}{\textbf{2.3. Thông tin bổ sung (Additional Information)}} \\
\hline
\textbf{Mục} & \textbf{Nội dung} \\
\hline
Business Rule & - \textbf{BR-UC2.7-1:} Tên Danh mục POS là bắt buộc. \newline - \textbf{BR-UC2.7-2:} Hỗ trợ cấu trúc danh mục cha-con. \\
\hline
Non-Functional Requirement & - \textbf{NFR-UC2.7-1 (Usability):} Giao diện tạo danh mục POS đơn giản. \newline - \textbf{NFR-UC2.7-2 (Performance):} Lưu danh mục mới nhanh chóng. \\
\hline
\end{longtable}

\subsubsection{Use Case UC-MD02-08: Xem Danh sách Danh mục POS}

\begin{longtable}{|m{4cm}|p{11cm}|}
\caption{Đặc tả Use Case UC-MD02-08: Xem Danh sách Danh mục POS} \label{tab:uc_md02_08_revised} \\
\hline
\multicolumn{2}{|c|}{\textbf{2.1. Tóm tắt (Summary)}} \\
\hline
\textbf{Mục} & \textbf{Nội dung} \\
\hline
\endhead % Header cho các trang tiếp theo
\hline
\endfoot % Footer cho bảng
\hline
\endlastfoot % Footer cho trang cuối cùng
Use Case Name & Xem Danh sách Danh mục POS \\
\hline
Use Case ID & UC-MD02-08 \\
\hline
Use Case Description & Cho phép Quản lý nhà hàng (US-01) xem danh sách tất cả các Danh mục Sản phẩm POS đã được tạo trong hệ thống, bao gồm tên, thứ tự, và cấu trúc cha-con (nếu có). \\
\hline
Actor & US-01 (Quản lý nhà hàng) \\
\hline
Priority & Must Have \\
\hline
Trigger & Cần kiểm tra các danh mục POS hiện có, tìm kiếm hoặc chuẩn bị cho việc tạo/sửa/xóa/sắp xếp. \\
\hline
Pre-Condition & - US-01 đã đăng nhập với quyền quản trị cấu hình Point of Sale. \\
\hline
Post-Condition & - Danh sách các danh mục POS được hiển thị. \\
\hline
\multicolumn{2}{|c|}{\textbf{2.2. Luồng thực thi (Flow)}} \\
\hline
\textbf{Mục} & \textbf{Nội dung} \\
\hline
Basic Flow & 1. US-01 truy cập vào phần cấu hình của Point of Sale > "Danh mục Sản phẩm POS". \newline 2. Hệ thống hiển thị danh sách các danh mục POS đã tạo, thường theo thứ tự và cấu trúc phân cấp. \newline 3. Thông tin hiển thị: Tên Danh mục, Danh mục Cha (nếu có), Số thứ tự. \newline 4. US-01 xem xét danh sách. \\
\hline
Alternative Flow & \textbf{4a. Tìm kiếm/Lọc/Sắp xếp:} Tương tự như UC-MD01-02. \\
\hline
Exception Flow & \textbf{2a. Lỗi tải danh sách.} \newline \textbf{2b. Không có danh mục nào.} \\
\hline
\multicolumn{2}{|c|}{\textbf{2.3. Thông tin bổ sung (Additional Information)}} \\
\hline
\textbf{Mục} & \textbf{Nội dung} \\
\hline
Business Rule & - \textbf{BR-UC2.8-1:} Danh sách phải hiển thị chính xác các danh mục POS. \\
\hline
Non-Functional Requirement & - \textbf{NFR-UC2.8-1 (Usability):} Danh sách rõ ràng, cấu trúc phân cấp dễ nhìn. \newline - \textbf{NFR-UC2.8-2 (Performance):} Tải danh sách nhanh. \\
\hline
\end{longtable}

\subsubsection{Use Case UC-MD02-09: Sửa Danh mục POS}

\begin{longtable}{|m{4cm}|p{11cm}|}
\caption{Đặc tả Use Case UC-MD02-09: Sửa Danh mục POS} \label{tab:uc_md02_09_revised} \\
\hline
\multicolumn{2}{|c|}{\textbf{2.1. Tóm tắt (Summary)}} \\
\hline
\textbf{Mục} & \textbf{Nội dung} \\
\hline
\endhead % Header cho các trang tiếp theo
\hline
\endfoot % Footer cho bảng
\hline
\endlastfoot % Footer cho trang cuối cùng
Use Case Name & Sửa Danh mục POS \\
\hline
Use Case ID & UC-MD02-09 \\
\hline
Use Case Description & Cho phép Quản lý nhà hàng (US-01) chỉnh sửa thông tin của một Danh mục Sản phẩm POS đã tồn tại, như Tên, Danh mục Cha, Số thứ tự, hoặc Hình ảnh. \\
\hline
Actor & US-01 (Quản lý nhà hàng) \\
\hline
Priority & Must Have \\
\hline
Trigger & Cần thay đổi thông tin hoặc cấu trúc của một danh mục POS. \\
\hline
Pre-Condition & - US-01 đã đăng nhập với quyền quản trị cấu hình Point of Sale. \newline - Danh mục POS cần sửa đã tồn tại. \\
\hline
Post-Condition & - Thông tin của danh mục POS được cập nhật. \newline - Thay đổi sẽ ảnh hưởng đến cách hiển thị trên POS. \\
\hline
\multicolumn{2}{|c|}{\textbf{2.2. Luồng thực thi (Flow)}} \\
\hline
\textbf{Mục} & \textbf{Nội dung} \\
\hline
Basic Flow & 1. US-01 truy cập danh sách Danh mục POS (UC-MD02-08). \newline 2. US-01 chọn danh mục muốn sửa. \newline 3. Hệ thống hiển thị form chi tiết danh mục. \newline 4. US-01 chọn "Sửa". \newline 5. US-01 chỉnh sửa các thông tin (Tên, Cha, Thứ tự, Ảnh...). \newline 6. US-01 chọn "Lưu". \newline 7. Hệ thống kiểm tra hợp lệ và lưu thay đổi. \newline 8. Hệ thống báo thành công. \\
\hline
Alternative Flow & Không có. \\
\hline
Exception Flow & \textbf{7a. Lỗi Xác thực/Lưu:} Tương tự UC-MD01-03. \\
\hline
\multicolumn{2}{|c|}{\textbf{2.3. Thông tin bổ sung (Additional Information)}} \\
\hline
\textbf{Mục} & \textbf{Nội dung} \\
\hline
Business Rule & - \textbf{BR-UC2.9-1:} Tên Danh mục POS không được để trống. \newline - \textbf{BR-UC2.9-2:} Thay đổi Danh mục Cha có thể thay đổi cấu trúc phân cấp. \\
\hline
Non-Functional Requirement & - \textbf{NFR-UC2.9-1 (Usability):} Dễ sử dụng. \newline - \textbf{NFR-UC2.9-2 (Performance):} Lưu nhanh. \\
\hline
\end{longtable}

\subsubsection{Use Case UC-MD02-10: Xóa Danh mục POS}

\begin{longtable}{|m{4cm}|p{11cm}|}
\caption{Đặc tả Use Case UC-MD02-10: Xóa Danh mục POS} \label{tab:uc_md02_10_revised} \\
\hline
\multicolumn{2}{|c|}{\textbf{2.1. Tóm tắt (Summary)}} \\
\hline
\textbf{Mục} & \textbf{Nội dung} \\
\hline
\endhead % Header cho các trang tiếp theo
\hline
\endfoot % Footer cho bảng
\hline
\endlastfoot % Footer cho trang cuối cùng
Use Case Name & Xóa Danh mục POS \\
\hline
Use Case ID & UC-MD02-10 \\
\hline
Use Case Description & Cho phép Quản lý nhà hàng (US-01) xóa một Danh mục Sản phẩm POS không còn sử dụng, với điều kiện danh mục đó không chứa sản phẩm nào hoặc không có danh mục con. \\
\hline
Actor & US-01 (Quản lý nhà hàng) \\
\hline
Priority & Should Have \\
\hline
Trigger & Một danh mục POS không còn cần thiết. \\
\hline
Pre-Condition & - US-01 đã đăng nhập với quyền quản trị cấu hình Point of Sale. \newline - Danh mục POS cần xóa đã tồn tại. \\
\hline
Post-Condition & - Nếu xóa thành công, danh mục POS bị xóa. \newline - Nếu không, danh mục vẫn tồn tại và hệ thống báo lỗi. \\
\hline
\multicolumn{2}{|c|}{\textbf{2.2. Luồng thực thi (Flow)}} \\
\hline
\textbf{Mục} & \textbf{Nội dung} \\
\hline
Basic Flow & 1. US-01 truy cập danh sách Danh mục POS (UC-MD02-08). \newline 2. US-01 chọn danh mục muốn xóa. \newline 3. US-01 chọn hành động "Xóa". \newline 4. Hệ thống yêu cầu xác nhận. US-01 xác nhận. \newline 5. Hệ thống kiểm tra điều kiện xóa (BR-UC2.10-1). \newline 6. Nếu thỏa điều kiện, hệ thống xóa danh mục và báo thành công. \newline 7. Nếu không thỏa, hệ thống báo lỗi và giải thích. \\
\hline
Alternative Flow & Xóa từ danh sách (tương tự UC-MD01-04). \\
\hline
Exception Flow & Lỗi hệ thống khi kiểm tra/xóa. \\
\hline
\multicolumn{2}{|c|}{\textbf{2.3. Thông tin bổ sung (Additional Information)}} \\
\hline
\textbf{Mục} & \textbf{Nội dung} \\
\hline
Business Rule & - \textbf{BR-UC2.10-1:} Không thể xóa Danh mục POS nếu đang chứa sản phẩm hoặc có danh mục con. Cần di chuyển/xóa sản phẩm/danh mục con trước. \\
\hline
Non-Functional Requirement & - \textbf{NFR-UC2.10-1 (Data Integrity):} Ràng buộc không cho xóa quan trọng. \newline - \textbf{NFR-UC2.10-2 (Usability):} Thông báo lỗi rõ ràng. \\
\hline
\end{longtable}

\subsubsection{Use Case UC-MD02-11: Sắp xếp Thứ tự Danh mục POS}

\begin{longtable}{|m{4cm}|p{11cm}|}
\caption{Đặc tả Use Case UC-MD02-11: Sắp xếp Thứ tự Danh mục POS} \label{tab:uc_md02_11_revised} \\
\hline
\multicolumn{2}{|c|}{\textbf{2.1. Tóm tắt (Summary)}} \\
\hline
\textbf{Mục} & \textbf{Nội dung} \\
\hline
\endhead % Header cho các trang tiếp theo
\hline
\endfoot % Footer cho bảng
\hline
\endlastfoot % Footer cho trang cuối cùng
Use Case Name & Sắp xếp Thứ tự Danh mục POS \\
\hline
Use Case ID & UC-MD02-11 \\
\hline
Use Case Description & Cho phép Quản lý nhà hàng (US-01) thay đổi thứ tự xuất hiện của các Danh mục Sản phẩm POS trên giao diện chọn món của POS, thường bằng cách sửa giá trị "Sequence" hoặc kéo thả. \\
\hline
Actor & US-01 (Quản lý nhà hàng) \\
\hline
Priority & Should Have \\
\hline
Trigger & Cần thay đổi cách các danh mục được sắp xếp trên màn hình POS để tối ưu hóa cho nhân viên hoặc làm nổi bật danh mục nào đó. \\
\hline
Pre-Condition & - US-01 đã đăng nhập với quyền quản trị cấu hình Point of Sale. \newline - Có ít nhất hai danh mục POS để sắp xếp. \\
\hline
Post-Condition & - Thứ tự mới của các danh mục POS được lưu lại. \newline - Thứ tự này sẽ được phản ánh trên giao diện POS sau khi đồng bộ. \\
\hline
\multicolumn{2}{|c|}{\textbf{2.2. Luồng thực thi (Flow)}} \\
\hline
\textbf{Mục} & \textbf{Nội dung} \\
\hline
Basic Flow (Kéo thả) & 1. US-01 truy cập danh sách Danh mục POS (UC-MD02-08), thường ở dạng Tree View hoặc List View có hỗ trợ kéo thả. \newline 2. US-01 nhấn giữ vào một danh mục và kéo nó đến vị trí mong muốn trong danh sách (so với các danh mục cùng cấp). \newline 3. US-01 thả chuột. \newline 4. Hệ thống tự động cập nhật giá trị "Sequence" cho các danh mục bị ảnh hưởng và lưu lại thứ tự mới. \newline 5. Giao diện danh sách cập nhật theo thứ tự mới. \\
\hline
Alternative Flow & \textbf{1a. Sửa trực tiếp trường Sequence:} \newline    1. US-01 mở form chi tiết của từng danh mục (UC-MD02-09). \newline    2. US-01 sửa giá trị trong trường "Sequence" (số nhỏ hơn hiển thị trước). \newline    3. US-01 lưu lại. \\
\hline
Exception Flow & \textbf{4a. Lỗi hệ thống khi lưu thứ tự:} \newline    1. Hệ thống gặp lỗi khi cập nhật giá trị sequence. \newline    2. Hệ thống báo lỗi. Thứ tự có thể không được lưu đúng. \\
\hline
\multicolumn{2}{|c|}{\textbf{2.3. Thông tin bổ sung (Additional Information)}} \\
\hline
\textbf{Mục} & \textbf{Nội dung} \\
\hline
Business Rule & - \textbf{BR-UC2.11-1:} Thứ tự hiển thị dựa trên trường "Sequence". Các danh mục cùng cấp được sắp xếp theo giá trị này. \\
\hline
Non-Functional Requirement & - \textbf{NFR-UC2.11-1 (Usability):} Chức năng kéo thả (nếu có) rất tiện lợi. Việc sắp xếp phải trực quan. \newline - \textbf{NFR-UC2.11-2 (Performance):} Cập nhật thứ tự phải nhanh. \\
\hline
\end{longtable}

\subsubsection{Use Case UC-MD02-12: Tạo mới Thuộc tính Sản phẩm}

\begin{longtable}{|m{4cm}|p{11cm}|}
\caption{Đặc tả Use Case UC-MD02-12: Tạo mới Thuộc tính Sản phẩm} \label{tab:uc_md02_12_revised} \\
\hline
\multicolumn{2}{|c|}{\textbf{2.1. Tóm tắt (Summary)}} \\
\hline
\textbf{Mục} & \textbf{Nội dung} \\
\hline
\endhead % Header cho các trang tiếp theo
\hline
\endfoot % Footer cho bảng
\hline
\endlastfoot % Footer cho trang cuối cùng
Use Case Name & Tạo mới Thuộc tính Sản phẩm \\
\hline
Use Case ID & UC-MD02-12 \\
\hline
Use Case Description & Cho phép người dùng có quyền (US-01/US-10) định nghĩa một đặc tính chung mới cho sản phẩm mà có thể có nhiều lựa chọn khác nhau (ví dụ: "Kích cỡ", "Màu sắc", "Độ cay", "Loại đế bánh"). Đây là bước đầu để tạo biến thể sản phẩm. \\
\hline
Actor & US-01 (Quản lý nhà hàng), US-10 (Quản trị viên Hệ thống) \\
\hline
Priority & Must Have (Nếu cần biến thể) \\
\hline
Trigger & Cần quản lý các phiên bản khác nhau của sản phẩm dựa trên một đặc tính mới chưa có trong hệ thống. \\
\hline
Pre-Condition & - Người dùng đã đăng nhập với quyền quản trị cấu hình sản phẩm/inventory. \\
\hline
Post-Condition & - Một bản ghi Thuộc tính (Attribute) mới được tạo và lưu. \newline - Thuộc tính này sẵn sàng để được thêm các Giá trị (UC-MD02-13) và sau đó gán vào sản phẩm (UC-MD02-14). \\
\hline
\multicolumn{2}{|c|}{\textbf{2.2. Luồng thực thi (Flow)}} \\
\hline
\textbf{Mục} & \textbf{Nội dung} \\
\hline
Basic Flow & 1. Người dùng truy cập khu vực cấu hình Thuộc tính (ví dụ: Inventory > Configuration > Product Attributes). \newline 2. Hệ thống hiển thị danh sách Thuộc tính đã có. \newline 3. Người dùng chọn "Tạo mới". \newline 4. Hệ thống hiển thị form tạo Thuộc tính. \newline 5. Người dùng nhập Tên Thuộc tính (Attribute Name) (ví dụ: "Kích cỡ Pizza") (Bắt buộc, duy nhất - BR-UC2.12-1). \newline 6. Người dùng chọn Loại hiển thị (Display Type - Radio, Select, Color) để xác định cách thuộc tính hiển thị khi chọn. \newline 7. Người dùng chọn Chế độ tạo Biến thể (Variants Creation Mode - Instantly, Dynamically, Never). "Instantly" thường dùng. \newline 8. Người dùng chọn "Lưu". \newline 9. Hệ thống kiểm tra và lưu Thuộc tính mới. \newline 10. Hệ thống báo thành công. \\
\hline
Alternative Flow & Lưu và Tạo mới. \\
\hline
Exception Flow & \textbf{9a. Lỗi Xác thực/Lưu:} Tên trống/trùng. Lỗi hệ thống. \\
\hline
\multicolumn{2}{|c|}{\textbf{2.3. Thông tin bổ sung (Additional Information)}} \\
\hline
\textbf{Mục} & \textbf{Nội dung} \\
\hline
Business Rule & - \textbf{BR-UC2.12-1:} Tên Thuộc tính phải là duy nhất. \\
\hline
Non-Functional Requirement & - \textbf{NFR-UC2.12-1 (Usability):} Dễ sử dụng. \newline - \textbf{NFR-UC2.12-2 (Performance):} Lưu nhanh. \\
\hline
\end{longtable}

\subsubsection{Use Case UC-MD02-13: Tạo mới Giá trị cho Thuộc tính}

\begin{longtable}{|m{4cm}|p{11cm}|}
\caption{Đặc tả Use Case UC-MD02-13: Tạo mới Giá trị cho Thuộc tính} \label{tab:uc_md02_13_revised} \\
\hline
\multicolumn{2}{|c|}{\textbf{2.1. Tóm tắt (Summary)}} \\
\hline
\textbf{Mục} & \textbf{Nội dung} \\
\hline
\endhead % Header cho các trang tiếp theo
\hline
\endfoot % Footer cho bảng
\hline
\endlastfoot % Footer cho trang cuối cùng
Use Case Name & Tạo mới Giá trị cho Thuộc tính \\
\hline
Use Case ID & UC-MD02-13 \\
\hline
Use Case Description & Sau khi một Thuộc tính đã được tạo (UC-MD02-12), cho phép người dùng có quyền (US-01/US-10) định nghĩa các lựa chọn/giá trị cụ thể cho Thuộc tính đó (ví dụ: cho Thuộc tính "Kích cỡ Pizza", tạo các Giá trị "Nhỏ", "Vừa", "Lớn"). \\
\hline
Actor & US-01 (Quản lý nhà hàng), US-10 (Quản trị viên Hệ thống) \\
\hline
Priority & Must Have (Nếu cần biến thể) \\
\hline
Trigger & Một Thuộc tính đã được tạo và cần định nghĩa các lựa chọn khả dĩ cho nó. \\
\hline
Pre-Condition & - Người dùng đã đăng nhập với quyền quản trị cấu hình sản phẩm/inventory. \newline - Thuộc tính cha đã được tạo (UC-MD02-12). \\
\hline
Post-Condition & - Một hoặc nhiều bản ghi Giá trị Thuộc tính (Attribute Value) mới được tạo và liên kết với Thuộc tính cha. \newline - Các Giá trị này sẵn sàng để được chọn khi gán Thuộc tính vào sản phẩm (UC-MD02-14). \\
\hline
\multicolumn{2}{|c|}{\textbf{2.2. Luồng thực thi (Flow)}} \\
\hline
\textbf{Mục} & \textbf{Nội dung} \\
\hline
Basic Flow & 1. Người dùng truy cập form chi tiết của một Thuộc tính (đã tạo ở UC-MD02-12). \newline 2. Người dùng chọn tab/mục "Giá trị" (Attribute Values). \newline 3. Người dùng chọn "Thêm một dòng" (Add a line) hoặc "Tạo mới". \newline 4. Hệ thống hiển thị dòng/form để nhập Giá trị. \newline 5. Người dùng nhập Tên Giá trị (Value Name) (ví dụ: "Nhỏ") (Bắt buộc, duy nhất trong Thuộc tính - BR-UC2.13-1). \newline 6. (Tùy chọn) Nếu Thuộc tính có Loại hiển thị là Color, người dùng chọn màu tương ứng. \newline 7. Người dùng lặp lại bước 3-6 để thêm các Giá trị khác cho Thuộc tính này. \newline 8. Người dùng chọn "Lưu" trên form Thuộc tính để lưu tất cả các Giá trị. \newline 9. Hệ thống kiểm tra và lưu các Giá trị mới. \newline 10. Hệ thống báo thành công. \\
\hline
Alternative Flow & Không có. \\
\hline
Exception Flow & \textbf{9a. Lỗi Xác thực/Lưu:} Tên Giá trị trống/trùng. Lỗi hệ thống. \\
\hline
\multicolumn{2}{|c|}{\textbf{2.3. Thông tin bổ sung (Additional Information)}} \\
\hline
\textbf{Mục} & \textbf{Nội dung} \\
\hline
Business Rule & - \textbf{BR-UC2.13-1:} Tên Giá trị phải duy nhất trong phạm vi Thuộc tính cha của nó. \\
\hline
Non-Functional Requirement & - \textbf{NFR-UC2.13-1 (Usability):} Việc thêm giá trị trong ngữ cảnh thuộc tính phải thuận tiện. \newline - \textbf{NFR-UC2.13-2 (Performance):} Lưu nhanh. \\
\hline
\end{longtable}

\subsubsection{Use Case UC-MD02-14: Gán Thuộc tính và Giá trị vào Sản phẩm Gốc}

\begin{longtable}{|m{4cm}|p{11cm}|}
\caption{Đặc tả Use Case UC-MD02-14: Gán Thuộc tính và Giá trị vào Sản phẩm Gốc} \label{tab:uc_md02_14_revised} \\
\hline
\multicolumn{2}{|c|}{\textbf{2.1. Tóm tắt (Summary)}} \\
\hline
\textbf{Mục} & \textbf{Nội dung} \\
\hline
\endhead % Header cho các trang tiếp theo
\hline
\endfoot % Footer cho bảng
\hline
\endlastfoot % Footer cho trang cuối cùng
Use Case Name & Gán Thuộc tính và Giá trị vào Sản phẩm Gốc \\
\hline
Use Case ID & UC-MD02-14 \\
\hline
Use Case Description & Cho phép Quản lý nhà hàng (US-01) áp dụng các Thuộc tính (đã tạo ở UC-MD02-12) và chọn các Giá trị cụ thể (đã tạo ở UC-MD02-13) cho một Sản phẩm Gốc (Product Template). Hành động này sẽ làm cơ sở để hệ thống tự động tạo ra các Sản phẩm Biến thể. \\
\hline
Actor & US-01 (Quản lý nhà hàng) \\
\hline
Priority & Must Have (Nếu cần biến thể) \\
\hline
Trigger & Cần tạo các phiên bản khác nhau cho một sản phẩm dựa trên các đặc tính đã định nghĩa. \\
\hline
Pre-Condition & - US-01 đã đăng nhập với quyền quản lý sản phẩm. \newline - Sản phẩm Gốc đã được tạo (UC-MD02-01). \newline - Các Thuộc tính và Giá trị liên quan đã được tạo (UC-MD02-12, UC-MD02-13). \\
\hline
Post-Condition & - Sản phẩm Gốc được liên kết với các dòng Thuộc tính, mỗi dòng Thuộc tính chứa các Giá trị được chọn. \newline - Hệ thống tự động tạo ra các bản ghi Sản phẩm Biến thể (Product Variant) tương ứng với mọi tổ hợp của các Giá trị đã chọn. \\
\hline
\multicolumn{2}{|c|}{\textbf{2.2. Luồng thực thi (Flow)}} \\
\hline
\textbf{Mục} & \textbf{Nội dung} \\
\hline
Basic Flow & 1. US-01 mở Form chi tiết của Sản phẩm Gốc cần cấu hình biến thể, ở chế độ Sửa. \newline 2. US-01 chuyển đến tab "Variants" (Biến thể). \newline 3. Trong phần "Attributes", US-01 chọn "Add a line". \newline 4. US-01 chọn một Thuộc tính từ danh sách thả xuống (ví dụ: "Kích cỡ Pizza"). \newline 5. Trong cột "Values" tương ứng, US-01 chọn (tick) vào (các) Giá trị sẽ áp dụng cho sản phẩm này (ví dụ: "Nhỏ", "Vừa", "Lớn"). \newline 6. US-01 lặp lại bước 3-5 để thêm các Thuộc tính và Giá trị khác nếu cần. \newline 7. US-01 chọn "Lưu" (Save) sản phẩm gốc. \newline 8. Hệ thống tự động tạo ra các bản ghi Sản phẩm Biến thể dựa trên tổ hợp các Giá trị đã chọn. Số lượng biến thể được tạo sẽ hiển thị trên form (ví dụ: nút "X Variants"). \\
\hline
Alternative Flow & Không có. \\
\hline
Exception Flow & \textbf{4a/5a. Lỗi chọn Thuộc tính/Giá trị:} Thuộc tính/Giá trị không tồn tại (phải được tạo trước). \newline \textbf{7a. Lỗi khi lưu cấu hình thuộc tính.} \newline \textbf{8a. Lỗi khi tự động tạo biến thể.} \\
\hline
\multicolumn{2}{|c|}{\textbf{2.3. Thông tin bổ sung (Additional Information)}} \\
\hline
\textbf{Mục} & \textbf{Nội dung} \\
\hline
Business Rule & - \textbf{BR-UC2.14-1:} Số lượng Biến thể tạo ra bằng tích số lượng Giá trị được chọn cho mỗi Thuộc tính. \\
\hline
Non-Functional Requirement & - \textbf{NFR-UC2.14-1 (Usability):} Giao diện gán thuộc tính/giá trị phải trực quan. \newline - \textbf{NFR-UC2.14-2 (Performance):} Việc tự động tạo biến thể (< 50 biến thể) phải nhanh (< 5 giây). \\
\hline
\end{longtable}

\subsubsection{Use Case UC-MD02-15: Cấu hình Giá/Phụ thu cho Biến thể Sản phẩm}

\begin{longtable}{|m{4cm}|p{11cm}|}
\caption{Đặc tả Use Case UC-MD02-15: Cấu hình Giá/Phụ thu cho Biến thể Sản phẩm} \label{tab:uc_md02_15_revised} \\
\hline
\multicolumn{2}{|c|}{\textbf{2.1. Tóm tắt (Summary)}} \\
\hline
\textbf{Mục} & \textbf{Nội dung} \\
\hline
\endhead % Header cho các trang tiếp theo
\hline
\endfoot % Footer cho bảng
\hline
\endlastfoot % Footer cho trang cuối cùng
Use Case Name & Cấu hình Giá/Phụ thu cho Biến thể Sản phẩm \\
\hline
Use Case ID & UC-MD02-15 \\
\hline
Use Case Description & Sau khi các Sản phẩm Biến thể đã được tạo (từ UC-MD02-14), cho phép Quản lý nhà hàng (US-01) truy cập vào từng biến thể cụ thể để đặt một giá bán riêng cho nó, hoặc thiết lập một giá trị phụ thu (price extra) dựa trên các giá trị thuộc tính tạo nên biến thể đó. \\
\hline
Actor & US-01 (Quản lý nhà hàng) \\
\hline
Priority & Must Have (Nếu các biến thể có giá khác nhau) \\
\hline
Trigger & Các biến thể khác nhau của cùng một sản phẩm gốc có giá bán không giống nhau. \\
\hline
Pre-Condition & - Các Sản phẩm Biến thể đã được tạo (UC-MD02-14). \newline - US-01 đang xem form Sản phẩm Gốc hoặc danh sách các Biến thể của nó. \\
\hline
Post-Condition & - Giá bán hoặc phụ thu của (các) Biến thể được chọn đã được cập nhật. \newline - Khi biến thể đó được chọn trên POS, giá sẽ được tính đúng theo cấu hình này. \\
\hline
\multicolumn{2}{|c|}{\textbf{2.2. Luồng thực thi (Flow)}} \\
\hline
\textbf{Mục} & \textbf{Nội dung} \\
\hline
Basic Flow (Đặt giá riêng cho biến thể) & 1. US-01 đang xem form Sản phẩm Gốc, nhấp vào nút "X Variants" để mở danh sách các Biến thể. \newline 2. US-01 chọn một Biến thể cụ thể từ danh sách để mở form chi tiết của Biến thể đó. \newline 3. US-01 chọn "Sửa" (Edit). \newline 4. US-01 tìm đến trường "Giá bán" (Sales Price) của Biến thể và nhập giá bán mong muốn cho riêng biến thể này. \newline 5. US-01 chọn "Lưu" (Save). \newline 6. Hệ thống lưu giá bán mới cho Biến thể. \\
\hline
Alternative Flow & \textbf{Basic Flow (Đặt phụ thu cho giá trị thuộc tính):} \newline    1. US-01 đang xem form Sản phẩm Gốc, ở tab "Variants", trong chế độ Sửa. \newline    2. Bên cạnh mỗi Giá trị của Thuộc tính (ví dụ: bên cạnh "Lớn" của "Kích cỡ Pizza"), có một cột "Phụ thu giá trị" (Value Price Extra). \newline    3. US-01 nhập số tiền phụ thu cho Giá trị đó (ví dụ: +20,000 VNĐ cho size Lớn). \newline    4. US-01 chọn "Lưu" (Save) Sản phẩm Gốc. \newline    5. Hệ thống sẽ tự động tính giá của Biến thể bằng cách cộng giá Sản phẩm Gốc với tổng các phụ thu của các Giá trị tạo nên Biến thể đó. \\
\hline
Exception Flow & \textbf{6a/5a-alt. Lỗi lưu giá/phụ thu:} Hệ thống báo lỗi khi lưu. \\
\hline
\multicolumn{2}{|c|}{\textbf{2.3. Thông tin bổ sung (Additional Information)}} \\
\hline
\textbf{Mục} & \textbf{Nội dung} \\
\hline
Business Rule & - \textbf{BR-UC2.15-1:} Nếu không đặt giá riêng cho Biến thể, nó sẽ kế thừa giá của Sản phẩm Gốc cộng với các phụ thu giá trị thuộc tính. \newline - \textbf{BR-UC2.15-2:} Giá bán phải là số không âm. \\
\hline
Non-Functional Requirement & - \textbf{NFR-UC2.15-1 (Usability):} Việc cấu hình giá/phụ thu phải rõ ràng. \newline - \textbf{NFR-UC2.15-2 (Accuracy):} Giá cuối cùng của biến thể phải được tính toán chính xác. \\
\hline
\end{longtable}

\subsubsection{Use Case UC-MD02-16: Thiết lập Sản phẩm được Bán trên POS}

\begin{longtable}{|m{4cm}|p{11cm}|}
\caption{Đặc tả Use Case UC-MD02-16: Thiết lập Sản phẩm được Bán trên POS} \label{tab:uc_md02_16_revised} \\
\hline
\multicolumn{2}{|c|}{\textbf{2.1. Tóm tắt (Summary)}} \\
\hline
\textbf{Mục} & \textbf{Nội dung} \\
\hline
\endhead % Header cho các trang tiếp theo
\hline
\endfoot % Footer cho bảng
\hline
\endlastfoot % Footer cho trang cuối cùng
Use Case Name & Thiết lập Sản phẩm được Bán trên POS \\
\hline
Use Case ID & UC-MD02-16 \\
\hline
Use Case Description & Cho phép Quản lý nhà hàng (US-01) đánh dấu hoặc bỏ đánh dấu một sản phẩm là có sẵn để bán trên giao diện Point of Sale (POS), qua đó kiểm soát những mặt hàng nào sẽ xuất hiện trên menu POS. \\
\hline
Actor & US-01 (Quản lý nhà hàng) \\
\hline
Priority & Must Have \\
\hline
Trigger & - Cần đưa một sản phẩm mới lên bán trên POS. \newline - Cần tạm thời ẩn một sản phẩm khỏi POS mà không cần lưu trữ. \\
\hline
Pre-Condition & - US-01 đang xem form chi tiết sản phẩm ở chế độ Sửa. \\
\hline
Post-Condition & - Trạng thái "Available in POS" của sản phẩm được cập nhật. \newline - Thay đổi này ảnh hưởng đến việc sản phẩm có hiển thị trên POS hay không sau khi đồng bộ. \\
\hline
\multicolumn{2}{|c|}{\textbf{2.2. Luồng thực thi (Flow)}} \\
\hline
\textbf{Mục} & \textbf{Nội dung} \\
\hline
Basic Flow & 1. US-01 đang ở form chi tiết Sản phẩm, chế độ Sửa. \newline 2. US-01 chuyển đến tab "Point of Sale" (hoặc "Sales"). \newline 3. US-01 tìm ô kiểm "Available in POS" (Có sẵn trong POS). \newline 4. US-01 đánh dấu (tick) để cho phép bán trên POS, hoặc bỏ đánh dấu để ẩn khỏi POS. \newline 5. US-01 chọn "Lưu". \newline 6. Hệ thống lưu thay đổi. \\
\hline
Alternative Flow & Không có. \\
\hline
Exception Flow & \textbf{6a. Lỗi lưu:} Hệ thống báo lỗi khi lưu. \\
\hline
\multicolumn{2}{|c|}{\textbf{2.3. Thông tin bổ sung (Additional Information)}} \\
\hline
\textbf{Mục} & \textbf{Nội dung} \\
\hline
Business Rule & - \textbf{BR-UC2.16-1:} Chỉ sản phẩm được đánh dấu "Available in POS" mới hiển thị trên menu POS. \\
\hline
Non-Functional Requirement & - \textbf{NFR-UC2.16-1 (Usability):} Ô kiểm dễ tìm. \newline - \textbf{NFR-UC2.16-2 (Integration):} Cấu hình phải đồng bộ đúng xuống POS. \\
\hline
\end{longtable}

\subsubsection{Use Case UC-MD02-17: Gán Sản phẩm vào Danh mục POS}

\begin{longtable}{|m{4cm}|p{11cm}|}
\caption{Đặc tả Use Case UC-MD02-17: Gán Sản phẩm vào Danh mục POS} \label{tab:uc_md02_17_revised} \\
\hline
\multicolumn{2}{|c|}{\textbf{2.1. Tóm tắt (Summary)}} \\
\hline
\textbf{Mục} & \textbf{Nội dung} \\
\hline
\endhead % Header cho các trang tiếp theo
\hline
\endfoot % Footer cho bảng
\hline
\endlastfoot % Footer cho trang cuối cùng
Use Case Name & Gán Sản phẩm vào Danh mục POS \\
\hline
Use Case ID & UC-MD02-17 \\
\hline
Use Case Description & Cho phép Quản lý nhà hàng (US-01) chỉ định một sản phẩm (đã được đánh dấu "Available in POS") sẽ thuộc về (các) Danh mục POS nào, để sản phẩm đó được hiển thị đúng nhóm trên giao diện chọn món của POS. \\
\hline
Actor & US-01 (Quản lý nhà hàng) \\
\hline
Priority & Must Have \\
\hline
Trigger & Cần phân loại một sản phẩm vào đúng nhóm trên menu POS. \\
\hline
Pre-Condition & - US-01 đang xem form chi tiết sản phẩm ở chế độ Sửa. \newline - Sản phẩm đã được đánh dấu "Available in POS" (UC-MD02-16). \newline - Các Danh mục POS liên quan đã được tạo (UC-MD02-07). \\
\hline
Post-Condition & - Liên kết giữa sản phẩm và (các) Danh mục POS được cập nhật. \newline - Sản phẩm sẽ hiển thị trong (các) danh mục đó trên POS sau khi đồng bộ. \\
\hline
\multicolumn{2}{|c|}{\textbf{2.2. Luồng thực thi (Flow)}} \\
\hline
\textbf{Mục} & \textbf{Nội dung} \\
\hline
Basic Flow & 1. US-01 đang ở form chi tiết Sản phẩm, chế độ Sửa, tab "Point of Sale". \newline 2. US-01 tìm trường "POS Category" (Danh mục POS). \newline 3. US-01 nhấp vào trường này và chọn một Danh mục POS từ danh sách thả xuống. \newline 4. US-01 chọn "Lưu". \newline 5. Hệ thống lưu thay đổi. \\
\hline
Alternative Flow & \textbf{3a. Gán nhiều danh mục (nếu hệ thống hỗ trợ trường Many2many riêng):} \newline    1. US-01 thao tác trên trường cho phép chọn nhiều danh mục để gán sản phẩm. \\
\hline
Exception Flow & \textbf{3b. Danh mục không tồn tại:} Danh sách trống nếu chưa tạo danh mục. \newline \textbf{5a. Lỗi lưu.} \\
\hline
\multicolumn{2}{|c|}{\textbf{2.3. Thông tin bổ sung (Additional Information)}} \\
\hline
\textbf{Mục} & \textbf{Nội dung} \\
\hline
Business Rule & - \textbf{BR-UC2.17-1:} Sản phẩm phải được gán vào ít nhất một Danh mục POS để hiển thị có tổ chức trên POS (trừ khi có khu vực "Chưa phân loại"). \\
\hline
Non-Functional Requirement & - \textbf{NFR-UC2.17-1 (Usability):} Việc chọn danh mục phải dễ dàng. \\
\hline
\end{longtable}

\subsubsection{Use Case UC-MD02-18: Tải lên/Thay đổi Hình ảnh Sản phẩm}

\begin{longtable}{|m{4cm}|p{11cm}|}
\caption{Đặc tả Use Case UC-MD02-18: Tải lên/Thay đổi Hình ảnh Sản phẩm} \label{tab:uc_md02_18_revised} \\
\hline
\multicolumn{2}{|c|}{\textbf{2.1. Tóm tắt (Summary)}} \\
\hline
\textbf{Mục} & \textbf{Nội dung} \\
\hline
\endhead % Header cho các trang tiếp theo
\hline
\endfoot % Footer cho bảng
\hline
\endlastfoot % Footer cho trang cuối cùng
Use Case Name & Tải lên/Thay đổi Hình ảnh Sản phẩm \\
\hline
Use Case ID & UC-MD02-18 \\
\hline
Use Case Description & Cho phép Quản lý nhà hàng (US-01) tải lên một hình ảnh mới hoặc thay thế hình ảnh hiện có cho một sản phẩm. \\
\hline
Actor & US-01 (Quản lý nhà hàng) \\
\hline
Priority & Should Have \\
\hline
Trigger & Cần thêm/cập nhật ảnh minh họa cho sản phẩm. \\
\hline
Pre-Condition & - US-01 đang xem form chi tiết sản phẩm ở chế độ Sửa. \newline - Có sẵn tệp hình ảnh phù hợp trên máy tính. \\
\hline
Post-Condition & - Sản phẩm được cập nhật với hình ảnh mới. \newline - Ảnh sẽ hiển thị trên các giao diện liên quan. \\
\hline
\multicolumn{2}{|c|}{\textbf{2.2. Luồng thực thi (Flow)}} \\
\hline
\textbf{Mục} & \textbf{Nội dung} \\
\hline
Basic Flow & 1. US-01 đang ở form chi tiết Sản phẩm, chế độ Sửa. \newline 2. US-01 nhấp vào khu vực hình ảnh (thường có biểu tượng máy ảnh hoặc bút chì). \newline 3. Hệ thống mở hộp thoại chọn tệp. US-01 chọn tệp ảnh và nhấn "Open". \newline 4. Hệ thống kiểm tra tệp (định dạng, kích thước - BR-UC2.18-1, BR-UC2.18-2). \newline 5. Nếu hợp lệ, hệ thống tải lên và hiển thị ảnh xem trước. \newline 6. US-01 chọn "Lưu". \newline 7. Hệ thống lưu ảnh mới cho sản phẩm. \\
\hline
Alternative Flow & Không có. \\
\hline
Exception Flow & \textbf{4a. Tệp không hợp lệ (định dạng/kích thước):} Hệ thống báo lỗi, yêu cầu chọn lại. \newline \textbf{5a. Lỗi tải tệp lên server.} \newline \textbf{7a. Lỗi lưu sản phẩm.} \\
\hline
\multicolumn{2}{|c|}{\textbf{2.3. Thông tin bổ sung (Additional Information)}} \\
\hline
\textbf{Mục} & \textbf{Nội dung} \\
\hline
Business Rule & - \textbf{BR-UC2.18-1:} Hỗ trợ các định dạng ảnh phổ biến (JPG, PNG...). \newline - \textbf{BR-UC2.18-2:} Có giới hạn kích thước tệp tối đa. \newline - \textbf{BR-UC2.18-3:} Ảnh tải lên ở cấp Sản phẩm Gốc. Biến thể dùng chung ảnh này. \\
\hline
Non-Functional Requirement & - \textbf{NFR-UC2.18-1 (Usability):} Tải ảnh đơn giản. \newline - \textbf{NFR-UC2.18-2 (Performance):} Tải ảnh và lưu nhanh. \\
\hline
\end{longtable}

\subsubsection{Use Case UC-MD02-19: Xóa Hình ảnh Sản phẩm}

\begin{longtable}{|m{4cm}|p{11cm}|}
\caption{Đặc tả Use Case UC-MD02-19: Xóa Hình ảnh Sản phẩm} \label{tab:uc_md02_19_revised} \\
\hline
\multicolumn{2}{|c|}{\textbf{2.1. Tóm tắt (Summary)}} \\
\hline
\textbf{Mục} & \textbf{Nội dung} \\
\hline
\endhead % Header cho các trang tiếp theo
\hline
\endfoot % Footer cho bảng
\hline
\endlastfoot % Footer cho trang cuối cùng
Use Case Name & Xóa Hình ảnh Sản phẩm \\
\hline
Use Case ID & UC-MD02-19 \\
\hline
Use Case Description & Cho phép Quản lý nhà hàng (US-01) xóa bỏ hình ảnh hiện tại đang được liên kết với một sản phẩm. \\
\hline
Actor & US-01 (Quản lý nhà hàng) \\
\hline
Priority & Should Have \\
\hline
Trigger & Hình ảnh sản phẩm không còn phù hợp hoặc không muốn hiển thị ảnh nữa. \\
\hline
Pre-Condition & - US-01 đang xem form chi tiết sản phẩm ở chế độ Sửa. \newline - Sản phẩm đang có hình ảnh. \\
\hline
Post-Condition & - Liên kết hình ảnh bị xóa khỏi sản phẩm. \newline - Sản phẩm sẽ hiển thị ảnh mặc định (placeholder) hoặc không có ảnh. \\
\hline
\multicolumn{2}{|c|}{\textbf{2.2. Luồng thực thi (Flow)}} \\
\hline
\textbf{Mục} & \textbf{Nội dung} \\
\hline
Basic Flow & 1. US-01 đang ở form chi tiết Sản phẩm, chế độ Sửa. \newline 2. US-01 di chuột vào khu vực hình ảnh sản phẩm hiện tại. \newline 3. Biểu tượng xóa ảnh (ví dụ: thùng rác, dấu 'x') xuất hiện. US-01 nhấp vào đó. \newline 4. Hệ thống (có thể) yêu cầu xác nhận xóa ảnh. US-01 xác nhận. \newline 5. Hệ thống xóa liên kết ảnh, khung ảnh trở thành trống hoặc hiển thị ảnh placeholder. \newline 6. US-01 chọn "Lưu". \newline 7. Hệ thống lưu thay đổi (sản phẩm không còn ảnh). \\
\hline
Alternative Flow & Không có. \\
\hline
Exception Flow & \textbf{7a. Lỗi lưu sản phẩm.} \\
\hline
\multicolumn{2}{|c|}{\textbf{2.3. Thông tin bổ sung (Additional Information)}} \\
\hline
\textbf{Mục} & \textbf{Nội dung} \\
\hline
Business Rule & - \textbf{BR-UC2.19-1:} Việc xóa ảnh là không thể hoàn tác trực tiếp (trừ khi tải lại ảnh cũ). \\
\hline
Non-Functional Requirement & - \textbf{NFR-UC2.19-1 (Usability):} Thao tác xóa ảnh phải rõ ràng. \\
\hline
\end{longtable}

\subsubsection{Use Case UC-MD02-20: Gán Danh mục Sản phẩm vào Máy in Bếp/KDS}

\begin{longtable}{|m{4cm}|p{11cm}|}
\caption{Đặc tả Use Case UC-MD02-20: Gán Danh mục Sản phẩm vào Máy in Bếp/KDS} \label{tab:uc_md02_20_revised} \\
\hline
\multicolumn{2}{|c|}{\textbf{2.1. Tóm tắt (Summary)}} \\
\hline
\textbf{Mục} & \textbf{Nội dung} \\
\hline
\endhead % Header cho các trang tiếp theo
\hline
\endfoot % Footer cho bảng
\hline
\endlastfoot % Footer cho trang cuối cùng
Use Case Name & Gán Danh mục Sản phẩm vào Máy in Bếp/KDS \\
\hline
Use Case ID & UC-MD02-20 \\
\hline
Use Case Description & Cho phép Quản lý nhà hàng (US-01) cấu hình trong cài đặt Point of Sale để chỉ định những Danh mục Sản phẩm POS nào sẽ được gửi đến một Máy in Bếp hoặc Màn hình KDS cụ thể khi có đơn hàng. \\
\hline
Actor & US-01 (Quản lý nhà hàng) \\
\hline
Priority & Must Have (nếu có nhiều điểm chuẩn bị/in) \\
\hline
Trigger & Cần thiết lập hoặc thay đổi quy tắc định tuyến đơn hàng cho các trạm bếp/bar. \\
\hline
Pre-Condition & - US-01 đã đăng nhập với quyền quản trị cấu hình POS. \newline - Các thiết bị Máy in Bếp/KDS đã được khai báo trong cấu hình POS (thường qua IoT Box). \newline - Các Danh mục POS liên quan đã được tạo (UC-MD02-07). \\
\hline
Post-Condition & - Quy tắc định tuyến (Danh mục POS -> Thiết bị Bếp/KDS) được lưu. \newline - Đơn hàng từ POS sẽ được gửi đúng nơi dựa trên cấu hình này. \\
\hline
\multicolumn{2}{|c|}{\textbf{2.2. Luồng thực thi (Flow)}} \\
\hline
\textbf{Mục} & \textbf{Nội dung} \\
\hline
Basic Flow & 1. US-01 truy cập Cấu hình Point of Sale, chọn một cấu hình POS cụ thể. \newline 2. US-01 tìm đến mục "Order Printers" (Máy in Đơn hàng) hoặc "Kitchen Display" (KDS). \newline 3. US-01 chọn một Máy in/KDS đã khai báo để chỉnh sửa. \newline 4. Trong form cấu hình của Máy in/KDS đó, US-01 tìm trường "Printed Product Categories" (hoặc tương tự). \newline 5. US-01 nhấp vào trường này và chọn (tick) vào (các) Danh mục POS mà món ăn thuộc các danh mục đó cần được gửi đến thiết bị này. \newline 6. US-01 xác nhận lựa chọn. \newline 7. US-01 lưu cấu hình cho Máy in/KDS đó. \newline 8. US-01 lưu lại toàn bộ cấu hình POS. \newline 9. Hệ thống báo thành công. \\
\hline
Alternative Flow & \textbf{5a. Bỏ chọn danh mục:} Loại bỏ một danh mục khỏi định tuyến của thiết bị. \\
\hline
Exception Flow & \textbf{2a. Chưa khai báo Máy in/KDS.} \newline \textbf{5b. Chưa có Danh mục POS nào.} \newline \textbf{8a. Lỗi lưu cấu hình POS.} \\
\hline
\multicolumn{2}{|c|}{\textbf{2.3. Thông tin bổ sung (Additional Information)}} \\
\hline
\textbf{Mục} & \textbf{Nội dung} \\
\hline
Business Rule & - \textbf{BR-UC2.20-1:} Định tuyến dựa trên Danh mục POS của sản phẩm. \newline - \textbf{BR-UC2.20-2:} Thay đổi cấu hình này yêu cầu POS client đồng bộ lại. \\
\hline
Non-Functional Requirement & - \textbf{NFR-UC2.20-1 (Usability):} Giao diện cấu hình định tuyến phải rõ ràng. \newline - \textbf{NFR-UC2.20-2 (Integration):} Cấu hình phải được IoT Box/dịch vụ in diễn giải đúng. \\
\hline
\end{longtable}

