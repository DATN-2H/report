\subsection{Module MD-09: Quản lý Phiên \& Báo cáo}
\subsubsection{Use Case UC-MD09-01: Xem Danh sách các Phiên POS đã Đóng}
\begin{longtable}{|m{4cm}|p{11cm}|}
\caption{Đặc tả Use Case UC-MD09-01: Xem Danh sách các Phiên POS đã Đóng} \label{tab:uc_md09_01_corrected} \\
\hline
\multicolumn{2}{|c|}{\textbf{2.1. Tóm tắt (Summary)}} \\
\hline
\textbf{Mục} & \textbf{Nội dung} \\
\hline
\endhead % Header cho các trang tiếp theo
\midrule
\endfoot % Footer cho bảng
\bottomrule
\endlastfoot % Footer cho trang cuối cùng
Use Case Name & Xem Danh sách các Phiên POS đã Đóng \\
\hline
Use Case ID & UC-MD09-01 \\
\hline
Use Case Description & Cho phép Quản lý nhà hàng (US-01) hoặc Kế toán (US-06) xem danh sách các phiên làm việc Point of Sale (POS) đã được đóng trong hệ thống, với các thông tin tóm tắt của mỗi phiên. \\
\hline
Actor & US-01 (Quản lý nhà hàng), US-06 (Kế toán) \\
\hline
Priority & Must Have \\
\hline
Trigger & Người dùng cần xem lại lịch sử các phiên POS đã hoạt động để chọn một phiên cụ thể xem chi tiết hoặc để có cái nhìn tổng quan. \\
\hline
Pre-Condition & - Người dùng đã đăng nhập vào hệ thống với quyền truy cập vào thông tin các phiên POS (thường trong module Point of Sale > Orders > Sessions). \newline - Có ít nhất một phiên POS đã được đóng trước đó (kết quả của UC-MD05-18). \\
\hline
Post-Condition & - Danh sách các phiên POS đã đóng được hiển thị cho người dùng. \newline - Người dùng có thể thấy các thông tin tóm tắt của mỗi phiên và có thể chọn một phiên để xem báo cáo chi tiết (UC-MD09-02). \\
\hline
\multicolumn{2}{|c|}{\textbf{2.2. Luồng thực thi (Flow)}} \\
\hline
\textbf{Mục} & \textbf{Nội dung} \\
\hline
Basic Flow & 1. Người dùng (US-01 hoặc US-06) truy cập vào module Point of Sale từ giao diện chính. \newline 2. Người dùng điều hướng đến mục "Đơn hàng" (Orders) và chọn "Phiên làm việc" (Sessions). \newline 3. Hệ thống hiển thị danh sách tất cả các phiên POS đã được tạo. Mặc định, danh sách có thể được sắp xếp theo ngày giờ đóng phiên giảm dần. \newline 4. Với mỗi phiên trong danh sách, hệ thống hiển thị các thông tin tóm tắt như: \newline    - Tên/Mã tham chiếu phiên (Session ID). \newline    - Tên Điểm bán hàng (Point of Sale) mà phiên đó thuộc về. \newline    - Tên Nhân viên mở phiên (Opened By). \newline    - Thời gian mở phiên (Opening Date). \newline    - Thời gian đóng phiên (Closing Date - nếu đã đóng). \newline    - Trạng thái của phiên (ví dụ: "In Progress", "Closed", "Posted"). \newline    - Tổng giá trị các đơn hàng của phiên (Total Orders Amount). \newline 5. Người dùng xem xét danh sách. \\
\hline
Alternative Flow & \textbf{3a. Lọc danh sách phiên:} \newline    1. Người dùng sử dụng các bộ lọc (Filters) có sẵn để thu hẹp danh sách, ví dụ: lọc theo Trạng thái ("Đã đóng" - Closed), lọc theo Điểm bán hàng, lọc theo Ngày mở/đóng phiên, lọc theo Nhân viên mở phiên. \newline    2. Hệ thống áp dụng bộ lọc và hiển thị lại danh sách kết quả. \newline \textbf{3b. Tìm kiếm phiên:} \newline    1. Người dùng nhập từ khóa (ví dụ: Session ID, tên nhân viên) vào ô tìm kiếm. \newline    2. Hệ thống thực hiện tìm kiếm và hiển thị các phiên khớp với từ khóa. \newline \textbf{3c. Sắp xếp danh sách phiên:} \newline    1. Người dùng nhấp vào tiêu đề của các cột (ví dụ: "Thời gian đóng phiên") để sắp xếp danh sách tăng dần hoặc giảm dần theo cột đó. \\
\hline
Exception Flow & \textbf{3d. Lỗi hệ thống khi tải danh sách phiên:} \newline    1. Hệ thống gặp lỗi kỹ thuật khi cố gắng tải danh sách các phiên POS. \newline    2. Hệ thống hiển thị một thông báo lỗi chung. \newline    3. Use Case kết thúc không thành công. \newline \textbf{3e. Không có phiên nào đã đóng (hoặc không có phiên nào khớp bộ lọc):} \newline    1. Nếu chưa có phiên POS nào được đóng trong hệ thống, hoặc không có phiên nào phù hợp với các tiêu chí lọc mà người dùng đã áp dụng. \newline    2. Hệ thống hiển thị danh sách trống hoặc một thông báo "Không tìm thấy phiên làm việc nào phù hợp." \\
\hline
\multicolumn{2}{|c|}{\textbf{2.3. Thông tin bổ sung (Additional Information)}} \\
\hline
\textbf{Mục} & \textbf{Nội dung} \\
\hline
Business Rule & - \textbf{BR-UC9.1-1:} Danh sách phải hiển thị chính xác tất cả các phiên POS đã được ghi nhận trong hệ thống, tuân theo các bộ lọc mà người dùng áp dụng. \newline - \textbf{BR-UC9.1-2:} Các thông tin tóm tắt hiển thị cho mỗi phiên phải là những thông tin cốt lõi giúp người dùng nhanh chóng nhận diện và lựa chọn phiên cần xem xét. \\
\hline
Non-Functional Requirement & - \textbf{NFR-UC9.1-1 (Usability):} Giao diện danh sách các phiên POS phải rõ ràng, dễ đọc. Các chức năng lọc, tìm kiếm, và sắp xếp phải dễ sử dụng và hiệu quả. \newline - \textbf{NFR-UC9.1-2 (Performance):} Thời gian tải danh sách các phiên POS phải nhanh chóng. \newline - \textbf{NFR-UC9.1-3 (Security):} Chỉ người dùng có quyền hạn phù hợp mới được phép truy cập danh sách và thông tin chi tiết của các phiên POS. \\
\hline
\end{longtable}

\subsubsection{Use Case UC-MD09-02: Xem Chi tiết một Phiên POS đã Đóng (Báo cáo Doanh thu Phiên)}
\begin{longtable}{|m{4cm}|p{11cm}|}
\caption{Đặc tả Use Case UC-MD09-02: Xem Chi tiết một Phiên POS đã Đóng (Báo cáo Doanh thu Phiên)} \label{tab:uc_md09_02_corrected} \\
\hline
\multicolumn{2}{|c|}{\textbf{2.1. Tóm tắt (Summary)}} \\
\hline
\textbf{Mục} & \textbf{Nội dung} \\
\hline
\endhead % Header cho các trang tiếp theo
\midrule
\endfoot % Footer cho bảng
\bottomrule
\endlastfoot % Footer cho trang cuối cùng
Use Case Name & Xem Chi tiết một Phiên POS đã Đóng (Báo cáo Doanh thu Phiên) \\
\hline
Use Case ID & UC-MD09-02 \\
\hline
Use Case Description & Cho phép Quản lý nhà hàng (US-01) hoặc Kế toán (US-06) xem lại toàn bộ thông tin tổng kết chi tiết của một phiên làm việc Point of Sale (POS) cụ thể đã được đóng trước đó. \\
\hline
Actor & US-01 (Quản lý nhà hàng), US-06 (Kế toán) \\
\hline
Priority & Must Have \\
\hline
Trigger & - Người dùng đã chọn một phiên POS cụ thể từ danh sách (UC-MD09-01) và muốn xem chi tiết. \newline - Cần đối soát doanh thu của một ngày/ca làm việc. \newline - Cần kiểm tra lại thông tin thanh toán/chênh lệch tiền mặt của một phiên đã qua. \\
\hline
Pre-Condition & - Người dùng (US-01/US-06) đã đăng nhập với quyền truy cập báo cáo chi tiết phiên POS. \newline - Phiên POS cần xem đã ở trạng thái "Đã đóng" (Closed) và đã được tổng kết (UC-MD05-18). \\
\hline
Post-Condition & - Báo cáo chi tiết về tài chính và hoạt động của phiên POS đã chọn được hiển thị. \newline - Người dùng nắm được hiệu quả hoạt động, tình hình thu chi của phiên đó. \\
\hline
\multicolumn{2}{|c|}{\textbf{2.2. Luồng thực thi (Flow)}} \\
\hline
\textbf{Mục} & \textbf{Nội dung} \\
\hline
Basic Flow & 1. Người dùng (US-01/US-06) đang xem danh sách các phiên POS đã đóng (UC-MD09-01). \newline 2. Người dùng chọn một phiên POS cụ thể. \newline 3. Hệ thống truy xuất dữ liệu tổng kết của phiên đã chọn. \newline 4. Hệ thống hiển thị giao diện báo cáo chi tiết, bao gồm: \newline    a. \textbf{Thông tin Chung Phiên:} Mã phiên, POS, Người mở/đóng, Thời gian mở/đóng. \newline    b. \textbf{Tóm tắt Doanh thu:} Tổng đơn hàng, Tổng giảm giá, Doanh thu thuần trước thuế, Tổng thuế, Tổng Doanh thu cuối cùng. \newline    c. \textbf{Chi tiết Thanh toán:} Tổng tiền theo từng Phương thức Thanh toán (Tiền mặt, Ví...). \newline    d. \textbf{Kiểm soát Tiền mặt (nếu có):} Số dư đầu ca, Tiền mặt dự kiến, Tiền mặt thực tế, Chênh lệch. \newline    e. \textbf{Thông tin Hoạt động:} Tổng số đơn, Tổng tiền boa. \newline    f. (Tùy chọn) Liên kết xem danh sách đơn hàng của phiên. \newline 5. Người dùng xem xét báo cáo. \\
\hline
Alternative Flow & \textbf{5a. In báo cáo chi tiết phiên:} \newline    1. Giao diện có nút "In Báo cáo" / "Xuất PDF". \newline    2. Người dùng nhấp vào. \newline    3. Hệ thống tạo file PDF báo cáo để tải về/in. \newline \textbf{5b. Điều hướng đến báo cáo liên quan:} \newline    1. Có thể có liên kết nhanh đến các báo cáo khác (ví dụ: bán hàng theo sản phẩm của phiên này). \\
\hline
Exception Flow & \textbf{3a. Lỗi tải dữ liệu chi tiết phiên:} \newline    1. Hệ thống gặp lỗi kỹ thuật. \newline    2. Hệ thống báo lỗi chung. Người dùng không xem được báo cáo. \newline \textbf{3b. Dữ liệu phiên không nhất quán/lỗi (hiếm):} \newline    1. Do sự cố trước đó, dữ liệu tổng kết bị thiếu/sai. \newline    2. Báo cáo hiển thị số liệu không chính xác. Cần quản trị viên kiểm tra. \\
\hline
\multicolumn{2}{|c|}{\textbf{2.3. Thông tin bổ sung (Additional Information)}} \\
\hline
\textbf{Mục} & \textbf{Nội dung} \\
\hline
Business Rule & - \textbf{BR-UC9.2-1:} Báo cáo phải phản ánh chính xác số liệu đã tổng kết khi đóng phiên (UC-MD05-18). \newline - \textbf{BR-UC9.2-2:} Dữ liệu thanh toán theo phương thức phải khớp tổng tiền thu. \newline - \textbf{BR-UC9.2-3:} Nếu có kiểm soát tiền mặt, báo cáo phải rõ ràng về đối chiếu và chênh lệch. \newline - \textbf{BR-UC9.2-4:} Dữ liệu này là cơ sở đối soát doanh thu và cho kế toán. \\
\hline
Non-Functional Requirement & - \textbf{NFR-UC9.2-1 (Usability):} Báo cáo rõ ràng, có cấu trúc, số liệu quan trọng dễ thấy. \newline - \textbf{NFR-UC9.2-2 (Performance):} Tải báo cáo chi tiết nhanh. \newline - \textbf{NFR-UC9.2-3 (Accuracy):} Số liệu chính xác tuyệt đối. \newline - \textbf{NFR-UC9.2-4 (Security):} Chỉ người có quyền mới được xem báo cáo tài chính này. \\
\hline
\end{longtable}

\subsubsection{Use Case UC-MD09-03: Xem Báo cáo Tổng hợp Bán hàng theo Sản phẩm/Danh mục POS}
\begin{longtable}{|m{4cm}|p{11cm}|}
\caption{Đặc tả Use Case UC-MD09-03: Xem Báo cáo Tổng hợp Bán hàng theo Sản phẩm/Danh mục POS} \label{tab:uc_md09_03_corrected} \\
\hline
\multicolumn{2}{|c|}{\textbf{2.1. Tóm tắt (Summary)}} \\
\hline
\textbf{Mục} & \textbf{Nội dung} \\
\hline
\endhead % Header cho các trang tiếp theo
\midrule
\endfoot % Footer cho bảng
\bottomrule
\endlastfoot % Footer cho trang cuối cùng
Use Case Name & Xem Báo cáo Tổng hợp Bán hàng theo Sản phẩm/Danh mục POS \\
\hline
Use Case ID & UC-MD09-03 \\
\hline
Use Case Description & Cung cấp cho Quản lý nhà hàng (US-01) hoặc Kế toán (US-06) báo cáo thống kê chi tiết về số lượng đã bán ra và tổng doanh thu của từng Sản phẩm (món ăn/đồ uống) hoặc được nhóm theo từng Danh mục Sản phẩm POS, trong một khoảng thời gian do người dùng lựa chọn. \\
\hline
Actor & US-01 (Quản lý nhà hàng), US-06 (Kế toán) \\
\hline
Priority & Must Have \\
\hline
Trigger & - Cần phân tích hiệu quả bán hàng của từng món ăn. \newline - Cần xem xét cơ cấu doanh thu theo từng nhóm món. \newline - Cần dữ liệu để lập kế hoạch mua nguyên vật liệu. \\
\hline
Pre-Condition & - Người dùng đã đăng nhập với quyền truy cập báo cáo bán hàng POS/Sales. \newline - Đã có dữ liệu giao dịch bán hàng từ các phiên POS đã đóng. \newline - Các Sản phẩm và Danh mục POS đã được định nghĩa. \\
\hline
Post-Condition & - Báo cáo thống kê bán hàng theo sản phẩm/danh mục POS được hiển thị cho khoảng thời gian đã chọn. \newline - Người dùng có thông tin để đánh giá hiệu quả kinh doanh. \\
\hline
\multicolumn{2}{|c|}{\textbf{2.2. Luồng thực thi (Flow)}} \\
\hline
\textbf{Mục} & \textbf{Nội dung} \\
\hline
Basic Flow (Xem báo cáo theo Sản phẩm) & 1. Người dùng (US-01/US-06) truy cập "Báo cáo" (Reporting) của POS hoặc Sales. \newline 2. Người dùng chọn loại báo cáo "Bán hàng theo Sản phẩm" (Sales by Product) hoặc "Phân tích Đơn hàng POS". \newline 3. Người dùng chọn Khoảng thời gian (Date Range) muốn xem (ví dụ: Hôm nay, Tuần này, Tùy chọn). \newline 4. (Tùy chọn) Người dùng có thể chọn các tùy chọn hiển thị/nhóm. Mặc định theo Sản phẩm. \newline 5. Người dùng nhấn "Xem báo cáo" / "Apply Filters". \newline 6. Hệ thống truy vấn, tổng hợp dữ liệu từ các dòng đơn hàng POS đã thanh toán trong khoảng thời gian đã chọn. \newline 7. Hệ thống hiển thị bảng báo cáo, mỗi dòng là một Sản phẩm/Biến thể, với các cột: Tên Sản phẩm, Số lượng bán, Doanh thu thuần, (Tùy chọn) Tổng giảm giá, (Tùy chọn) Tổng thuế, Tổng doanh thu. \newline 8. Có dòng tổng cộng ở cuối. \newline 9. Người dùng xem xét báo cáo. \\
\hline
Alternative Flow & \textbf{4a. Xem báo cáo theo Danh mục POS:} \newline    1. Người dùng chọn nhóm theo "Danh mục Sản phẩm POS". \newline    2. Hệ thống hiển thị báo cáo với mỗi dòng là một Danh mục POS, các cột số liệu là tổng của các sản phẩm thuộc danh mục đó. \newline \textbf{9a. Sắp xếp/Lọc/Xem Biểu đồ:} \newline    1. Giao diện cho phép sắp xếp theo cột, áp dụng bộ lọc bổ sung (theo POS, Nhân viên), hoặc chuyển sang dạng xem biểu đồ. \\
\hline
Exception Flow & \textbf{6a. Lỗi hệ thống khi truy vấn/tổng hợp dữ liệu:} \newline    1. Hệ thống báo lỗi chung. \newline \textbf{6b. Không có dữ liệu bán hàng phù hợp:} \newline    1. Hệ thống hiển thị báo cáo trống hoặc thông báo "Không có dữ liệu." \\
\hline
\multicolumn{2}{|c|}{\textbf{2.3. Thông tin bổ sung (Additional Information)}} \\
\hline
\textbf{Mục} & \textbf{Nội dung} \\
\hline
Business Rule & - \textbf{BR-UC9.3-1 (V3):} Báo cáo tổng hợp từ các đơn hàng đã thanh toán, đã đóng trong kỳ. Đơn hủy không tính vào doanh thu. \newline - \textbf{BR-UC9.3-2 (V3):} Định nghĩa các cột số liệu (Doanh thu thuần, Tổng doanh thu...) phải rõ ràng. \newline - \textbf{BR-UC9.3-3 (V3):} Nếu sản phẩm có biến thể, cho phép xem theo biến thể hoặc tổng hợp theo sản phẩm gốc. \\
\hline
Non-Functional Requirement & - \textbf{NFR-UC9.3-1 (V3 - Usability):} Giao diện báo cáo thân thiện, dễ tùy chỉnh. \newline - \textbf{NFR-UC9.3-2 (V3 - Performance):} Tạo báo cáo nhanh, kể cả với dữ liệu lớn. \newline - \textbf{NFR-UC9.3-3 (V3 - Accuracy):} Số liệu thống kê phải chính xác 100\%. \\
\hline
\end{longtable}

\subsubsection{Use Case UC-MD09-04: Xem Báo cáo Hiệu suất Bán hàng của Nhân viên POS}
\begin{longtable}{|m{4cm}|p{11cm}|}
\caption{Đặc tả Use Case UC-MD09-04: Xem Báo cáo Hiệu suất Bán hàng của Nhân viên POS} \label{tab:uc_md09_04_corrected} \\
\hline
\multicolumn{2}{|c|}{\textbf{2.1. Tóm tắt (Summary)}} \\
\hline
\textbf{Mục} & \textbf{Nội dung} \\
\hline
\endhead % Header cho các trang tiếp theo
\midrule
\endfoot % Footer cho bảng
\bottomrule
\endlastfoot % Footer cho trang cuối cùng
Use Case Name & Xem Báo cáo Hiệu suất Bán hàng của Nhân viên POS \\
\hline
Use Case ID & UC-MD09-04 \\
\hline
Use Case Description & Cung cấp cho Quản lý nhà hàng (US-01) báo cáo thống kê về hoạt động bán hàng trên Point of Sale (POS) của từng nhân viên trong một khoảng thời gian do người dùng lựa chọn. Báo cáo thường bao gồm tổng doanh thu, số lượng đơn hàng đã xử lý, và có thể cả tổng tiền boa. \\
\hline
Actor & US-01 (Quản lý nhà hàng) \\
\hline
Priority & Should Have \\
\hline
Trigger & - Quản lý muốn đánh giá hiệu suất làm việc của từng nhân viên POS. \newline - Cần dữ liệu để xem xét khen thưởng, tính hoa hồng (nếu có), hoặc đào tạo. \\
\hline
Pre-Condition & - US-01 đã đăng nhập với quyền truy cập báo cáo POS/Nhân viên. \newline - Dữ liệu giao dịch POS đã được ghi nhận và liên kết đúng với nhân viên thực hiện. \\
\hline
Post-Condition & - Báo cáo thống kê hiệu suất theo từng nhân viên được hiển thị. \newline - Quản lý có thông tin để đánh giá và ra quyết định nhân sự. \\
\hline
\multicolumn{2}{|c|}{\textbf{2.2. Luồng thực thi (Flow)}} \\
\hline
\textbf{Mục} & \textbf{Nội dung} \\
\hline
Basic Flow & 1. US-01 truy cập "Báo cáo" (Reporting) của POS. \newline 2. US-01 chọn loại báo cáo "Doanh thu theo Nhân viên" (Sales by Employee/Salesperson). \newline 3. US-01 chọn Khoảng thời gian báo cáo. \newline 4. US-01 nhấn "Xem báo cáo". \newline 5. Hệ thống truy vấn, tổng hợp dữ liệu đơn hàng POS đã đóng trong kỳ, nhóm theo nhân viên xử lý. \newline 6. Hệ thống hiển thị bảng báo cáo, mỗi dòng là một Nhân viên, với các cột: Tên Nhân viên, Số lượng đơn hàng, Tổng Doanh thu, (Tùy chọn) Tổng tiền boa. \newline 7. Có thể có dòng tổng cộng. \newline 8. US-01 xem xét báo cáo. \\
\hline
Alternative Flow & \textbf{6a. Xem chi tiết đơn hàng của nhân viên:} US-01 có thể nhấp vào tên nhân viên để xem danh sách đơn hàng cụ thể. \newline \textbf{6b. Lọc theo nhân viên/vai trò.} \\
\hline
Exception Flow & Tương tự UC-MD09-03 (Lỗi truy vấn/tổng hợp, Không có dữ liệu). \\
\hline
\multicolumn{2}{|c|}{\textbf{2.3. Thông tin bổ sung (Additional Information)}} \\
\hline
\textbf{Mục} & \textbf{Nội dung} \\
\hline
Business Rule & - \textbf{BR-UC9.4-1 (V3):} Dữ liệu phải tổng hợp dựa trên việc ghi nhận chính xác nhân viên xử lý đơn hàng. \newline - \textbf{BR-UC9.4-2 (V3):} Các chỉ số hiệu suất cần được định nghĩa rõ ràng. \\
\hline
Non-Functional Requirement & - \textbf{NFR-UC9.4-1 (V3 - Usability):} Báo cáo dễ đọc, dễ so sánh. \newline - \textbf{NFR-UC9.4-2 (V3 - Performance):} Tạo báo cáo nhanh. \newline - \textbf{NFR-UC9.4-3 (V3 - Accuracy):} Số liệu chính xác. \newline - \textbf{NFR-UC9.4-4 (V3 - Security):} Thông tin hiệu suất nhân viên nhạy cảm, cần kiểm soát quyền truy cập. \\
\hline
\end{longtable}

\subsubsection{Use Case UC-MD09-05: Xem Báo cáo Quản lý Tiền Đặt cọc}
\begin{longtable}{|m{4cm}|p{11cm}|}
\caption{Đặc tả Use Case UC-MD09-05: Xem Báo cáo Quản lý Tiền Đặt cọc} \label{tab:uc_md09_05_corrected} \\
\hline
\multicolumn{2}{|c|}{\textbf{2.1. Tóm tắt (Summary)}} \\
\hline
\textbf{Mục} & \textbf{Nội dung} \\
\hline
\endhead % Header cho các trang tiếp theo
\midrule
\endfoot % Footer cho bảng
\bottomrule
\endlastfoot % Footer cho trang cuối cùng
Use Case Name & Xem Báo cáo Quản lý Tiền Đặt cọc \\
\hline
Use Case ID & UC-MD09-05 \\
\hline
Use Case Description & Cung cấp cho Quản lý nhà hàng (US-01) hoặc Kế toán (US-06) báo cáo tổng hợp về tình hình thu và sử dụng tiền đặt cọc từ các lượt đặt chỗ trong một khoảng thời gian. Bao gồm tổng cọc đã thu, đã áp dụng vào hóa đơn, và đã bị mất (do khách hủy). \\
\hline
Actor & US-01 (Quản lý nhà hàng), US-06 (Kế toán) \\
\hline
Priority & Must Have \\
\hline
Trigger & - Cần theo dõi dòng tiền đặt cọc. \newline - Đối soát doanh thu từ cọc. \newline - Phân tích tỷ lệ khách hủy sau khi cọc. \\
\hline
Pre-Condition & - Người dùng đã đăng nhập với quyền truy cập báo cáo Đặt chỗ/Tài chính. \newline - Hệ thống có chức năng đặt chỗ với yêu cầu đặt cọc và ghi nhận trạng thái (MD-03). \newline - Có dữ liệu về các lượt đặt chỗ đã thanh toán cọc, đã sử dụng hoặc đã hủy. \\
\hline
Post-Condition & - Báo cáo tổng hợp về tình hình tiền đặt cọc được hiển thị. \newline - Người dùng có thông tin để quản lý và đối soát dòng tiền này. \\
\hline
\multicolumn{2}{|c|}{\textbf{2.2. Luồng thực thi (Flow)}} \\
\hline
\textbf{Mục} & \textbf{Nội dung} \\
\hline
Basic Flow & 1. Người dùng (US-01/US-06) truy cập "Báo cáo" của module Đặt chỗ hoặc Kế toán. \newline 2. Người dùng chọn loại báo cáo "Tiền đặt cọc" (Deposits Report). \newline 3. Người dùng chọn Khoảng thời gian báo cáo. \newline 4. Người dùng nhấn "Xem báo cáo". \newline 5. Hệ thống truy vấn dữ liệu từ các bản ghi Đặt chỗ, giao dịch thanh toán cọc, và trạng thái cuối cùng của đặt chỗ trong khoảng thời gian đó. \newline 6. Hệ thống hiển thị báo cáo tổng hợp, gồm: \newline    - Tổng số tiền đặt cọc đã thu. \newline    - Tổng số tiền đặt cọc đã áp dụng vào hóa đơn. \newline    - Tổng số tiền đặt cọc bị mất/không hoàn lại. \newline    - (Tùy chọn) Số dư cọc chưa sử dụng. \newline    - (Tùy chọn) Danh sách chi tiết các giao dịch cọc. \newline 7. Người dùng xem xét báo cáo. \\
\hline
Alternative Flow & \textbf{6a. Phân tích chi tiết hơn:} Báo cáo có thể cho phép xem chi tiết các đặt chỗ ứng với từng loại giao dịch cọc. \\
\hline
Exception Flow & Tương tự UC-MD09-03 (Lỗi truy vấn/tổng hợp, Không có dữ liệu). \\
\hline
\multicolumn{2}{|c|}{\textbf{2.3. Thông tin bổ sung (Additional Information)}} \\
\hline
\textbf{Mục} & \textbf{Nội dung} \\
\hline
Business Rule & - \textbf{BR-UC9.5-1 (V3):} Báo cáo phải phân biệt rõ ràng giữa các trạng thái của tiền đặt cọc. \newline - \textbf{BR-UC9.5-2 (V3):} Dữ liệu phải tổng hợp chính xác từ các bản ghi liên quan. \newline - \textbf{BR-UC9.5-3 (V3):} Logic xác định tiền cọc bị mất phải dựa trên trạng thái hủy và chính sách hoàn cọc. \\
\hline
Non-Functional Requirement & - \textbf{NFR-UC9.5-1 (V3 - Accuracy):} Số liệu báo cáo tiền đặt cọc phải chính xác tuyệt đối. \newline - \textbf{NFR-UC9.5-2 (V3 - Usability):} Báo cáo trình bày rõ ràng, dễ hiểu. \newline - \textbf{NFR-UC9.5-3 (V3 - Performance):} Tốc độ tạo báo cáo chấp nhận được. \\
\hline
\end{longtable}

\subsubsection{Use Case UC-MD09-06: Xem Báo cáo Doanh thu theo Loại hình Đơn hàng}
\begin{longtable}{|m{4cm}|p{11cm}|}
\caption{Đặc tả Use Case UC-MD09-06: Xem Báo cáo Doanh thu theo Loại hình Đơn hàng} \label{tab:uc_md09_06_corrected} \\
\hline
\multicolumn{2}{|c|}{\textbf{2.1. Tóm tắt (Summary)}} \\
\hline
\textbf{Mục} & \textbf{Nội dung} \\
\hline
\endhead % Header cho các trang tiếp theo
\midrule
\endfoot % Footer cho bảng
\bottomrule
\endlastfoot % Footer cho trang cuối cùng
Use Case Name & Xem Báo cáo Doanh thu theo Loại hình Đơn hàng \\
\hline
Use Case ID & UC-MD09-06 \\
\hline
Use Case Description & Cung cấp cho Quản lý nhà hàng (US-01) hoặc Kế toán (US-06) một báo cáo phân tích tổng doanh thu (và có thể cả số lượng đơn hàng) được tạo ra từ mỗi loại hình phục vụ khác nhau mà nhà hàng hỗ trợ: Ăn tại chỗ (Eat-in), Mang về (Takeout), và Giao hàng (Delivery) trong một khoảng thời gian. \\
\hline
Actor & US-01 (Quản lý nhà hàng), US-06 (Kế toán) \\
\hline
Priority & Must Have \\
\hline
Trigger & - Cần phân tích hiệu quả kinh doanh và đóng góp doanh thu của từng kênh/loại hình phục vụ. \\
\hline
Pre-Condition & - Người dùng đã đăng nhập với quyền truy cập báo cáo POS/Bán hàng. \newline - Hệ thống POS có khả năng ghi nhận chính xác loại hình cho mỗi đơn hàng (Eat-in, Takeout, Delivery). \newline - Đã có dữ liệu giao dịch từ các loại hình đơn hàng khác nhau. \\
\hline
Post-Condition & - Báo cáo phân tích doanh thu theo từng loại hình được hiển thị. \newline - Người dùng có thông tin để so sánh hiệu quả giữa các kênh bán hàng. \\
\hline
\multicolumn{2}{|c|}{\textbf{2.2. Luồng thực thi (Flow)}} \\
\hline
\textbf{Mục} & \textbf{Nội dung} \\
\hline
Basic Flow & 1. Người dùng (US-01/US-06) truy cập "Báo cáo" của POS hoặc Sales. \newline 2. Người dùng chọn loại báo cáo "Doanh thu theo Loại hình" (Sales by Order Type). \newline 3. Người dùng chọn Khoảng thời gian báo cáo. \newline 4. Người dùng nhấn "Xem báo cáo". \newline 5. Hệ thống truy vấn dữ liệu đơn hàng POS đã đóng trong kỳ, nhóm theo "Loại hình đơn hàng". \newline 6. Hệ thống hiển thị báo cáo (bảng/biểu đồ) thể hiện: Loại hình, Tổng Doanh thu, (Tùy chọn) Số lượng đơn, Tỷ trọng doanh thu. \newline 7. Người dùng xem xét báo cáo. \\
\hline
Alternative Flow & \textbf{6a. Xem chi tiết hơn:} Có thể nhấp vào một loại hình để xem chi tiết hơn (ví dụ: sản phẩm bán chạy của kênh đó). \\
\hline
Exception Flow & Tương tự UC-MD09-03 (Lỗi truy vấn/tổng hợp, Không có dữ liệu). \\
\hline
\multicolumn{2}{|c|}{\textbf{2.3. Thông tin bổ sung (Additional Information)}} \\
\hline
\textbf{Mục} & \textbf{Nội dung} \\
\hline
Business Rule & - \textbf{BR-UC9.6-1 (V3):} Hệ thống phải ghi nhận và phân loại chính xác loại hình cho mỗi đơn hàng. \newline - \textbf{BR-UC9.6-2 (V3):} Báo cáo phải tổng hợp đúng doanh thu và số lượng đơn cho từng loại hình. \\
\hline
Non-Functional Requirement & - \textbf{NFR-UC9.6-1 (V3 - Usability):} Báo cáo so sánh trực quan. \newline - \textbf{NFR-UC9.6-2 (V3 - Performance):} Tạo báo cáo hiệu năng tốt. \newline - \textbf{NFR-UC9.6-3 (V3 - Accuracy):} Số liệu phân loại chính xác. \\
\hline
\end{longtable}

\subsubsection{Use Case UC-MD09-07: Xuất Dữ liệu từ các Báo cáo}
\begin{longtable}{|m{4cm}|p{11cm}|}
\caption{Đặc tả Use Case UC-MD09-07: Xuất Dữ liệu từ các Báo cáo} \label{tab:uc_md09_07_corrected} \\
\hline
\multicolumn{2}{|c|}{\textbf{2.1. Tóm tắt (Summary)}} \\
\hline
\textbf{Mục} & \textbf{Nội dung} \\
\hline
\endhead % Header cho các trang tiếp theo
\midrule
\endfoot % Footer cho bảng
\bottomrule
\endlastfoot % Footer cho trang cuối cùng
Use Case Name & Xuất Dữ liệu từ các Báo cáo \\
\hline
Use Case ID & UC-MD09-07 \\
\hline
Use Case Description & Cho phép Người dùng (US-01/US-06) đang xem một báo cáo trong hệ thống có thể xuất dữ liệu của báo cáo đó ra một tệp tin theo định dạng phổ biến như Excel (.xlsx) hoặc CSV (.csv). \\
\hline
Actor & US-01 (Quản lý nhà hàng), US-06 (Kế toán) \\
\hline
Priority & Should Have \\
\hline
Trigger & - Muốn lưu trữ bản sao báo cáo ngoại tuyến. \newline - Muốn chia sẻ dữ liệu với người không có tài khoản. \newline - Muốn phân tích sâu hơn bằng công cụ ngoài. \\
\hline
Pre-Condition & - Người dùng đang xem một giao diện báo cáo/danh sách dữ liệu có chức năng "Xuất". \\
\hline
Post-Condition & - Tệp tin chứa dữ liệu báo cáo được tải về máy người dùng theo định dạng đã chọn. \\
\hline
\multicolumn{2}{|c|}{\textbf{2.2. Luồng thực thi (Flow)}} \\
\hline
\textbf{Mục} & \textbf{Nội dung} \\
\hline
Basic Flow & 1. Người dùng (US-01/US-06) đang xem báo cáo/danh sách muốn xuất. \newline 2. Người dùng nhấp nút "Xuất" (Export) / "Tải xuống" (Download). \newline 3. Hệ thống (có thể) hiển thị tùy chọn: chọn trường dữ liệu, chọn định dạng tệp (Excel/CSV), đặt tên tệp. \newline 4. Người dùng chọn tùy chọn và nhấn "Xuất". \newline 5. Hệ thống xử lý, truy xuất dữ liệu. \newline 6. Hệ thống tạo tệp theo định dạng đã chọn. \newline 7. Hệ thống kích hoạt trình duyệt tải tệp về. \newline 8. Người dùng lưu tệp. \\
\hline
Alternative Flow & \textbf{3a. Xuất nhanh với định dạng mặc định:} Nút "Xuất" có thể xuất trực tiếp ra định dạng mặc định không cần qua bước tùy chọn. \newline \textbf{4a. Xuất chỉ dữ liệu đã chọn (nếu là danh sách):} Hệ thống hỏi muốn xuất tất cả hay chỉ các dòng đã chọn. \\
\hline
Exception Flow & \textbf{5a. Lỗi truy xuất dữ liệu / tạo tệp:} \newline    1. Hệ thống báo lỗi (dữ liệu quá lớn, lỗi định dạng...). \newline    2. Tệp không được tạo hoặc bị lỗi. \newline \textbf{5b. Không có dữ liệu để xuất:} Hệ thống báo "Không có dữ liệu để xuất." \\
\hline
\multicolumn{2}{|c|}{\textbf{2.3. Thông tin bổ sung (Additional Information)}} \\
\hline
\textbf{Mục} & \textbf{Nội dung} \\
\hline
Business Rule & - \textbf{BR-UC9.7-1 (V3):} Hỗ trợ xuất Excel (.xlsx) và CSV (.csv). \newline - \textbf{BR-UC9.7-2 (V3):} Dữ liệu xuất ra phải khớp với dữ liệu hiển thị (bao gồm bộ lọc/sắp xếp). \newline - \textbf{BR-UC9.7-3 (V3):} Cấu trúc cột trong tệp xuất nên tương ứng giao diện. \\
\hline
Non-Functional Requirement & - \textbf{NFR-UC9.7-1 (V3 - Usability):} Chức năng xuất dễ tìm, dễ dùng. \newline - \textbf{NFR-UC9.7-2 (V3 - Performance):} Tạo và tải tệp nhanh. Với dữ liệu lớn, có thể cần xử lý nền. \newline - \textbf{NFR-UC9.7-3 (V3 - Compatibility):} Tệp xuất tương thích các phần mềm bảng tính phổ biến. \newline - \textbf{NFR-UC9.7-4 (V3 - Security):} Quyền xuất dữ liệu nhạy cảm cần được kiểm soát. \\
\hline
\end{longtable}

\subsubsection{Use Case UC-MD09-08: Xem Báo cáo Thanh toán theo Phương thức}
\begin{longtable}{|m{4cm}|p{11cm}|}
\caption{Đặc tả Use Case UC-MD09-08: Xem Báo cáo Thanh toán theo Phương thức} \label{tab:uc_md09_08_corrected} \\
\hline
\multicolumn{2}{|c|}{\textbf{2.1. Tóm tắt (Summary)}} \\
\hline
\textbf{Mục} & \textbf{Nội dung} \\
\hline
\endhead % Header cho các trang tiếp theo
\midrule
\endfoot % Footer cho bảng
\bottomrule
\endlastfoot % Footer cho trang cuối cùng
Use Case Name & Xem Báo cáo Thanh toán theo Phương thức \\
\hline
Use Case ID & UC-MD09-08 \\
\hline
Use Case Description & Cho phép Quản lý nhà hàng (US-01) hoặc Kế toán (US-06) xem báo cáo thống kê tổng số tiền đã thu được qua từng phương thức thanh toán khác nhau (ví dụ: Tiền mặt, Ví điện tử ABC, Chuyển khoản...) trong một khoảng thời gian do người dùng lựa chọn. \\
\hline
Actor & US-01 (Quản lý nhà hàng), US-06 (Kế toán) \\
\hline
Priority & Must Have \\
\hline
Trigger & - Cần đối soát dòng tiền thu được từ các kênh thanh toán khác nhau. \newline - Muốn phân tích xu hướng sử dụng các phương thức thanh toán của khách hàng. \newline - Chuẩn bị số liệu cho việc hạch toán kế toán. \\
\hline
Pre-Condition & - Người dùng đã đăng nhập với quyền truy cập báo cáo POS/Thanh toán. \newline - Các phương thức thanh toán đã được cấu hình. \newline - Đã có dữ liệu giao dịch thanh toán từ các phiên POS đã đóng. \\
\hline
Post-Condition & - Báo cáo thống kê tổng tiền thu theo từng phương thức thanh toán được hiển thị. \newline - Người dùng có thông tin để đối soát và phân tích dòng tiền. \\
\hline
\multicolumn{2}{|c|}{\textbf{2.2. Luồng thực thi (Flow)}} \\
\hline
\textbf{Mục} & \textbf{Nội dung} \\
\hline
Basic Flow & 1. Người dùng (US-01/US-06) truy cập "Báo cáo" của POS hoặc Kế toán. \newline 2. Người dùng chọn loại báo cáo "Thanh toán theo Phương thức" (Payments by Method). \newline 3. Người dùng chọn Khoảng thời gian báo cáo. \newline 4. Người dùng nhấn "Xem báo cáo". \newline 5. Hệ thống truy vấn các bản ghi thanh toán trong kỳ, nhóm theo Phương thức thanh toán. \newline 6. Hệ thống hiển thị bảng báo cáo, mỗi dòng là một Phương thức thanh toán, với các cột: Tên Phương thức, Tổng số tiền thu, (Tùy chọn) Số lượng giao dịch. \newline 7. Có dòng tổng cộng. \newline 8. Người dùng xem xét báo cáo. \\
\hline
Alternative Flow & \textbf{6a. Lọc theo POS (nếu nhiều điểm bán).} \newline \textbf{6b. Xem dạng biểu đồ (ví dụ: biểu đồ tròn tỷ trọng). } \\
\hline
Exception Flow & Tương tự UC-MD09-03 (Lỗi truy vấn/tổng hợp, Không có dữ liệu). \\
\hline
\multicolumn{2}{|c|}{\textbf{2.3. Thông tin bổ sung (Additional Information)}} \\
\hline
\textbf{Mục} & \textbf{Nội dung} \\
\hline
Business Rule & - \textbf{BR-UC9.8-1 (V3):} Báo cáo phải tổng hợp chính xác tổng tiền thu cho mỗi phương thức. \newline - \textbf{BR-UC9.8-2 (V3):} Dữ liệu này quan trọng cho đối soát với sao kê ngân hàng/cổng thanh toán. \\
\hline
Non-Functional Requirement & - \textbf{NFR-UC9.8-1 (V3 - Accuracy):} Số liệu chính xác. \newline - \textbf{NFR-UC9.8-2 (V3 - Usability):} Báo cáo dễ hiểu, dễ so sánh. \\
\hline
\end{longtable}

\subsubsection{Use Case UC-MD09-09: Xem Báo cáo các Khoản Giảm giá/Khuyến mãi đã Áp dụng}
\begin{longtable}{|m{4cm}|p{11cm}|}
\caption{Đặc tả Use Case UC-MD09-09: Xem Báo cáo các Khoản Giảm giá/Khuyến mãi đã Áp dụng} \label{tab:uc_md09_09_corrected} \\
\hline
\multicolumn{2}{|c|}{\textbf{2.1. Tóm tắt (Summary)}} \\
\hline
\textbf{Mục} & \textbf{Nội dung} \\
\hline
\endhead % Header cho các trang tiếp theo
\midrule
\endfoot % Footer cho bảng
\bottomrule
\endlastfoot % Footer cho trang cuối cùng
Use Case Name & Xem Báo cáo các Khoản Giảm giá/Khuyến mãi đã Áp dụng \\
\hline
Use Case ID & UC-MD09-09 \\
\hline
Use Case Description & Cho phép Quản lý nhà hàng (US-01) xem báo cáo thống kê về tổng giá trị các khoản giảm giá hoặc các chương trình khuyến mãi đã được áp dụng cho các đơn hàng trong một khoảng thời gian nhất định. \\
\hline
Actor & US-01 (Quản lý nhà hàng) \\
\hline
Priority & Should Have \\
\hline
Trigger & - Cần đánh giá hiệu quả của các chương trình khuyến mãi. \newline - Cần theo dõi tổng chi phí (doanh thu bị giảm) của các hoạt động giảm giá. \\
\hline
Pre-Condition & - US-01 đã đăng nhập với quyền truy cập báo cáo Bán hàng/Khuyến mãi. \newline - Hệ thống POS có chức năng áp dụng giảm giá/khuyến mãi. \newline - Đã có dữ liệu đơn hàng áp dụng giảm giá/khuyến mãi. \\
\hline
Post-Condition & - Báo cáo thống kê về các khoản giảm giá/khuyến mãi được hiển thị. \newline - Quản lý có thông tin để phân tích hiệu quả và chi phí. \\
\hline
\multicolumn{2}{|c|}{\textbf{2.2. Luồng thực thi (Flow)}} \\
\hline
\textbf{Mục} & \textbf{Nội dung} \\
\hline
Basic Flow & 1. US-01 truy cập "Báo cáo" của POS hoặc Sales. \newline 2. US-01 chọn loại báo cáo "Giảm giá đã Áp dụng" (Applied Discounts / Promotions Report). \newline 3. US-01 chọn Khoảng thời gian báo cáo. \newline 4. US-01 nhấn "Xem báo cáo". \newline 5. Hệ thống truy vấn dữ liệu đơn hàng có áp dụng giảm giá/khuyến mãi trong kỳ. \newline 6. Hệ thống hiển thị báo cáo, có thể gồm: \newline    - \textbf{Tổng hợp chung:} Tổng giá trị giảm giá. \newline    - \textbf{Chi tiết theo Chương trình (nếu có):} Tên chương trình, Số lần áp dụng, Tổng giá trị giảm. \newline    - \textbf{Chi tiết theo Sản phẩm (nếu giảm giá trên sản phẩm):} Tên sản phẩm, Số lượng được giảm, Tổng giá trị giảm. \newline 7. US-01 xem xét báo cáo. \\
\hline
Alternative Flow & \textbf{6a. Lọc theo loại giảm giá/chương trình cụ thể.} \\
\hline
Exception Flow & Tương tự UC-MD09-03 (Lỗi truy vấn/tổng hợp, Không có dữ liệu). \\
\hline
\multicolumn{2}{|c|}{\textbf{2.3. Thông tin bổ sung (Additional Information)}} \\
\hline
\textbf{Mục} & \textbf{Nội dung} \\
\hline
Business Rule & - \textbf{BR-UC9.9-1 (V3):} Hệ thống phải ghi nhận chính xác mọi khoản giảm giá/khuyến mãi. \newline - \textbf{BR-UC9.9-2 (V3):} Báo cáo cần phân biệt rõ các loại giảm giá (nếu có). \\
\hline
Non-Functional Requirement & - \textbf{NFR-UC9.9-1 (V3 - Accuracy):} Số liệu giảm giá chính xác. \newline - \textbf{NFR-UC9.9-2 (V3 - Usability):} Báo cáo dễ hiểu, dễ đánh giá. \\
\hline
\end{longtable}

\subsubsection{Use Case UC-MD09-10: Xem Báo cáo các Đơn hàng/Món ăn đã Hủy (Void)}
\begin{longtable}{|m{4cm}|p{11cm}|}
\caption{Đặc tả Use Case UC-MD09-10: Xem Báo cáo các Đơn hàng/Món ăn đã Hủy (Void)} \label{tab:uc_md09_10_corrected} \\
\hline
\multicolumn{2}{|c|}{\textbf{2.1. Tóm tắt (Summary)}} \\
\hline
\textbf{Mục} & \textbf{Nội dung} \\
\hline
\endhead % Header cho các trang tiếp theo
\midrule
\endfoot % Footer cho bảng
\bottomrule
\endlastfoot % Footer cho trang cuối cùng
Use Case Name & Xem Báo cáo các Đơn hàng/Món ăn đã Hủy (Void) \\
\hline
Use Case ID & UC-MD09-10 \\
\hline
Use Case Description & Cho phép Quản lý nhà hàng (US-01) xem báo cáo thống kê về các món ăn hoặc toàn bộ đơn hàng đã bị hủy (voided) trên POS trong một khoảng thời gian. Báo cáo có thể bao gồm sản phẩm bị hủy, số lượng, giá trị, nhân viên thực hiện, và lý do hủy (nếu có). \\
\hline
Actor & US-01 (Quản lý nhà hàng) \\
\hline
Priority & Should Have \\
\hline
Trigger & - Cần kiểm tra, phân tích tình hình hủy món/đơn để kiểm soát thất thoát. \newline - Cần theo dõi hoạt động hủy của nhân viên. \\
\hline
Pre-Condition & - US-01 đã đăng nhập với quyền truy cập báo cáo POS/kiểm soát. \newline - Hệ thống POS có chức năng hủy món/đơn (UC-MD05-21, UC-MD05-22) và ghi nhận hành động hủy. \\
\hline
Post-Condition & - Báo cáo thống kê về các trường hợp hủy món/đơn được hiển thị. \newline - Quản lý có thông tin để phân tích và đưa ra biện pháp. \\
\hline
\multicolumn{2}{|c|}{\textbf{2.2. Luồng thực thi (Flow)}} \\
\hline
\textbf{Mục} & \textbf{Nội dung} \\
\hline
Basic Flow & 1. US-01 truy cập "Báo cáo" của POS. \newline 2. US-01 chọn loại báo cáo "Món ăn/Đơn hàng đã Hủy" (Voided Items/Orders Report). \newline 3. US-01 chọn Khoảng thời gian báo cáo. \newline 4. US-01 nhấn "Xem báo cáo". \newline 5. Hệ thống truy vấn các dòng món ăn/đơn hàng đã được đánh dấu hủy trong kỳ. \newline 6. Hệ thống hiển thị bảng báo cáo, gồm: Ngày giờ hủy, Tên Sản phẩm/Mã Đơn hàng, Số lượng, Giá trị, Nhân viên hủy, Lý do hủy (nếu có). \newline 7. US-01 xem xét báo cáo. \\
\hline
Alternative Flow & \textbf{6a. Lọc theo Nhân viên hoặc Lý do hủy.} \newline \textbf{6b. Tổng hợp theo Lý do hủy hoặc Nhân viên.} \\
\hline
Exception Flow & Tương tự UC-MD09-03 (Lỗi truy vấn/tổng hợp, Không có dữ liệu). \\
\hline
\multicolumn{2}{|c|}{\textbf{2.3. Thông tin bổ sung (Additional Information)}} \\
\hline
\textbf{Mục} & \textbf{Nội dung} \\
\hline
Business Rule & - \textbf{BR-UC9.10-1 (V3):} Mọi hành động hủy món/đơn phải được ghi nhận chi tiết. \newline - \textbf{BR-UC9.10-2 (V3):} Yêu cầu nhập lý do khi hủy là thực hành tốt. \\
\hline
Non-Functional Requirement & - \textbf{NFR-UC9.10-1 (V3 - Auditability):} Dữ liệu hủy phải đầy đủ, không thể sửa đổi. \newline - \textbf{NFR-UC9.10-2 (V3 - Usability):} Báo cáo dễ đọc, dễ lọc. \\
\hline
\end{longtable}

\subsubsection{Use Case UC-MD09-11: (Tùy chọn) Thiết lập và Lên lịch Gửi Báo cáo Tự động}
\begin{longtable}{|m{4cm}|p{11cm}|}
\caption{Đặc tả Use Case UC-MD09-11: (Tùy chọn) Thiết lập và Lên lịch Gửi Báo cáo Tự động} \label{tab:uc_md09_11_corrected} \\
\hline
\multicolumn{2}{|c|}{\textbf{2.1. Tóm tắt (Summary)}} \\
\hline
\textbf{Mục} & \textbf{Nội dung} \\
\hline
\endhead % Header cho các trang tiếp theo
\midrule
\endfoot % Footer cho bảng
\bottomrule
\endlastfoot % Footer cho trang cuối cùng
Use Case Name & (Tùy chọn) Thiết lập và Lên lịch Gửi Báo cáo Tự động \\
\hline
Use Case ID & UC-MD09-11 \\
\hline
Use Case Description & Cho phép Quản lý nhà hàng (US-01) hoặc Kế toán (US-06) cấu hình để hệ thống tự động tạo và gửi một số loại báo cáo nhất định (ví dụ: báo cáo doanh thu cuối ngày) đến một hoặc nhiều địa chỉ email theo một lịch trình định kỳ. \\
\hline
Actor & US-01 (Quản lý nhà hàng), US-06 (Kế toán) \\
\hline
Priority & Nice to Have \\
\hline
Trigger & Muốn nhận báo cáo quan trọng tự động, định kỳ mà không cần truy cập hệ thống thủ công. \\
\hline
Pre-Condition & - Người dùng đã đăng nhập với quyền quản trị báo cáo/cấu hình hệ thống. \newline - Hệ thống có khả năng tạo các báo cáo mục tiêu. \newline - Hệ thống gửi email (Outgoing Email Server) đã được cấu hình (UC-MD10-04). \\
\hline
Post-Condition & - Lịch trình gửi báo cáo tự động được thiết lập và kích hoạt. \newline - Hệ thống tự động tạo và gửi báo cáo đến email đã chỉ định theo lịch. \\
\hline
\multicolumn{2}{|c|}{\textbf{2.2. Luồng thực thi (Flow)}} \\
\hline
\textbf{Mục} & \textbf{Nội dung} \\
\hline
Basic Flow (Thiết lập lịch gửi mới) & 1. Người dùng (US-01/US-06) truy cập khu vực cấu hình "Báo cáo Tự động" / "Scheduled Actions for Reports". \newline 2. Người dùng chọn "Tạo mới Lịch gửi Báo cáo". \newline 3. Hệ thống hiển thị form cấu hình: \newline    a. Tên Lịch gửi. \newline    b. Loại Báo cáo (chọn từ danh sách báo cáo hỗ trợ). \newline    c. Tần suất Gửi (Hàng ngày, Hàng tuần...). \newline    d. Thời điểm Gửi (Ngày/giờ cụ thể). \newline    e. Người nhận (một hoặc nhiều địa chỉ email). \newline    f. Định dạng Tệp đính kèm (PDF, Excel). \newline    g. (Tùy chọn) Nội dung Email. \newline    h. (Tùy chọn) Tham số Báo cáo (ví dụ: khoảng thời gian là "Ngày hôm qua"). \newline 4. Người dùng điền thông tin cấu hình. \newline 5. Người dùng chọn "Lưu và Kích hoạt". \newline 6. Hệ thống lưu và kích hoạt lịch gửi. \newline 7. Hệ thống báo thành công. \\
\hline
Alternative Flow & \textbf{1a. Sửa/Vô hiệu hóa Lịch gửi đã có.} \newline \textbf{3i. Gửi thử (Test Send):} Gửi ngay một bản báo cáo mẫu để kiểm tra. \\
\hline
Exception Flow & \textbf{6a. Lỗi lưu/cấu hình lịch gửi.} \newline \textbf{Khi hệ thống tự động chạy (Background):} Lỗi tạo báo cáo, Lỗi gửi email (cần log và thông báo admin). \\
\hline
\multicolumn{2}{|c|}{\textbf{2.3. Thông tin bổ sung (Additional Information)}} \\
\hline
\textbf{Mục} & \textbf{Nội dung} \\
\hline
Business Rule & - \textbf{BR-UC9.11-1:} Hệ thống phải có tác vụ nền để tự động kiểm tra và thực hiện gửi báo cáo. \newline - \textbf{BR-UC9.11-2:} Người dùng phải chọn được từ danh sách các báo cáo hệ thống hỗ trợ gửi tự động. \newline - \textbf{BR-UC9.11-3:} Tham số báo cáo (ví dụ: khoảng thời gian tương đối) phải được diễn giải đúng. \\
\hline
Non-Functional Requirement & - \textbf{NFR-UC9.11-1 (Usability):} Giao diện thiết lập lịch gửi phải dễ hiểu, linh hoạt. \newline - \textbf{NFR-UC9.11-2 (Reliability):} Cơ chế gửi tự động phải ổn định, đúng lịch. Cần xử lý lỗi và thông báo. \newline - \textbf{NFR-UC9.11-3 (Performance):} Việc tạo và gửi báo cáo tự động không làm ảnh hưởng nhiều đến hiệu năng hệ thống. \\
\hline
\end{longtable}