\subsection{Module MD-01: Quản lý Lịch làm việc (Scheduling)}

\subsubsection{Use Case UC-MD01-01: Tạo mới Vai trò Công việc}

\begin{longtable}{|m{4cm}|p{11cm}|}
\caption{Đặc tả Use Case UC-MD01-01: Tạo mới Vai trò Công việc} \label{tab:uc_md01_01_revised} \\
\hline
\multicolumn{2}{|c|}{\textbf{2.1. Tóm tắt (Summary)}} \\
\hline
\textbf{Mục} & \textbf{Nội dung} \\
\hline
\endhead % Header cho các trang tiếp theo
\hline
\endfoot % Footer cho bảng
\hline
\endlastfoot % Footer cho trang cuối cùng
Use Case Name & Tạo mới Vai trò Công việc \\
\hline
Use Case ID & UC-MD01-01 \\
\hline
Use Case Description & Cho phép Quản lý nhà hàng (US-01) định nghĩa một vai trò công việc mới (ví dụ: "Đầu bếp chính", "Phục vụ bàn", "Lễ tân") trong hệ thống. Vai trò này sẽ được sử dụng để xác định yêu cầu nhân sự khi tạo ca làm việc và gán cho nhân viên. \\
\hline
Actor & US-01 (Quản lý nhà hàng) \\
\hline
Priority & Must Have \\
\hline
Trigger & Nhà hàng có một vị trí công việc mới cần được quản lý lịch trình hoặc cần phân loại rõ hơn các vai trò hiện có. \\
\hline
Pre-Condition & - US-01 đã đăng nhập vào hệ thống với quyền quản trị module Lập lịch (Planning) hoặc cấu hình nhân sự liên quan. \\
\hline
Post-Condition & - Một bản ghi vai trò công việc mới được tạo và lưu trong hệ thống. \newline - Vai trò mới này sẵn sàng để được chọn khi tạo khung ca làm việc (UC-MD01-05) hoặc gán cho hồ sơ nhân viên. \newline - Hệ thống ghi nhận hoạt động. \\
\hline
\multicolumn{2}{|c|}{\textbf{2.2. Luồng thực thi (Flow)}} \\
\hline
\textbf{Mục} & \textbf{Nội dung} \\
\hline
Basic Flow & 1. US-01 truy cập vào khu vực quản lý/cấu hình Vai trò Công việc (thường trong module Planning > Configuration > Roles). \newline 2. Hệ thống hiển thị danh sách các vai trò công việc hiện có (UC-MD01-02). \newline 3. US-01 chọn hành động "Tạo mới" (Create). \newline 4. Hệ thống hiển thị form để nhập thông tin vai trò mới. \newline 5. US-01 nhập Tên Vai trò (Role Name) (bắt buộc, ví dụ: "Bếp Chính", "Phục Vụ Ca Sáng"). \newline 6. (Tùy chọn) US-01 chọn màu sắc đại diện cho vai trò này (Color Index) để dễ phân biệt trên lịch biểu Gantt. \newline 7. (Tùy chọn) US-01 nhập Mô tả chi tiết hơn về vai trò (Description). \newline 8. US-01 chọn lệnh "Lưu" (Save). \newline 9. Hệ thống kiểm tra tính hợp lệ của dữ liệu (ví dụ: Tên vai trò không được để trống, có thể kiểm tra trùng tên - BR-UC1.1-1). \newline 10. Hệ thống lưu bản ghi vai trò mới vào cơ sở dữ liệu. \newline 11. Hệ thống cập nhật danh sách vai trò, hiển thị vai trò mới được tạo. \newline 12. Hệ thống hiển thị thông báo tạo vai trò thành công. \newline 13. Hệ thống ghi nhận hoạt động vào Activity Log. \\
\hline
Alternative Flow & \textbf{8a. Lưu và Tạo mới (Save \& New):} \newline    1. Thay vì chọn "Lưu", US-01 chọn "Lưu và Tạo mới". \newline    2. Hệ thống thực hiện các bước 9, 10, 13. \newline    3. Hệ thống hiển thị lại một form vai trò trống (quay lại bước 4) để US-01 tiếp tục nhập vai trò khác. \\
\hline
Exception Flow & \textbf{9a. Dữ liệu không hợp lệ:} \newline    1. Hệ thống phát hiện Tên Vai trò bị bỏ trống hoặc (nếu có kiểm tra) trùng với tên vai trò khác đã tồn tại. \newline    2. Hệ thống hiển thị thông báo lỗi cụ thể. \newline    3. Hệ thống giữ nguyên form và cho phép US-01 chỉnh sửa. Use Case quay lại bước 5. \newline \textbf{10a. Lỗi hệ thống khi lưu:} \newline    1. Hệ thống gặp sự cố kỹ thuật trong quá trình lưu dữ liệu. \newline    2. Hệ thống hiển thị thông báo lỗi chung. \newline    3. Use Case kết thúc trong trạng thái lỗi. \\
\hline
\multicolumn{2}{|c|}{\textbf{2.3. Thông tin bổ sung (Additional Information)}} \\
\hline
\textbf{Mục} & \textbf{Nội dung} \\
\hline
Business Rule & - \textbf{BR-UC1.1-1:} Tên Vai trò công việc là thông tin bắt buộc và nên là duy nhất để dễ phân biệt. \newline - \textbf{BR-UC1.1-2:} Màu sắc đại diện giúp trực quan hóa lịch trình theo vai trò trên biểu đồ Gantt. \\
\hline
Non-Functional Requirement & - \textbf{NFR-UC1.1-1 (Usability):} Giao diện tạo vai trò phải đơn giản, các trường thông tin rõ ràng. \newline - \textbf{NFR-UC1.1-2 (Performance):} Thời gian lưu vai trò mới phải nhanh chóng (dưới 2 giây). \\
\hline
\end{longtable}

\subsubsection{Use Case UC-MD01-02: Xem Danh sách Vai trò Công việc}

\begin{longtable}{|m{4cm}|p{11cm}|}
\caption{Đặc tả Use Case UC-MD01-02: Xem Danh sách Vai trò Công việc} \label{tab:uc_md01_02_revised} \\
\hline
\multicolumn{2}{|c|}{\textbf{2.1. Tóm tắt (Summary)}} \\
\hline
\textbf{Mục} & \textbf{Nội dung} \\
\hline
\endhead % Header cho các trang tiếp theo
\hline
\endfoot % Footer cho bảng
\hline
\endlastfoot % Footer cho trang cuối cùng
Use Case Name & Xem Danh sách Vai trò Công việc \\
\hline
Use Case ID & UC-MD01-02 \\
\hline
Use Case Description & Cho phép Quản lý nhà hàng (US-01) xem danh sách tất cả các vai trò công việc đã được định nghĩa trong hệ thống, bao gồm tên và các thông tin cơ bản khác. \\
\hline
Actor & US-01 (Quản lý nhà hàng) \\
\hline
Priority & Must Have \\
\hline
Trigger & Quản lý nhà hàng cần kiểm tra lại các vai trò hiện có, tìm kiếm một vai trò cụ thể, hoặc chuẩn bị cho việc tạo/sửa/xóa vai trò. \\
\hline
Pre-Condition & - US-01 đã đăng nhập vào hệ thống với quyền quản trị module Lập lịch (Planning) hoặc cấu hình nhân sự liên quan. \\
\hline
Post-Condition & - Danh sách các vai trò công việc được hiển thị cho US-01. \newline - US-01 có thể xem các thông tin cơ bản của từng vai trò. \\
\hline
\multicolumn{2}{|c|}{\textbf{2.2. Luồng thực thi (Flow)}} \\
\hline
\textbf{Mục} & \textbf{Nội dung} \\
\hline
Basic Flow & 1. US-01 truy cập vào khu vực quản lý/cấu hình Vai trò Công việc (thường trong module Planning > Configuration > Roles). \newline 2. Hệ thống truy vấn cơ sở dữ liệu và hiển thị danh sách các vai trò công việc đã được tạo. \newline 3. Với mỗi vai trò trong danh sách, hệ thống hiển thị các thông tin cơ bản như: Tên Vai trò, Màu sắc đại diện (nếu có). \newline 4. US-01 xem xét danh sách. \\
\hline
Alternative Flow & \textbf{4a. Tìm kiếm/Lọc vai trò:} \newline    1. Giao diện danh sách cung cấp ô tìm kiếm. US-01 nhập tên vai trò cần tìm. \newline    2. Hệ thống lọc và hiển thị các vai trò khớp với từ khóa tìm kiếm. \newline \textbf{4b. Sắp xếp danh sách:} \newline    1. US-01 nhấp vào tiêu đề cột (ví dụ: Tên Vai trò) để sắp xếp danh sách theo thứ tự tăng dần/giảm dần. \\
\hline
Exception Flow & \textbf{2a. Lỗi tải danh sách:} \newline    1. Hệ thống gặp lỗi khi truy vấn dữ liệu vai trò. \newline    2. Hệ thống hiển thị thông báo lỗi. \newline \textbf{2b. Không có vai trò nào được định nghĩa:} \newline    1. Nếu chưa có vai trò nào được tạo. \newline    2. Hệ thống hiển thị danh sách trống hoặc thông báo "Chưa có vai trò nào được tạo." \\
\hline
\multicolumn{2}{|c|}{\textbf{2.3. Thông tin bổ sung (Additional Information)}} \\
\hline
\textbf{Mục} & \textbf{Nội dung} \\
\hline
Business Rule & - \textbf{BR-UC1.2-1:} Danh sách phải hiển thị chính xác tất cả các vai trò công việc đang hoạt động trong hệ thống. \\
\hline
Non-Functional Requirement & - \textbf{NFR-UC1.2-1 (Usability):} Giao diện danh sách phải rõ ràng, dễ đọc. Chức năng tìm kiếm/lọc (nếu có) phải hiệu quả. \newline - \textbf{NFR-UC1.2-2 (Performance):} Thời gian tải danh sách vai trò (ví dụ: < 50 vai trò) phải nhanh chóng. \\
\hline
\end{longtable}

\subsubsection{Use Case UC-MD01-03: Sửa Thông tin Vai trò Công việc}

\begin{longtable}{|m{4cm}|p{11cm}|}
\caption{Đặc tả Use Case UC-MD01-03: Sửa Thông tin Vai trò Công việc} \label{tab:uc_md01_03_revised} \\
\hline
\multicolumn{2}{|c|}{\textbf{2.1. Tóm tắt (Summary)}} \\
\hline
\textbf{Mục} & \textbf{Nội dung} \\
\hline
\endhead % Header cho các trang tiếp theo
\hline
\endfoot % Footer cho bảng
\hline
\endlastfoot % Footer cho trang cuối cùng
Use Case Name & Sửa Thông tin Vai trò Công việc \\
\hline
Use Case ID & UC-MD01-03 \\
\hline
Use Case Description & Cho phép Quản lý nhà hàng (US-01) chỉnh sửa các thông tin của một vai trò công việc đã tồn tại trong hệ thống, ví dụ: thay đổi tên, màu sắc đại diện, hoặc mô tả. \\
\hline
Actor & US-01 (Quản lý nhà hàng) \\
\hline
Priority & Must Have \\
\hline
Trigger & - Cần cập nhật tên của một vai trò cho phù hợp hơn. \newline - Muốn thay đổi màu sắc đại diện của vai trò trên lịch biểu. \newline - Cần bổ sung/chỉnh sửa mô tả cho vai trò. \\
\hline
Pre-Condition & - US-01 đã đăng nhập vào hệ thống với quyền quản trị module Lập lịch (Planning) hoặc cấu hình nhân sự liên quan. \newline - Vai trò công việc cần sửa đã tồn tại trong hệ thống. \\
\hline
Post-Condition & - Thông tin của vai trò công việc được chọn đã được cập nhật trong cơ sở dữ liệu. \newline - Các thay đổi (ví dụ: tên, màu sắc) sẽ được phản ánh trên các giao diện liên quan (ví dụ: lịch biểu Gantt, form gán nhân viên). \newline - Hệ thống ghi nhận hoạt động. \\
\hline
\multicolumn{2}{|c|}{\textbf{2.2. Luồng thực thi (Flow)}} \\
\hline
\textbf{Mục} & \textbf{Nội dung} \\
\hline
Basic Flow & 1. US-01 truy cập danh sách Vai trò Công việc (UC-MD01-02). \newline 2. US-01 tìm và chọn (nhấp vào) vai trò công việc muốn sửa. \newline 3. Hệ thống hiển thị form chi tiết của vai trò đó ở chế độ xem. \newline 4. US-01 chọn hành động "Sửa" (Edit). \newline 5. Hệ thống cho phép chỉnh sửa các trường thông tin trên form (Tên Vai trò, Màu sắc, Mô tả...). \newline 6. US-01 thực hiện các thay đổi mong muốn. \newline 7. US-01 chọn lệnh "Lưu" (Save). \newline 8. Hệ thống kiểm tra tính hợp lệ của dữ liệu đã thay đổi (ví dụ: Tên Vai trò không được để trống, có thể kiểm tra trùng tên - BR-UC1.3-1). \newline 9. Hệ thống cập nhật thông tin mới cho bản ghi vai trò trong cơ sở dữ liệu. \newline 10. Hệ thống chuyển form về chế độ xem với thông tin đã cập nhật. \newline 11. Hệ thống hiển thị thông báo cập nhật thành công. \newline 12. Hệ thống ghi nhận hoạt động vào Activity Log. \\
\hline
Alternative Flow & Không có luồng thay thế đáng kể. \\
\hline
Exception Flow & \textbf{8a. Dữ liệu không hợp lệ:} \newline    1. Hệ thống phát hiện Tên Vai trò bị bỏ trống hoặc (nếu có kiểm tra) trùng với tên vai trò khác. \newline    2. Hệ thống hiển thị thông báo lỗi. \newline    3. Hệ thống giữ nguyên form chỉnh sửa để US-01 sửa lại. Use Case quay lại bước 6. \newline \textbf{9a. Lỗi hệ thống khi cập nhật:} \newline    1. Hệ thống gặp sự cố kỹ thuật khi lưu thay đổi. \newline    2. Hệ thống hiển thị thông báo lỗi chung. \newline    3. Use Case kết thúc không thành công, thay đổi có thể không được lưu. \\
\hline
\multicolumn{2}{|c|}{\textbf{2.3. Thông tin bổ sung (Additional Information)}} \\
\hline
\textbf{Mục} & \textbf{Nội dung} \\
\hline
Business Rule & - \textbf{BR-UC1.3-1:} Tên Vai trò công việc sau khi sửa không được để trống và nên là duy nhất. \newline - \textbf{BR-UC1.3-2:} Việc thay đổi tên hoặc màu sắc của vai trò sẽ ảnh hưởng đến cách nó hiển thị trên các lịch trình đã có sử dụng vai trò đó. \\
\hline
Non-Functional Requirement & - \textbf{NFR-UC1.3-1 (Usability):} Giao diện sửa vai trò phải dễ sử dụng. \newline - \textbf{NFR-UC1.3-2 (Performance):} Thời gian lưu thay đổi phải nhanh chóng. \\
\hline
\end{longtable}

\subsubsection{Use Case UC-MD01-04: Xóa Vai trò Công việc}

\begin{longtable}{|m{4cm}|p{11cm}|}
\caption{Đặc tả Use Case UC-MD01-04: Xóa Vai trò Công việc} \label{tab:uc_md01_04_revised} \\
\hline
\multicolumn{2}{|c|}{\textbf{2.1. Tóm tắt (Summary)}} \\
\hline
\textbf{Mục} & \textbf{Nội dung} \\
\hline
\endhead % Header cho các trang tiếp theo
\hline
\endfoot % Footer cho bảng
\hline
\endlastfoot % Footer cho trang cuối cùng
Use Case Name & Xóa Vai trò Công việc \\
\hline
Use Case ID & UC-MD01-04 \\
\hline
Use Case Description & Cho phép Quản lý nhà hàng (US-01) xóa bỏ một vai trò công việc không còn được sử dụng khỏi hệ thống, với điều kiện vai trò đó không đang được gán cho bất kỳ nhân viên nào hoặc sử dụng trong bất kỳ ca làm việc nào. \\
\hline
Actor & US-01 (Quản lý nhà hàng) \\
\hline
Priority & Should Have \\
\hline
Trigger & Một vai trò công việc không còn phù hợp hoặc không còn cần thiết cho hoạt động của nhà hàng. \\
\hline
Pre-Condition & - US-01 đã đăng nhập vào hệ thống với quyền quản trị module Lập lịch (Planning) hoặc cấu hình nhân sự liên quan. \newline - Vai trò công việc cần xóa đã tồn tại trong hệ thống. \\
\hline
Post-Condition & - Nếu xóa thành công, bản ghi vai trò công việc bị xóa khỏi cơ sở dữ liệu. \newline - Vai trò đó không còn xuất hiện trong danh sách và không thể sử dụng để tạo ca hoặc gán cho nhân viên. \newline - Nếu không thể xóa (do đang được sử dụng), vai trò vẫn tồn tại và hệ thống thông báo lỗi. \newline - Hệ thống ghi nhận hoạt động (thành công hoặc thất bại). \\
\hline
\multicolumn{2}{|c|}{\textbf{2.2. Luồng thực thi (Flow)}} \\
\hline
\textbf{Mục} & \textbf{Nội dung} \\
\hline
Basic Flow & 1. US-01 truy cập danh sách Vai trò Công việc (UC-MD01-02). \newline 2. US-01 chọn vai trò công việc muốn xóa. \newline 3. US-01 chọn hành động "Xóa" (Delete) (thường từ menu Hành động trên form chi tiết hoặc từ danh sách). \newline 4. Hệ thống hiển thị hộp thoại yêu cầu xác nhận việc xóa. \newline 5. US-01 xác nhận muốn xóa. \newline 6. Hệ thống kiểm tra xem vai trò này có đang được tham chiếu ở bất kỳ đâu không (ví dụ: gán cho nhân viên, sử dụng trong ca làm việc đã tạo - BR-UC1.4-1). \newline 7. \textbf{Nếu vai trò KHÔNG đang được sử dụng:} \newline    a. Hệ thống xóa bản ghi vai trò khỏi cơ sở dữ liệu. \newline    b. Hệ thống cập nhật lại danh sách vai trò (vai trò vừa xóa biến mất). \newline    c. Hệ thống hiển thị thông báo xóa thành công. \newline 8. \textbf{Nếu vai trò CÓ đang được sử dụng:} \newline    a. Hệ thống KHÔNG xóa vai trò. \newline    b. Hệ thống hiển thị thông báo lỗi, giải thích rằng không thể xóa vì vai trò đang được sử dụng và có thể chỉ rõ nơi đang sử dụng (ví dụ: "Không thể xóa vai trò 'Phục vụ' vì đang được gán cho 5 nhân viên và sử dụng trong 10 ca làm việc."). \newline 9. Hệ thống ghi nhận hoạt động. \\
\hline
Alternative Flow & \textbf{2a. Xóa từ danh sách (List View):} \newline    1. US-01 chọn một hoặc nhiều vai trò từ danh sách (tick vào ô vuông). \newline    2. US-01 chọn hành động "Xóa" từ menu chung của danh sách. \newline    3. Use Case tiếp tục từ bước 4. \\
\hline
Exception Flow & \textbf{6a. Lỗi hệ thống khi kiểm tra tham chiếu hoặc khi xóa:} \newline    1. Hệ thống gặp sự cố kỹ thuật trong quá trình kiểm tra hoặc xóa. \newline    2. Hệ thống hiển thị thông báo lỗi chung. \newline    3. Hành động xóa có thể không thành công. \\
\hline
\multicolumn{2}{|c|}{\textbf{2.3. Thông tin bổ sung (Additional Information)}} \\
\hline
\textbf{Mục} & \textbf{Nội dung} \\
\hline
Business Rule & - \textbf{BR-UC1.4-1:} Một vai trò công việc không thể bị xóa nếu nó đang được liên kết với bất kỳ bản ghi nào khác trong hệ thống (ví dụ: hồ sơ nhân viên, ca làm việc đã tạo, mẫu lịch trình). Người dùng phải gỡ bỏ các liên kết này trước khi có thể xóa vai trò. \newline - \textbf{BR-UC1.4-2:} Việc xóa vai trò là hành động vĩnh viễn và không thể hoàn tác (trừ khi có cơ chế sao lưu/phục hồi). \\
\hline
Non-Functional Requirement & - \textbf{NFR-UC1.4-1 (Data Integrity):} Ràng buộc không cho xóa vai trò đang sử dụng là rất quan trọng để đảm bảo tính toàn vẹn dữ liệu của hệ thống. \newline - \textbf{NFR-UC1.4-2 (Usability):} Thông báo lỗi khi không thể xóa cần phải rõ ràng, giải thích lý do. \\
\hline
\end{longtable}

\subsubsection{Use Case UC-MD01-05: Tạo Khung Ca làm việc}

\begin{longtable}{|m{4cm}|p{11cm}|}
\caption{Đặc tả Use Case UC-MD01-05: Tạo Khung Ca làm việc} \label{tab:uc_md01_05_revised} \\
\hline
\multicolumn{2}{|c|}{\textbf{2.1. Tóm tắt (Summary)}} \\
\hline
\textbf{Mục} & \textbf{Nội dung} \\
\hline
\endhead % Header cho các trang tiếp theo
\hline
\endfoot % Footer cho bảng
\hline
\endlastfoot % Footer cho trang cuối cùng
Use Case Name & Tạo Khung Ca làm việc \\
\hline
Use Case ID & UC-MD01-05 \\
\hline
Use Case Description & Cho phép Quản lý nhà hàng (US-01) định nghĩa các khung thời gian làm việc cụ thể (ca) với các yêu cầu về vai trò công việc và số lượng nhân viên cần thiết cho mỗi vai trò đó trong ca. Các khung ca này ban đầu chưa được gán nhân viên cụ thể. \\
\hline
Actor & US-01 (Quản lý nhà hàng) \\
\hline
Priority & Must Have \\
\hline
Trigger & Quản lý nhà hàng cần lập kế hoạch nhân sự cho một ngày hoặc một khoảng thời gian, xác định nhu cầu về số lượng và loại nhân viên cho từng ca. \\
\hline
Pre-Condition & - US-01 đã đăng nhập vào hệ thống với quyền quản lý lịch trình. \newline - Các Vai trò Công việc (ví dụ: Phục vụ, Bếp) đã được định nghĩa trong hệ thống (UC-MD01-01). \\
\hline
Post-Condition & - Một hoặc nhiều bản ghi khung ca làm việc mới được tạo và lưu trong hệ thống với trạng thái "Nháp" (Draft). \newline - Mỗi khung ca hiển thị trên giao diện lịch trình (ví dụ: Gantt chart) dưới dạng các slot trống cần được lấp đầy bởi nhân viên. \\
\hline
\multicolumn{2}{|c|}{\textbf{2.2. Luồng thực thi (Flow)}} \\
\hline
\textbf{Mục} & \textbf{Nội dung} \\
\hline
Basic Flow & 1. US-01 truy cập chức năng quản lý lịch làm việc. \newline 2. US-01 chọn hành động để tạo ca làm việc mới (ví dụ: nhấn nút "New" trên lịch hoặc click trực tiếp vào một ô thời gian/nhân viên trên Gantt chart). \newline 3. Hệ thống hiển thị form/dialog để nhập thông tin khung ca làm việc. \newline 4. US-01 nhập/chọn các thông tin chi tiết: \newline    - Ngày bắt đầu và Giờ bắt đầu của ca. \newline    - Ngày kết thúc và Giờ kết thúc của ca. \newline    - (Tùy chọn) Lặp lại (ví dụ: hàng ngày, hàng tuần trong một khoảng thời gian). \newline 5. Trong phần yêu cầu nhân sự của ca: \newline    a. US-01 chọn một Vai trò Công việc từ danh sách (đã tạo ở UC-MD01-01). \newline    b. US-01 nhập Số lượng nhân viên cần cho vai trò đó trong ca này. \newline    c. US-01 có thể thêm nhiều dòng yêu cầu vai trò khác nhau cho cùng một khung ca. \newline 6. (Tùy chọn) US-01 nhập Ghi chú cho khung ca. \newline 7. US-01 chọn lệnh "Lưu" hoặc "Tạo". \newline 8. Hệ thống kiểm tra tính hợp lệ của dữ liệu (ví dụ: thời gian kết thúc sau thời gian bắt đầu, số lượng nhân viên là số dương - BR-UC1.5-1, BR-UC1.5-2). \newline 9. Hệ thống lưu thông tin khung ca làm việc vào cơ sở dữ liệu với trạng thái "Nháp". \newline 10. Hệ thống cập nhật giao diện lịch trình để hiển thị các slot trống của khung ca mới. \\
\hline
Alternative Flow & \textbf{2a. Tạo nhanh trên Gantt chart:} \newline    1. US-01 kéo chuột trên một dòng thời gian của một ngày trên Gantt chart để định nghĩa nhanh giờ bắt đầu và kết thúc cho một ca. \newline    2. Hệ thống mở form tạo ca với thời gian đã được điền sẵn. US-01 tiếp tục nhập các thông tin còn lại (yêu cầu vai trò, số lượng). \\
\hline
Exception Flow & \textbf{8a. Dữ liệu không hợp lệ:} \newline    1. Hệ thống phát hiện lỗi (ví dụ: giờ kết thúc trước giờ bắt đầu, số lượng nhân viên không hợp lệ). \newline    2. Hệ thống hiển thị thông báo lỗi, chỉ rõ trường sai. \newline    3. Hệ thống giữ nguyên form để US-01 sửa. Use Case quay lại bước 4. \newline \textbf{9a. Lỗi hệ thống khi lưu:} \newline    1. Hệ thống gặp lỗi khi lưu khung ca. \newline    2. Hệ thống báo lỗi chung. \\
\hline
\multicolumn{2}{|c|}{\textbf{2.3. Thông tin bổ sung (Additional Information)}} \\
\hline
\textbf{Mục} & \textbf{Nội dung} \\
\hline
Business Rule & - \textbf{BR-UC1.5-1:} Giờ kết thúc của ca làm việc phải lớn hơn giờ bắt đầu. \newline - \textbf{BR-UC1.5-2:} Số lượng nhân viên yêu cầu cho mỗi vai trò phải là một số nguyên dương. \newline - \textbf{BR-UC1.5-3:} Khung ca mới tạo mặc định có trạng thái là "Nháp". \newline - \textbf{BR-UC1.5-4:} Chỉ có thể chọn các Vai trò Công việc đã được định nghĩa trong hệ thống. \\
\hline
Non-Functional Requirement & - \textbf{NFR-UC1.5-1 (Usability):} Giao diện tạo khung ca phải trực quan, dễ dàng chọn ngày giờ, vai trò và số lượng. Việc tạo nhanh trên Gantt chart là một lợi thế. \newline - \textbf{NFR-UC1.5-2 (Performance):} Thời gian lưu khung ca và cập nhật lịch phải nhanh. \\
\hline
\end{longtable}

\subsubsection{Use Case UC-MD01-06: Gán Nhân viên vào Ca làm việc}

\begin{longtable}{|m{4cm}|p{11cm}|}
\caption{Đặc tả Use Case UC-MD01-06: Gán Nhân viên vào Ca làm việc} \label{tab:uc_md01_06_revised} \\
\hline
\multicolumn{2}{|c|}{\textbf{2.1. Tóm tắt (Summary)}} \\
\hline
\textbf{Mục} & \textbf{Nội dung} \\
\hline
\endhead % Header cho các trang tiếp theo
\hline
\endfoot % Footer cho bảng
\hline
\endlastfoot % Footer cho trang cuối cùng
Use Case Name & Gán Nhân viên vào Ca làm việc \\
\hline
Use Case ID & UC-MD01-06 \\
\hline
Use Case Description & Cho phép Quản lý nhà hàng (US-01) chỉ định một nhân viên cụ thể vào một vị trí/vai trò còn trống trong một khung ca làm việc đang ở trạng thái Nháp, dựa trên sự phù hợp về vai trò và tính khả dụng của nhân viên. \\
\hline
Actor & US-01 (Quản lý nhà hàng) \\
\hline
Priority & Must Have \\
\hline
Trigger & Quản lý nhà hàng cần phân công nhân sự cụ thể cho các khung ca đã được tạo (UC-MD01-05). \\
\hline
Pre-Condition & - US-01 đã đăng nhập với quyền quản lý lịch trình. \newline - Tồn tại ít nhất một khung ca làm việc ở trạng thái "Nháp" có vị trí/vai trò chưa được gán. \newline - Dữ liệu nhân viên (với vai trò được định nghĩa và thông tin không sẵn sàng nếu có) đã tồn tại trong hệ thống. \\
\hline
Post-Condition & - Nhân viên được chọn được gán thành công vào vị trí/vai trò cụ thể trong bản ghi ca làm việc. \newline - Giao diện lịch trình (Gantt chart) được cập nhật, hiển thị tên nhân viên đã gán. \newline - Nếu việc gán gây trùng lịch, hệ thống sẽ đưa ra cảnh báo (Hệ thống tự động, người dùng xử lý ở UC-MD01-08). \\
\hline
\multicolumn{2}{|c|}{\textbf{2.2. Luồng thực thi (Flow)}} \\
\hline
\textbf{Mục} & \textbf{Nội dung} \\
\hline
Basic Flow & 1. US-01 đang xem giao diện quản lý lịch làm việc (ví dụ: Gantt chart - UC-MD01-07). \newline 2. US-01 chọn một vị trí/slot còn trống (chưa có nhân viên) trong một khung ca làm việc (đã tạo ở UC-MD01-05). Vị trí này ứng với một Vai trò cụ thể. \newline 3. Hệ thống hiển thị danh sách các nhân viên thỏa mãn các điều kiện sau (BR-UC1.6-1, BR-UC1.6-2): \newline    - Có Vai trò công việc phù hợp với yêu cầu của vị trí. \newline    - Khả dụng trong khung thời gian của ca làm việc (không bị đánh dấu không sẵn sàng - UC-MD01-15). \newline    - (Hệ thống có thể ưu tiên hiển thị nhân viên chưa bị trùng lịch). \newline 4. US-01 chọn một nhân viên từ danh sách hiển thị. \newline 5. US-01 xác nhận việc gán (ví dụ: nhấn nút "Gán", kéo thả nhân viên vào slot, hoặc chọn "Lưu" trên form ca nếu đang sửa ca). \newline 6. Hệ thống thực hiện lưu thông tin gán nhân viên vào vị trí đã chọn của ca làm việc. \newline 7. Hệ thống tự động kiểm tra xem việc gán này có gây trùng lịch cho nhân viên không. \newline    a. Nếu có trùng lịch, hệ thống đưa ra cảnh báo trực quan trên Gantt chart (ví dụ: đổi màu ca bị trùng). Việc xử lý chi tiết thuộc UC-MD01-08. Ca vẫn được gán. \newline 8. Hệ thống cập nhật giao diện Gantt chart, hiển thị tên nhân viên vừa được gán vào vị trí đó. \\
\hline
Alternative Flow & \textbf{3a. Lọc/Tìm kiếm nhân viên trong danh sách gợi ý:} \newline    1. Nếu danh sách nhân viên gợi ý dài, US-01 sử dụng chức năng tìm kiếm/lọc để nhanh chóng tìm nhân viên. \newline \textbf{5a. Gán bằng cách kéo thả trên Gantt:} \newline    1. US-01 kéo tên một nhân viên (từ danh sách nhân viên hoặc từ một ca khác của họ) và thả vào slot trống trên Gantt chart. \newline    2. Hệ thống tự động thực hiện việc gán. \\
\hline
Exception Flow & \textbf{3b. Không có nhân viên phù hợp và khả dụng:} \newline    1. Hệ thống không tìm thấy nhân viên nào thỏa mãn điều kiện. \newline    2. Hệ thống hiển thị thông báo "Không tìm thấy nhân viên phù hợp." \newline    3. US-01 không thể gán nhân viên cho slot đó tại thời điểm này. \newline \textbf{6a. Lỗi hệ thống khi gán:} \newline    1. Hệ thống gặp lỗi kỹ thuật khi lưu thông tin gán. \newline    2. Hệ thống báo lỗi chung. \\
\hline
\multicolumn{2}{|c|}{\textbf{2.3. Thông tin bổ sung (Additional Information)}} \\
\hline
\textbf{Mục} & \textbf{Nội dung} \\
\hline
Business Rule & - \textbf{BR-UC1.6-1:} Chỉ những nhân viên có vai trò khớp với yêu cầu của vị trí mới được gợi ý để gán. \newline - \textbf{BR-UC1.6-2:} Nhân viên sẽ không được gợi ý (hoặc được đánh dấu là không khả dụng) nếu họ đã báo không sẵn sàng (UC-MD01-15) trong thời gian ca làm việc. \newline - \textbf{BR-UC1.6-3:} Hệ thống cho phép gán nhân viên ngay cả khi gây trùng lịch, nhưng phải có cảnh báo rõ ràng (xử lý ở UC-MD01-08). \newline - \textbf{BR-UC1.6-4:} Việc gán nhân viên chỉ áp dụng cho các ca làm việc đang ở trạng thái "Nháp". \\
\hline
Non-Functional Requirement & - \textbf{NFR-UC1.6-1 (Performance):} Việc tải danh sách nhân viên gợi ý và thực hiện gán phải nhanh. \newline - \textbf{NFR-UC1.6-2 (Usability):} Giao diện gán nhân viên (danh sách gợi ý, kéo thả) phải trực quan. Thông tin về tính khả dụng của nhân viên cần rõ ràng. \\
\hline
\end{longtable}

\subsubsection{Use Case UC-MD01-07: Xem Lịch làm việc Tổng quan (Gantt)}

\begin{longtable}{|m{4cm}|p{11cm}|}
\caption{Đặc tả Use Case UC-MD01-07: Xem Lịch làm việc Tổng quan (Gantt)} \label{tab:uc_md01_07_revised} \\
\hline
\multicolumn{2}{|c|}{\textbf{2.1. Tóm tắt (Summary)}} \\
\hline
\textbf{Mục} & \textbf{Nội dung} \\
\hline
\endhead % Header cho các trang tiếp theo
\hline
\endfoot % Footer cho bảng
\hline
\endlastfoot % Footer cho trang cuối cùng
Use Case Name & Xem Lịch làm việc Tổng quan (Gantt) \\
\hline
Use Case ID & UC-MD01-07 \\
\hline
Use Case Description & Cho phép Quản lý nhà hàng (US-01) xem tổng quan lịch làm việc của nhân viên (bao gồm ca nháp và ca đã xuất bản) dưới dạng biểu đồ Gantt trực quan, với khả năng thay đổi khung thời gian hiển thị (ngày, tuần, tháng) và lọc/nhóm theo nhân viên hoặc vai trò công việc. \\
\hline
Actor & US-01 (Quản lý nhà hàng) \\
\hline
Priority & Must Have \\
\hline
Trigger & Quản lý nhà hàng muốn xem xét, đánh giá, kiểm tra, hoặc điều chỉnh lịch làm việc đã được phân công hoặc đang trong quá trình lập kế hoạch. \\
\hline
Pre-Condition & - US-01 đã đăng nhập vào hệ thống với quyền truy cập module Lập lịch. \newline - Có dữ liệu lịch làm việc (khung ca, ca đã gán) tồn tại trong hệ thống. \\
\hline
Post-Condition & - Biểu đồ Gantt hiển thị trực quan lịch làm việc theo các tiêu chí (thời gian, nhân viên/vai trò) do người dùng chọn hoặc theo mặc định. \newline - US-01 có cái nhìn tổng quan về việc phân bổ nhân sự, phát hiện khoảng trống hoặc chồng chéo. \\
\hline
\multicolumn{2}{|c|}{\textbf{2.2. Luồng thực thi (Flow)}} \\
\hline
\textbf{Mục} & \textbf{Nội dung} \\
\hline
Basic Flow & 1. US-01 truy cập vào module/chức năng quản lý Lịch làm việc (Planning). \newline 2. Hệ thống tự động tải và hiển thị biểu đồ Gantt với chế độ xem mặc định (ví dụ: Tuần hiện tại, nhóm theo Nhân viên - mỗi nhân viên một hàng). \newline 3. Mỗi ca làm việc được biểu diễn bằng một thanh (bar) trên dòng thời gian của nhân viên tương ứng. Màu sắc của thanh có thể biểu thị vai trò hoặc trạng thái (Nháp/Xuất bản). Chiều dài thanh biểu thị thời lượng ca. \newline 4. US-01 xem xét lịch trình hiển thị trên biểu đồ. \\
\hline
Alternative Flow & \textbf{4a. Thay đổi Thang Thời Gian:} \newline    1. US-01 chọn một thang thời gian khác (Ngày, Tuần, Tháng) từ các nút điều khiển. \newline    2. Hệ thống tải lại và hiển thị Gantt chart cho thang thời gian mới. \newline \textbf{4b. Lọc/Nhóm theo Vai Trò:} \newline    1. US-01 chọn tùy chọn nhóm theo Vai trò. \newline    2. Hệ thống hiển thị Gantt chart với mỗi hàng là một Vai trò, các ca của nhân viên thuộc vai trò đó được hiển thị trên cùng hàng hoặc các hàng con. \newline    3. Hoặc, US-01 chọn lọc để chỉ hiển thị các nhân viên/ca thuộc một Vai trò cụ thể. \newline \textbf{4c. Điều hướng Thời gian:} \newline    1. US-01 sử dụng nút điều hướng ("<", ">") để xem tuần/tháng trước hoặc tuần/tháng tiếp theo. \newline \textbf{4d. Xem chi tiết ca (Hover/Click):} \newline    1. US-01 di chuột qua một thanh ca. Hệ thống hiển thị tooltip tóm tắt (Nhân viên, Vai trò, Giờ, Trạng thái). \newline    2. US-01 nhấp vào thanh ca để mở form chi tiết của ca đó (cho phép sửa nếu là Nháp). \\
\hline
Exception Flow & \textbf{2a. Không có dữ liệu lịch trình:} \newline    1. Hệ thống không tìm thấy ca làm việc nào. \newline    2. Hệ thống hiển thị biểu đồ trống hoặc thông báo "Không có lịch trình". \newline \textbf{2b. Lỗi tải dữ liệu:} \newline    1. Hệ thống gặp lỗi khi truy vấn hoặc hiển thị dữ liệu. \newline    2. Hệ thống báo lỗi chung. \\
\hline
\multicolumn{2}{|c|}{\textbf{2.3. Thông tin bổ sung (Additional Information)}} \\
\hline
\textbf{Mục} & \textbf{Nội dung} \\
\hline
Business Rule & - \textbf{BR-UC1.7-1:} Chế độ xem mặc định là Tuần hiện tại, nhóm theo Nhân viên. \newline - \textbf{BR-UC1.7-2:} Phân biệt rõ ràng giữa ca Nháp (ví dụ: sọc chéo) và ca Đã xuất bản (màu liền). \newline - \textbf{BR-UC1.7-3:} Cảnh báo trùng lịch (từ hệ thống) phải được hiển thị rõ ràng trên Gantt chart (ví dụ: ca bị đỏ). \\
\hline
Non-Functional Requirement & - \textbf{NFR-UC1.7-1 (Performance):} Tải Gantt chart phải nhanh, kể cả với nhiều nhân viên/ca. \newline - \textbf{NFR-UC1.7-2 (Usability):} Biểu đồ dễ đọc, dễ điều hướng, các tùy chọn lọc/nhóm phải trực quan. \newline - \textbf{NFR-UC1.7-3 (Accuracy):} Dữ liệu trên Gantt phải chính xác và đồng bộ. \\
\hline
\end{longtable}

\subsubsection{Use Case UC-MD01-08: Xử lý Cảnh báo Trùng lịch}

\begin{longtable}{|m{4cm}|p{11cm}|}
\caption{Đặc tả Use Case UC-MD01-08: Xử lý Cảnh báo Trùng lịch} \label{tab:uc_md01_08_revised} \\
\hline
\multicolumn{2}{|c|}{\textbf{2.1. Tóm tắt (Summary)}} \\
\hline
\textbf{Mục} & \textbf{Nội dung} \\
\hline
\endhead % Header cho các trang tiếp theo
\hline
\endfoot % Footer cho bảng
\hline
\endlastfoot % Footer cho trang cuối cùng
Use Case Name & Xử lý Cảnh báo Trùng lịch \\
\hline
Use Case ID & UC-MD01-08 \\
\hline
Use Case Description & Khi Quản lý nhà hàng (US-01) thực hiện gán ca cho nhân viên (UC-MD01-06) và hệ thống tự động phát hiện việc gán đó gây ra trùng lịch cho nhân viên (hai hoặc nhiều ca có thời gian chồng chéo), hệ thống sẽ hiển thị cảnh báo trực quan. Use Case này mô tả hành động của Quản lý khi nhìn thấy cảnh báo đó và quyết định cách xử lý. \\
\hline
Actor & US-01 (Quản lý nhà hàng) \\
\hline
Priority & Must Have \\
\hline
Trigger & - Hệ thống hiển thị cảnh báo trực quan (ví dụ: các ca bị đổi thành màu đỏ) trên lịch biểu Gantt sau khi một nhân viên được gán vào ca gây trùng lịch. \newline - Quản lý nhà hàng chủ động rà soát lịch và phát hiện các ca bị đánh dấu trùng lịch. \\
\hline
Pre-Condition & - US-01 đang xem lịch làm việc tổng quan (UC-MD01-07). \newline - Có ít nhất một nhân viên được gán vào các ca làm việc có thời gian bị trùng lặp và hệ thống đã đánh dấu cảnh báo. \\
\hline
Post-Condition & - Xung đột lịch được giải quyết (ví dụ: một ca bị hủy gán, đổi giờ, đổi nhân viên). \newline - HOẶC Quản lý chấp nhận tình trạng trùng lịch (nếu hệ thống cho phép) và cảnh báo vẫn còn đó. \newline - Lịch trình được cập nhật. \\
\hline
\multicolumn{2}{|c|}{\textbf{2.2. Luồng thực thi (Flow)}} \\
\hline
\textbf{Mục} & \textbf{Nội dung} \\
\hline
Basic Flow (Điều chỉnh để giải quyết xung đột) & 1. US-01 quan sát lịch biểu Gantt và nhận thấy (các) ca làm việc của một nhân viên (ví dụ: Nhân viên A) đang được đánh dấu là trùng lịch (ví dụ: màu đỏ). \newline 2. US-01 nhấp vào một trong các ca bị trùng để xem chi tiết hoặc để bắt đầu điều chỉnh. \newline 3. US-01 đánh giá tình hình và quyết định một trong các hành động sau để giải quyết xung đột: \newline    a. \textbf{Hủy gán nhân viên khỏi một ca:} Mở chi tiết ca bị trùng, xóa Nhân viên A khỏi ca đó (để ca đó trở thành slot trống hoặc gán cho người khác). \newline    b. \textbf{Thay đổi thời gian của một ca:} Điều chỉnh giờ bắt đầu/kết thúc của một trong các ca bị trùng sao cho chúng không còn chồng chéo nữa (có thể thực hiện bằng cách kéo thả trên Gantt - UC-MD01-09, hoặc sửa trên form chi tiết ca). \newline    c. \textbf{Gán một trong các ca cho nhân viên khác:} Giữ nguyên thời gian ca, nhưng đổi Nhân viên A sang một nhân viên khác phù hợp và khả dụng cho một trong các ca bị trùng. \newline 4. Sau khi thực hiện điều chỉnh, US-01 chọn "Lưu" (nếu đang sửa trên form) hoặc hệ thống tự động lưu (nếu kéo thả trên Gantt). \newline 5. Hệ thống kiểm tra lại. Nếu xung đột đã được giải quyết, cảnh báo trực quan (màu đỏ) trên các ca đó biến mất. \\
\hline
Alternative Flow & \textbf{3d. Chấp nhận trùng lịch (Nếu hệ thống/chính sách cho phép):} \newline    1. US-01 xem xét và quyết định rằng việc trùng lịch là chấp nhận được trong trường hợp cụ thể này (ví dụ: nhân viên đồng ý làm thêm giờ chồng gối, hoặc một ca là đào tạo không thực sự làm việc). \newline    2. US-01 không thực hiện hành động nào để thay đổi việc gán ca. Cảnh báo trùng lịch có thể vẫn còn hiển thị. \\
\hline
Exception Flow & \textbf{4a. Lỗi khi lưu thay đổi:} \newline    1. Hệ thống gặp lỗi khi cố gắng lưu các điều chỉnh của US-01. \newline    2. Thay đổi không được lưu, cảnh báo trùng lịch có thể vẫn còn. \newline \textbf{4b. Điều chỉnh mới lại gây ra trùng lịch khác:} \newline    1. Khi US-01 điều chỉnh một ca, nó lại gây ra một xung đột mới với một ca khác của cùng nhân viên hoặc của nhân viên được gán mới. \newline    2. Hệ thống sẽ hiển thị cảnh báo cho xung đột mới này. US-01 cần tiếp tục xử lý. \\
\hline
\multicolumn{2}{|c|}{\textbf{2.3. Thông tin bổ sung (Additional Information)}} \\
\hline
\textbf{Mục} & \textbf{Nội dung} \\
\hline
Business Rule & - \textbf{BR-UC1.8-1:} Hệ thống phải cung cấp cảnh báo trực quan rõ ràng khi phát hiện trùng lịch. \newline - \textbf{BR-UC1.8-2:} Quản lý nhà hàng có trách nhiệm xem xét và xử lý các cảnh báo trùng lịch để đảm bảo lịch trình hợp lý, trừ khi có lý do đặc biệt để chấp nhận trùng lịch. \newline - \textbf{BR-UC1.8-3:} Việc giải quyết trùng lịch có thể bao gồm việc thay đổi nhân viên, thay đổi thời gian ca, hoặc hủy bỏ việc gán nhân viên cho một ca. \\
\hline
Non-Functional Requirement & - \textbf{NFR-UC1.8-1 (Usability):} Cảnh báo trùng lịch phải dễ nhận biết. Các công cụ để điều chỉnh ca (sửa form, kéo thả trên Gantt) phải dễ sử dụng. \newline - \textbf{NFR-UC1.8-2 (Responsiveness):} Sau khi quản lý điều chỉnh, hệ thống phải nhanh chóng kiểm tra lại và cập nhật trạng thái cảnh báo. \\
\hline
\end{longtable}

\subsubsection{Use Case UC-MD01-09: Điều chỉnh Ca làm việc trên Gantt}

\begin{longtable}{|m{4cm}|p{11cm}|}
\caption{Đặc tả Use Case UC-MD01-09: Điều chỉnh Ca làm việc trên Gantt} \label{tab:uc_md01_09_revised} \\
\hline
\multicolumn{2}{|c|}{\textbf{2.1. Tóm tắt (Summary)}} \\
\hline
\textbf{Mục} & \textbf{Nội dung} \\
\hline
\endhead % Header cho các trang tiếp theo
\hline
\endfoot % Footer cho bảng
\hline
\endlastfoot % Footer cho trang cuối cùng
Use Case Name & Điều chỉnh Ca làm việc trên Gantt \\
\hline
Use Case ID & UC-MD01-09 \\
\hline
Use Case Description & Cho phép Quản lý nhà hàng (US-01) tương tác trực tiếp với biểu đồ Gantt để thực hiện các điều chỉnh nhanh đối với các ca làm việc (đang ở trạng thái Nháp), như thay đổi thời gian bắt đầu/kết thúc bằng cách kéo dài/thu ngắn thanh ca, hoặc di chuyển toàn bộ ca sang một thời điểm khác hoặc gán cho nhân viên khác bằng thao tác kéo thả. \\
\hline
Actor & US-01 (Quản lý nhà hàng) \\
\hline
Priority & Should Have \\
\hline
Trigger & Quản lý nhà hàng đang xem lịch biểu Gantt (UC-MD01-07) và cần thực hiện các thay đổi nhỏ, nhanh chóng cho một hoặc nhiều ca làm việc Nháp. \\
\hline
Pre-Condition & - US-01 đang xem lịch làm việc tổng quan dạng Gantt. \newline - Có các ca làm việc (đặc biệt là ca Nháp) hiển thị trên biểu đồ. \\
\hline
Post-Condition & - Thông tin (thời gian, nhân viên được gán) của ca làm việc được điều chỉnh và lưu lại. \newline - Biểu đồ Gantt được cập nhật để phản ánh thay đổi. \newline - Nếu thay đổi gây trùng lịch, hệ thống hiển thị cảnh báo (xử lý ở UC-MD01-08). \\
\hline
\multicolumn{2}{|c|}{\textbf{2.2. Luồng thực thi (Flow)}} \\
\hline
\textbf{Mục} & \textbf{Nội dung} \\
\hline
Basic Flow (Thay đổi thời lượng/thời điểm ca) & 1. US-01 đang xem lịch biểu Gantt. \newline 2. US-01 xác định một ca làm việc (thường là ca Nháp) cần điều chỉnh. \newline 3. \textbf{Để thay đổi thời gian kết thúc:} US-01 di chuột đến cạnh phải của thanh biểu diễn ca, con trỏ chuột thay đổi thành biểu tượng thay đổi kích thước. US-01 nhấn giữ và kéo cạnh phải sang trái (thu ngắn) hoặc sang phải (kéo dài) đến thời điểm kết thúc mong muốn rồi thả chuột. \newline 4. \textbf{Để thay đổi thời gian bắt đầu:} Tương tự bước 3 nhưng với cạnh trái của thanh ca. \newline 5. \textbf{Để di chuyển toàn bộ ca sang thời điểm khác (cùng nhân viên):} US-01 nhấn giữ vào giữa thanh ca và kéo nó đến một vị trí thời gian mới trên cùng dòng nhân viên rồi thả chuột. \newline 6. Sau khi thả chuột, hệ thống tự động cập nhật thời gian mới cho ca làm việc đó. \newline 7. Hệ thống kiểm tra trùng lịch (nếu có thay đổi về thời gian hoặc nhân viên) và hiển thị cảnh báo nếu cần. \newline 8. Thay đổi được lưu tự động (hoặc có thể cần một hành động lưu tổng thể tùy cấu hình). \\
\hline
Alternative Flow & \textbf{5a. Di chuyển ca sang nhân viên khác (Re-assign):} \newline    1. US-01 nhấn giữ vào giữa thanh ca và kéo nó từ dòng của nhân viên hiện tại sang một dòng của nhân viên khác, vào một khoảng thời gian mong muốn. \newline    2. Hệ thống cập nhật ca đó được gán cho nhân viên mới và vào thời điểm mới. \newline    3. Use Case tiếp tục từ bước 7. \\
\hline
Exception Flow & \textbf{6a. Thao tác không hợp lệ / Bị chặn:} \newline    1. US-01 cố gắng kéo thả ca vào một vị trí không hợp lệ (ví dụ: kéo ca chồng lên một ca đã khóa, hoặc kéo ra ngoài phạm vi lịch cho phép). \newline    2. Hệ thống không cho phép thả hoặc hoàn tác lại vị trí cũ, có thể kèm thông báo. \newline \textbf{6b. Lỗi lưu thay đổi:} \newline    1. Hệ thống gặp lỗi khi cố gắng lưu các thay đổi do tương tác trên Gantt. \newline    2. Hệ thống báo lỗi, thay đổi có thể không được áp dụng. \\
\hline
\multicolumn{2}{|c|}{\textbf{2.3. Thông tin bổ sung (Additional Information)}} \\
\hline
\textbf{Mục} & \textbf{Nội dung} \\
\hline
Business Rule & - \textbf{BR-UC1.9-1:} Các thao tác điều chỉnh trực tiếp trên Gantt thường chỉ nên áp dụng cho các ca làm việc đang ở trạng thái "Nháp". Ca "Đã xuất bản" có thể bị khóa hoặc yêu cầu quy trình khác để thay đổi. \newline - \textbf{BR-UC1.9-2:} Mọi thay đổi về thời gian hoặc nhân viên đều phải kích hoạt lại việc kiểm tra trùng lịch của hệ thống. \\
\hline
Non-Functional Requirement & - \textbf{NFR-UC1.9-1 (Usability):} Các thao tác kéo thả, thay đổi kích thước trên Gantt phải mượt mà, chính xác và có phản hồi trực quan rõ ràng cho người dùng. \newline - \textbf{NFR-UC1.9-2 (Performance):} Phản hồi của hệ thống sau khi người dùng thả chuột (cập nhật, kiểm tra trùng lịch) phải nhanh chóng. \\
\hline
\end{longtable}

\subsubsection{Use Case UC-MD01-10: Sao chép Lịch làm việc (Ngày/Tuần/Tùy chỉnh)}

\begin{longtable}{|m{4cm}|p{11cm}|}
\caption{Đặc tả Use Case UC-MD01-10: Sao chép Lịch làm việc (Ngày/Tuần/Tùy chỉnh)} \label{tab:uc_md01_10_revised} \\
\hline
\multicolumn{2}{|c|}{\textbf{2.1. Tóm tắt (Summary)}} \\
\hline
\textbf{Mục} & \textbf{Nội dung} \\
\hline
\endhead % Header cho các trang tiếp theo
\hline
\endfoot % Footer cho bảng
\hline
\endlastfoot % Footer cho trang cuối cùng
Use Case Name & Sao chép Lịch làm việc (Ngày/Tuần/Tùy chỉnh) \\
\hline
Use Case ID & UC-MD01-10 \\
\hline
Use Case Description & Cho phép Quản lý nhà hàng (US-01) tạo nhanh lịch làm việc cho một khoảng thời gian mới (ví dụ: ngày mai, tuần tới) bằng cách sao chép toàn bộ cấu trúc ca và phân công nhân viên (tùy chọn) từ một khoảng thời gian đã có lịch trước đó. \\
\hline
Actor & US-01 (Quản lý nhà hàng) \\
\hline
Priority & Must Have \\
\hline
Trigger & Quản lý nhà hàng muốn tiết kiệm thời gian khi lập lịch cho một giai đoạn mới, dựa trên một lịch trình cũ tương tự. \\
\hline
Pre-Condition & - US-01 đã đăng nhập với quyền quản lý lịch trình. \newline - Tồn tại lịch làm việc (ít nhất một ca) trong khoảng thời gian được chọn làm nguồn. \\
\hline
Post-Condition & - Các ca làm việc mới, giống (về thời gian tương đối, vai trò, tùy chọn cả nhân viên) các ca của khoảng thời gian nguồn, được tạo ra trong khoảng thời gian đích với trạng thái "Nháp". \newline - Giao diện Gantt chart cập nhật, hiển thị các ca nháp mới. \newline - US-01 có thể chỉnh sửa các ca nháp mới này trước khi xuất bản. \\
\hline
\multicolumn{2}{|c|}{\textbf{2.2. Luồng thực thi (Flow)}} \\
\hline
\textbf{Mục} & \textbf{Nội dung} \\
\hline
Basic Flow (Sao chép Tuần) & 1. US-01 đang xem lịch biểu Gantt, hiển thị tuần muốn dùng làm nguồn (Tuần Nguồn). \newline 2. US-01 chọn hành động "Sao chép Tuần" (Copy Week) (thường có trong menu hoặc nút lệnh của giao diện Planning). \newline 3. Hệ thống (thường mặc định) đề xuất sao chép sang tuần kế tiếp (Tuần Đích). US-01 có thể thay đổi Tuần Đích nếu cần. \newline 4. (Tùy chọn) Hệ thống có thể hỏi xem có muốn sao chép cả phân công nhân viên hay chỉ sao chép khung ca. US-01 lựa chọn. \newline 5. US-01 xác nhận hành động sao chép. \newline 6. Hệ thống truy xuất thông tin tất cả các ca làm việc (thời gian, vai trò, nhân viên đã gán - nếu được chọn sao chép) thuộc Tuần Nguồn. \newline 7. Đối với mỗi ca làm việc từ Tuần Nguồn, hệ thống tạo một bản ghi ca mới trong Tuần Đích với cùng ngày trong tuần và thời gian, cùng vai trò, và cùng nhân viên (nếu đã chọn). \newline 8. Tất cả các ca mới tạo ở Tuần Đích đều có trạng thái "Nháp". \newline 9. Hệ thống cập nhật Gantt chart để hiển thị các ca mới ở Tuần Đích. \newline 10. Hệ thống hiển thị thông báo sao chép thành công. \\
\hline
Alternative Flow & \textbf{2a. Sao chép Ngày:} \newline    1. US-01 chọn một ngày cụ thể làm nguồn và chọn hành động "Sao chép Ngày". \newline    2. US-01 chọn ngày đích. \newline    3. Các bước tiếp theo tương tự, nhưng chỉ sao chép các ca của ngày nguồn sang ngày đích. \newline \textbf{2b. Sao chép khoảng thời gian tùy chỉnh:} \newline    1. US-01 chọn một khoảng thời gian nguồn (ví dụ: 3 ngày) và hành động "Sao chép Lịch trình". \newline    2. US-01 chọn ngày bắt đầu cho khoảng thời gian đích. \newline    3. Các bước tiếp theo tương tự. \\
\hline
Exception Flow & \textbf{6a. Khoảng thời gian nguồn không có ca nào:} \newline    1. Hệ thống không tìm thấy ca làm việc nào trong khoảng thời gian nguồn đã chọn. \newline    2. Hệ thống báo "Không có lịch trình để sao chép." \newline \textbf{7a. Lỗi hệ thống khi tạo ca mới:} \newline    1. Hệ thống gặp lỗi kỹ thuật khi tạo bản ghi ca mới. \newline    2. Hệ thống báo lỗi, quá trình sao chép có thể bị gián đoạn. \\
\hline
\multicolumn{2}{|c|}{\textbf{2.3. Thông tin bổ sung (Additional Information)}} \\
\hline
\textbf{Mục} & \textbf{Nội dung} \\
\hline
Business Rule & - \textbf{BR-UC1.10-1:} Tất cả các ca được tạo từ sao chép đều phải ở trạng thái "Nháp". \newline - \textbf{BR-UC1.10-2:} Việc có sao chép phân công nhân viên hay không là một tùy chọn quan trọng. Mặc định thường là có. \newline - \textbf{BR-UC1.10-3:} Sau khi sao chép, các ca mới phải tuân theo quy tắc kiểm tra trùng lịch khi quản lý xem xét. \\
\hline
Non-Functional Requirement & - \textbf{NFR-UC1.10-1 (Usability):} Chức năng sao chép phải dễ sử dụng, việc chọn khoảng thời gian nguồn/đích phải trực quan. \newline - \textbf{NFR-UC1.10-2 (Performance):} Sao chép lịch của một tuần (~200 ca) phải dưới 5 giây. \newline - \textbf{NFR-UC1.10-3 (Accuracy):} Thông tin ca sao chép phải chính xác. \\
\hline
\end{longtable}

\subsubsection{Use Case UC-MD01-11: Lưu Lịch làm việc Nháp}

\begin{longtable}{|m{4cm}|p{11cm}|}
\caption{Đặc tả Use Case UC-MD01-11: Lưu Lịch làm việc Nháp} \label{tab:uc_md01_11_revised} \\
\hline
\multicolumn{2}{|c|}{\textbf{2.1. Tóm tắt (Summary)}} \\
\hline
\textbf{Mục} & \textbf{Nội dung} \\
\hline
\endhead % Header cho các trang tiếp theo
\hline
\endfoot % Footer cho bảng
\hline
\endlastfoot % Footer cho trang cuối cùng
Use Case Name & Lưu Lịch làm việc Nháp \\
\hline
Use Case ID & UC-MD01-11 \\
\hline
Use Case Description & Cho phép Quản lý nhà hàng (US-01) lưu lại các thay đổi đã thực hiện trên lịch trình (tạo khung ca, gán nhân viên, điều chỉnh ca) dưới dạng bản nháp mà chưa cần công khai chính thức cho nhân viên. \\
\hline
Actor & US-01 (Quản lý nhà hàng) \\
\hline
Priority & Must Have \\
\hline
Trigger & Quản lý nhà hàng đã thực hiện một số thay đổi trong việc lập lịch và muốn lưu lại tiến độ công việc mà chưa sẵn sàng xuất bản. \\
\hline
Pre-Condition & - US-01 đang trong giao diện quản lý lịch làm việc. \newline - Đã có ít nhất một thay đổi được thực hiện trên lịch trình (ví dụ: tạo ca mới, gán nhân viên, sửa ca). \\
\hline
Post-Condition & - Tất cả các thay đổi về khung ca, phân công nhân viên được lưu vào cơ sở dữ liệu với trạng thái "Nháp" (Draft) cho các ca/phần chưa được xuất bản. \newline - Lịch trình nháp này chỉ quản lý nhìn thấy và có thể tiếp tục chỉnh sửa sau. \newline - Nhân viên chưa nhận được thông báo về các thay đổi này. \\
\hline
\multicolumn{2}{|c|}{\textbf{2.2. Luồng thực thi (Flow)}} \\
\hline
\textbf{Mục} & \textbf{Nội dung} \\
\hline
Basic Flow & 1. US-01 đã thực hiện các thao tác tạo khung ca (UC-MD01-05), gán nhân viên (UC-MD01-06), hoặc điều chỉnh ca trên Gantt (UC-MD01-09). \newline 2. US-01 chọn hành động "Lưu" (Save) trên giao diện chung của module Planning hoặc khi đóng một form chi tiết ca.
\newline 3. Hệ thống ghi nhận tất cả các thay đổi vào cơ sở dữ liệu. Các ca mới tạo hoặc ca được sửa đổi mà chưa từng xuất bản sẽ giữ/có trạng thái "Nháp". \newline 4. Hệ thống hiển thị thông báo lưu thành công (nếu có nút lưu riêng) hoặc các thay đổi được phản ánh ngay trên Gantt. \\
\hline
Alternative Flow & \textbf{2a. Tự động lưu (Autosave):} \newline    1. Một số hành động (ví dụ: kéo thả trên Gantt) có thể được hệ thống tự động lưu lại ngay sau khi thực hiện mà không cần US-01 nhấn nút "Lưu" riêng. \\
\hline
Exception Flow & \textbf{3a. Lỗi hệ thống khi lưu:} \newline    1. Hệ thống gặp lỗi kỹ thuật khi cố gắng lưu dữ liệu lịch trình nháp. \newline    2. Hệ thống báo lỗi chung. Các thay đổi có thể không được lưu. \\
\hline
\multicolumn{2}{|c|}{\textbf{2.3. Thông tin bổ sung (Additional Information)}} \\
\hline
\textbf{Mục} & \textbf{Nội dung} \\
\hline
Business Rule & - \textbf{BR-UC1.11-1:} Tất cả các ca làm việc mới được tạo hoặc điều chỉnh (mà chưa được xuất bản trước đó) đều phải ở trạng thái "Nháp". \newline - \textbf{BR-UC1.11-2:} Lịch trình nháp chỉ hiển thị cho người dùng có quyền quản lý lịch, không hiển thị cho nhân viên thường. \\
\hline
Non-Functional Requirement & - \textbf{NFR-UC1.11-1 (Reliability):} Việc lưu nháp phải đáng tin cậy, đảm bảo không mất dữ liệu khi quản lý đang làm việc. \newline - \textbf{NFR-UC1.11-2 (Performance):} Thao tác lưu (kể cả autosave) phải nhanh chóng, không làm gián đoạn quá trình lập lịch. \\
\hline
\end{longtable}

\subsubsection{Use Case UC-MD01-12: Xuất bản Lịch làm việc}

\begin{longtable}{|m{4cm}|p{11cm}|}
\caption{Đặc tả Use Case UC-MD01-12: Xuất bản Lịch làm việc} \label{tab:uc_md01_12_revised} \\
\hline
\multicolumn{2}{|c|}{\textbf{2.1. Tóm tắt (Summary)}} \\
\hline
\textbf{Mục} & \textbf{Nội dung} \\
\hline
\endhead % Header cho các trang tiếp theo
\hline
\endfoot % Footer cho bảng
\hline
\endlastfoot % Footer cho trang cuối cùng
Use Case Name & Xuất bản Lịch làm việc \\
\hline
Use Case ID & UC-MD01-12 \\
\hline
Use Case Description & Cho phép Quản lý nhà hàng (US-01) chính thức hóa một phần hoặc toàn bộ lịch làm việc đã được xếp (chuyển các ca từ trạng thái "Nháp" sang "Đã xuất bản"). Lịch đã xuất bản sẽ sẵn sàng để nhân viên xem. \\
\hline
Actor & US-01 (Quản lý nhà hàng) \\
\hline
Priority & Must Have \\
\hline
Trigger & Quản lý nhà hàng đã hoàn tất việc lập lịch cho một khoảng thời gian (ví dụ: tuần tới) và muốn công bố chính thức cho nhân viên. \\
\hline
Pre-Condition & - US-01 đã đăng nhập với quyền quản lý lịch trình. \newline - Tồn tại ít nhất một ca làm việc (đã gán nhân viên) ở trạng thái "Nháp" trong phạm vi cần xuất bản. \newline - US-01 đã rà soát và hài lòng với lịch trình nháp. \\
\hline
Post-Condition & - Các ca làm việc được chọn trong phạm vi xuất bản chuyển trạng thái từ "Nháp" thành "Đã xuất bản" (Published). \newline - Các ca "Đã xuất bản" trở nên chính thức và có thể được xem bởi nhân viên liên quan (thông qua UC-MD01-14). \newline - Giao diện Gantt chart cập nhật, thể hiện trạng thái "Đã xuất bản" của các ca (ví dụ: mất sọc chéo, đổi màu). \\
\hline
\multicolumn{2}{|c|}{\textbf{2.2. Luồng thực thi (Flow)}} \\
\hline
\textbf{Mục} & \textbf{Nội dung} \\
\hline
Basic Flow & 1. US-01 truy cập giao diện quản lý lịch làm việc (ví dụ: Gantt view). \newline 2. US-01 chọn (các) ca làm việc ở trạng thái "Nháp" muốn xuất bản HOẶC chọn một hành động xuất bản theo phạm vi (ví dụ: nút "Publish" cho tuần hiện tại/kế tiếp, hoặc chọn nhiều ca rồi chọn "Publish"). \newline 3. Hệ thống (có thể) hiển thị danh sách các ca sẽ được xuất bản và yêu cầu xác nhận từ US-01. \newline 4. US-01 xác nhận hành động xuất bản. \newline 5. Hệ thống xác định tất cả các ca làm việc "Nháp" thuộc phạm vi đã chọn. \newline 6. Hệ thống cập nhật trạng thái của các ca này thành "Đã xuất bản" (Published) trong cơ sở dữ liệu. \newline 7. Hệ thống cập nhật giao diện Gantt chart để phản ánh trạng thái "Đã xuất bản" mới của các ca. \newline 8. Hệ thống hiển thị thông báo "Lịch trình đã được xuất bản thành công." \\
\hline
Alternative Flow & \textbf{2a. Xuất bản kèm gửi thông báo:} \newline    1. Nút "Publish" có thể được thiết kế để bao gồm cả việc gửi thông báo (UC-MD01-13) ngay sau khi xuất bản. Trong trường hợp đó, Use Case này và UC-MD01-13 sẽ diễn ra nối tiếp. \\
\hline
Exception Flow & \textbf{2b. Không có ca nháp nào được chọn/tìm thấy:} \newline    1. Hệ thống không tìm thấy ca "Nháp" nào trong phạm vi US-01 đã chọn. \newline    2. Hệ thống hiển thị thông báo "Không có ca làm việc nào ở trạng thái Nháp để xuất bản." \newline \textbf{6a. Lỗi cập nhật trạng thái ca:} \newline    1. Hệ thống gặp lỗi khi cố gắng cập nhật trạng thái ca trong cơ sở dữ liệu. \newline    2. Hệ thống báo lỗi "Không thể cập nhật trạng thái ca làm việc." \newline    3. Các ca có thể vẫn ở trạng thái "Nháp". \\
\hline
\multicolumn{2}{|c|}{\textbf{2.3. Thông tin bổ sung (Additional Information)}} \\
\hline
\textbf{Mục} & \textbf{Nội dung} \\
\hline
Business Rule & - \textbf{BR-UC1.12-1:} Chỉ các ca làm việc ở trạng thái "Nháp" và đã được gán nhân viên mới nên được chuyển sang trạng thái "Đã xuất bản". \newline - \textbf{BR-UC1.12-2:} Hành động "Xuất bản" là hành động chính thức hóa lịch làm việc. \newline - \textbf{BR-UC1.12-3:} Sau khi xuất bản, việc thay đổi ca (ví dụ: hủy ca, đổi nhân viên) có thể yêu cầu quy trình khác (ví dụ: hủy xuất bản trước, hoặc gửi thông báo cập nhật cho nhân viên bị ảnh hưởng). \\
\hline
Non-Functional Requirement & - \textbf{NFR-UC1.12-1 (Usability):} Hành động xuất bản phải rõ ràng. \newline - \textbf{NFR-UC1.12-2 (Performance):} Việc xuất bản lịch trình của một tuần (~200 ca) phải nhanh chóng. \\
\hline
\end{longtable}

\subsubsection{Use Case UC-MD01-13: Gửi Thông báo Lịch làm việc cho Nhân viên}

\begin{longtable}{|m{4cm}|p{11cm}|}
\caption{Đặc tả Use Case UC-MD01-13: Gửi Thông báo Lịch làm việc cho Nhân viên} \label{tab:uc_md01_13_revised} \\
\hline
\multicolumn{2}{|c|}{\textbf{2.1. Tóm tắt (Summary)}} \\
\hline
\textbf{Mục} & \textbf{Nội dung} \\
\hline
\endhead % Header cho các trang tiếp theo
\hline
\endfoot % Footer cho bảng
\hline
\endlastfoot % Footer cho trang cuối cùng
Use Case Name & Gửi Thông báo Lịch làm việc cho Nhân viên \\
\hline
Use Case ID & UC-MD01-13 \\
\hline
Use Case Description & Sau khi Quản lý nhà hàng (US-01) đã xuất bản lịch làm việc (UC-MD01-12), cho phép Quản lý kích hoạt hệ thống tự động gửi thông báo (qua email hoặc thông báo trong ứng dụng) đến từng nhân viên có ca trong lịch vừa xuất bản, thông báo về lịch trình cá nhân của họ. \\
\hline
Actor & US-01 (Quản lý nhà hàng - Kích hoạt), System (Thực hiện gửi) \\
\hline
Priority & Must Have \\
\hline
Trigger & - Lịch làm việc vừa được xuất bản (kết thúc UC-MD01-12). \newline - Hoặc Quản lý muốn gửi lại thông báo cho một lịch đã xuất bản trước đó. \\
\hline
Pre-Condition & - Lịch làm việc (ít nhất một phần) đã ở trạng thái "Đã xuất bản". \newline - Thông tin liên hệ (email) của nhân viên đã được cấu hình chính xác trong hệ thống. \newline
\hline
Post-Condition & - Thông báo chứa tóm tắt lịch trình cá nhân được gửi (hoặc đưa vào hàng đợi gửi) thành công đến kênh liên lạc của từng nhân viên có ca được xuất bản/chọn để thông báo. \newline - Nhân viên được thông báo về lịch làm việc của mình. \\
\hline
\multicolumn{2}{|c|}{\textbf{2.2. Luồng thực thi (Flow)}} \\
\hline
\textbf{Mục} & \textbf{Nội dung} \\
\hline
Basic Flow (Thường đi kèm Xuất bản) & 1. Tiếp nối từ UC-MD01-12 (Xuất bản lịch), hoặc US-01 chọn các ca đã xuất bản và chọn hành động "Gửi thông báo" / "Send Schedule". \newline 2. Hệ thống xác định danh sách các nhân viên có ca làm việc trong phạm vi được xuất bản/chọn. \newline 3. Đối với mỗi nhân viên trong danh sách: \newline    a. Hệ thống tạo nội dung thông báo (ví dụ: email) liệt kê các ca làm việc (Ngày, Giờ bắt đầu, Giờ kết thúc, Vai trò) của nhân viên đó trong phạm vi liên quan. \newline    b. Hệ thống gửi thông báo này đến địa chỉ email (hoặc kênh khác) đã đăng ký của nhân viên. \newline 4. Hệ thống hiển thị thông báo "Thông báo lịch trình đã được gửi." cho US-01. \\
\hline
Alternative Flow & \textbf{1a. Gửi thông báo riêng lẻ:} \newline    1. US-01 chọn một hoặc nhiều ca đã xuất bản của một hoặc nhiều nhân viên. \newline    2. US-01 chọn hành động "Gửi thông báo lịch". \newline    3. Hệ thống chỉ gửi thông báo cho các nhân viên liên quan đến các ca đã chọn. \\
\hline
Exception Flow & \textbf{3c. Lỗi gửi thông báo:} \newline    1. Hệ thống gặp lỗi khi cố gắng tạo hoặc gửi thông báo cho một hoặc nhiều nhân viên (ví dụ: email không hợp lệ, lỗi máy chủ mail). \newline    2. Hệ thống (nên) ghi nhận lỗi chi tiết vào log. \newline    3. Hệ thống có thể thông báo cho US-01 về việc gửi lỗi cho một số nhân viên. \\
\hline
\multicolumn{2}{|c|}{\textbf{2.3. Thông tin bổ sung (Additional Information)}} \\
\hline
\textbf{Mục} & \textbf{Nội dung} \\
\hline
Business Rule & - \textbf{BR-UC1.13-1:} Thông báo chỉ nên được gửi cho các ca đã ở trạng thái "Đã xuất bản". \newline - \textbf{BR-UC1.13-2:} Thông báo phải được gửi đến đúng kênh liên lạc đã đăng ký của nhân viên. \newline - \textbf{BR-UC1.13-3:} Nội dung thông báo phải rõ ràng, bao gồm các chi tiết ca làm việc cần thiết. \\
\hline
Non-Functional Requirement & - \textbf{NFR-UC1.13-1 (Reliability):} Hệ thống gửi thông báo phải đáng tin cậy. \newline - \textbf{NFR-UC1.13-2 (Timeliness):} Thông báo nên được gửi đi nhanh chóng sau khi lịch được xuất bản hoặc khi quản lý kích hoạt. \\
\hline
\end{longtable}

\subsubsection{Use Case UC-MD01-14: Xem Lịch làm việc Cá nhân}

\begin{longtable}{|m{4cm}|p{11cm}|}
\caption{Đặc tả Use Case UC-MD01-14: Xem Lịch làm việc Cá nhân} \label{tab:uc_md01_14_revised} \\
\hline
\multicolumn{2}{|c|}{\textbf{2.1. Tóm tắt (Summary)}} \\
\hline
\textbf{Mục} & \textbf{Nội dung} \\
\hline
\endhead % Header cho các trang tiếp theo
\hline
\endfoot % Footer cho bảng
\hline
\endlastfoot % Footer cho trang cuối cùng
Use Case Name & Xem Lịch làm việc Cá nhân \\
\hline
Use Case ID & UC-MD01-14 \\
\hline
Use Case Description & Cho phép Nhân viên (US-07) xem các ca làm việc đã được Quản lý nhà hàng (US-01) xuất bản chính thức cho bản thân mình thông qua cổng thông tin nhân viên hoặc ứng dụng di động (nếu có). \\
\hline
Actor & US-07 (Nhân viên) \\
\hline
Priority & Must Have \\
\hline
Trigger & Nhân viên muốn biết lịch trình làm việc sắp tới của mình. \\
\hline
Pre-Condition & - Nhân viên (US-07) có tài khoản và mật khẩu hợp lệ để đăng nhập vào cổng thông tin/ứng dụng nhân viên. \newline - Quản lý nhà hàng (US-01) đã tạo, gán và xuất bản (UC-MD01-12) ít nhất một ca làm việc cho nhân viên này. \newline - Cổng thông tin/ứng dụng nhân viên đang hoạt động. \\
\hline
Post-Condition & - Lịch làm việc đã được xuất bản của nhân viên được hiển thị rõ ràng cho nhân viên đó xem. \newline - Nhân viên nắm được thông tin về các ca làm việc sắp tới của mình (ngày, giờ, vai trò). \\
\hline
\multicolumn{2}{|c|}{\textbf{2.2. Luồng thực thi (Flow)}} \\
\hline
\textbf{Mục} & \textbf{Nội dung} \\
\hline
Basic Flow & 1. Nhân viên (US-07) mở ứng dụng di động hoặc truy cập trang web cổng thông tin nhân viên. \newline 2. US-07 nhập thông tin đăng nhập (tên người dùng/email và mật khẩu) và chọn Đăng nhập. \newline 3. Hệ thống xác thực thông tin đăng nhập thành công. \newline 4. US-07 chọn mục menu "Lịch làm việc của tôi", "My Schedule" hoặc tương tự. \newline 5. Hệ thống truy xuất tất cả các ca làm việc có trạng thái "Đã xuất bản" được gán cho US-07 trong khoảng thời gian mặc định (ví dụ: tuần hiện tại và tuần tới). \newline 6. Hệ thống hiển thị danh sách các ca làm việc này, bao gồm thông tin: Ngày, Giờ bắt đầu, Giờ kết thúc, Vai trò, (Tùy chọn) Địa điểm/Khu vực. \newline 7. US-07 xem lịch trình của mình. \\
\hline
Alternative Flow & \textbf{6a. Thay đổi khoảng thời gian xem:} \newline    1. US-07 sử dụng các điều khiển (ví dụ: chọn tuần/tháng, lịch nhỏ) để chọn một khoảng thời gian khác muốn xem. \newline    2. Hệ thống thực hiện lại bước 5 và 6 cho khoảng thời gian mới. \newline \textbf{6b. Xem dạng lịch (Calendar View):} \newline    1. Nếu hệ thống hỗ trợ, US-07 có thể chuyển sang xem lịch dưới dạng lịch tháng/tuần, với các ca được hiển thị trực quan trên đó. \\
\hline
Exception Flow & \textbf{3a. Đăng nhập không thành công:} \newline    1. Hệ thống xác thực thông tin đăng nhập thất bại. \newline    2. Hệ thống hiển thị thông báo lỗi đăng nhập. \newline \textbf{5a. Không có lịch làm việc được xuất bản:} \newline    1. Hệ thống không tìm thấy ca làm việc nào có trạng thái "Đã xuất bản" cho US-07. \newline    2. Hệ thống hiển thị thông báo "Bạn chưa có lịch làm việc nào được xếp." hoặc lịch trống. \newline \textbf{5b. Lỗi hệ thống khi truy xuất dữ liệu:} \newline    1. Hệ thống gặp sự cố kỹ thuật khi lấy dữ liệu lịch trình. \newline    2. Hệ thống hiển thị thông báo lỗi chung. \\
\hline
\multicolumn{2}{|c|}{\textbf{2.3. Thông tin bổ sung (Additional Information)}} \\
\hline
\textbf{Mục} & \textbf{Nội dung} \\
\hline
Business Rule & - \textbf{BR-UC1.14-1:} Nhân viên chỉ có thể xem các ca làm việc của chính mình đã được xuất bản. \newline - \textbf{BR-UC1.14-2:} Thông tin hiển thị phải đầy đủ (Ngày, Giờ, Vai trò). \\
\hline
Non-Functional Requirement & - \textbf{NFR-UC1.14-1 (Usability):} Giao diện xem lịch phải đơn giản, rõ ràng, dễ đọc trên cả máy tính và di động. \newline - \textbf{NFR-UC1.14-2 (Performance):} Thời gian tải lịch làm việc phải nhanh. \newline - \textbf{NFR-UC1.14-3 (Security):} Đảm bảo chỉ nhân viên mới xem được lịch của mình. \\
\hline
\end{longtable}

\subsubsection{Use Case UC-MD01-15: Đánh dấu Khoảng thời gian Không sẵn sàng}

\begin{longtable}{|m{4cm}|p{11cm}|}
\caption{Đặc tả Use Case UC-MD01-15: Đánh dấu Khoảng thời gian Không sẵn sàng} \label{tab:uc_md01_15_revised} \\
\hline
\multicolumn{2}{|c|}{\textbf{2.1. Tóm tắt (Summary)}} \\
\hline
\textbf{Mục} & \textbf{Nội dung} \\
\hline
\endhead % Header cho các trang tiếp theo
\hline
\endfoot % Footer cho bảng
\hline
\endlastfoot % Footer cho trang cuối cùng
Use Case Name & Đánh dấu Khoảng thời gian Không sẵn sàng \\
\hline
Use Case ID & UC-MD01-15 \\
\hline
Use Case Description & Cho phép Nhân viên (US-07), nếu được Quản lý nhà hàng (US-01) cho phép trong cấu hình, chủ động thông báo cho hệ thống về các khoảng thời gian cụ thể mà họ không thể làm việc (ví dụ: do nghỉ phép đã được duyệt, lịch hẹn cá nhân quan trọng). Thông tin này sẽ được Quản lý xem xét khi xếp lịch. \\
\hline
Actor & US-07 (Nhân viên) \\
\hline
Priority & Should Have \\
\hline
Trigger & Nhân viên biết trước mình sẽ không thể làm việc trong một khoảng thời gian và muốn thông báo cho quản lý để tránh bị xếp lịch. \\
\hline
Pre-Condition & - Nhân viên (US-07) đã đăng nhập vào cổng thông tin/ứng dụng nhân viên. \newline - Chức năng "Đánh dấu không sẵn sàng" đã được Quản lý kích hoạt. \\
\hline
Post-Condition & - Một bản ghi về khoảng thời gian không sẵn sàng (ngày/giờ bắt đầu, ngày/giờ kết thúc, lý do - tùy chọn) được tạo và liên kết với nhân viên đó. \newline - Thông tin này sẽ hiển thị cho Quản lý khi họ thực hiện gán ca (UC-MD01-06), giúp tránh xếp lịch cho nhân viên vào thời gian họ đã báo bận. \\
\hline
\multicolumn{2}{|c|}{\textbf{2.2. Luồng thực thi (Flow)}} \\
\hline
\textbf{Mục} & \textbf{Nội dung} \\
\hline
Basic Flow & 1. Nhân viên (US-07) truy cập cổng thông tin/ứng dụng nhân viên và chọn chức năng "Báo cáo không sẵn sàng" / "My Unavailability". \newline 2. Hệ thống hiển thị giao diện để nhập thông tin. \newline 3. US-07 chọn Ngày bắt đầu và Giờ bắt đầu không sẵn sàng. \newline 4. US-07 chọn Ngày kết thúc và Giờ kết thúc không sẵn sàng. \newline 5. (Tùy chọn) US-07 nhập Lý do không sẵn sàng. \newline 6. US-07 chọn lệnh "Lưu" / "Gửi". \newline 7. Hệ thống kiểm tra tính hợp lệ của ngày giờ (kết thúc sau bắt đầu - BR-UC1.15-1). \newline 8. Hệ thống lưu bản ghi thông tin không sẵn sàng. \newline 9. Hệ thống hiển thị thông báo lưu thành công. \\
\hline
Alternative Flow & \textbf{8a. Xem/Sửa/Xóa lịch không sẵn sàng đã báo:} \newline    1. Nhân viên có thể xem lại, sửa đổi hoặc xóa các khoảng thời gian không sẵn sàng đã báo trước đó (nếu còn trong thời hạn cho phép sửa/xóa). \\
\hline
Exception Flow & \textbf{7a. Dữ liệu ngày giờ không hợp lệ:} \newline    1. Hệ thống phát hiện thời gian kết thúc trước hoặc bằng thời gian bắt đầu. \newline    2. Hệ thống báo lỗi. Use Case quay lại bước 3. \newline \textbf{8a. Lỗi hệ thống khi lưu:} \newline    1. Hệ thống gặp lỗi khi lưu dữ liệu. \newline    2. Hệ thống báo lỗi chung. \\
\hline
\multicolumn{2}{|c|}{\textbf{2.3. Thông tin bổ sung (Additional Information)}} \\
\hline
\textbf{Mục} & \textbf{Nội dung} \\
\hline
Business Rule & - \textbf{BR-UC1.15-1:} Thời gian kết thúc không sẵn sàng phải sau thời gian bắt đầu. \newline - \textbf{BR-UC1.15-2:} Việc nhân viên đánh dấu không sẵn sàng là thông tin tham khảo. Quản lý vẫn có quyền quyết định cuối cùng về việc xếp lịch (tùy chính sách). \newline - \textbf{BR-UC1.15-3:} Có thể có quy định về việc phải báo trước bao lâu (ví dụ: trước 24 giờ). \\
\hline
Non-Functional Requirement & - \textbf{NFR-UC1.15-1 (Usability):} Giao diện nhập liệu phải dễ sử dụng, chọn ngày giờ thuận tiện. \newline - \textbf{NFR-UC1.15-2 (Integration):} Dữ liệu này phải được module Planning sử dụng hiệu quả khi quản lý gán ca. \\
\hline
\end{longtable}

\subsubsection{Use Case UC-MD01-16: (Tùy chọn) Yêu cầu Đổi ca/Xin nghỉ}

\begin{longtable}{|m{4cm}|p{11cm}|}
\caption{Đặc tả Use Case UC-MD01-16: (Tùy chọn) Yêu cầu Đổi ca/Xin nghỉ} \label{tab:uc_md01_16_revised} \\
\hline
\multicolumn{2}{|c|}{\textbf{2.1. Tóm tắt (Summary)}} \\
\hline
\textbf{Mục} & \textbf{Nội dung} \\
\hline
\endhead % Header cho các trang tiếp theo
\hline
\endfoot % Footer cho bảng
\hline
\endlastfoot % Footer cho trang cuối cùng
Use Case Name & (Tùy chọn) Yêu cầu Đổi ca/Xin nghỉ \\
\hline
Use Case ID & UC-MD01-16 \\
\hline
Use Case Description & Cho phép Nhân viên (US-07), nếu được cấu hình, gửi yêu cầu đổi một ca làm việc đã được phân công cho mình với một nhân viên khác, hoặc gửi yêu cầu xin nghỉ cho một ca/khoảng thời gian cụ thể. Yêu cầu này sẽ được gửi đến Quản lý để xem xét. \\
\hline
Actor & US-07 (Nhân viên) \\
\hline
Priority & Nice to Have \\
\hline
Trigger & - Nhân viên có việc đột xuất không thể làm ca đã được xếp. \newline - Nhân viên muốn đổi ca với đồng nghiệp. \newline - Nhân viên muốn xin nghỉ phép cho một khoảng thời gian. \\
\hline
Pre-Condition & - Nhân viên (US-07) đã đăng nhập vào cổng thông tin/ứng dụng nhân viên. \newline - Chức năng Yêu cầu Đổi ca/Xin nghỉ đã được Quản lý kích hoạt. \newline - Nhân viên có ít nhất một ca làm việc đã được xuất bản và phân công. \\
\hline
Post-Condition & - Một yêu cầu đổi ca hoặc xin nghỉ được tạo và gửi đến Quản lý (US-01). \newline - Yêu cầu ở trạng thái "Chờ phê duyệt" (Pending Approval). \newline - Nhân viên nhận được thông báo đã gửi yêu cầu thành công. \\
\hline
\multicolumn{2}{|c|}{\textbf{2.2. Luồng thực thi (Flow)}} \\
\hline
\textbf{Mục} & \textbf{Nội dung} \\
\hline
Basic Flow (Yêu cầu Xin nghỉ) & 1. Nhân viên (US-07) truy cập cổng thông tin/ứng dụng, vào mục "Lịch của tôi" hoặc "Yêu cầu nghỉ". \newline 2. US-07 chọn "Tạo yêu cầu nghỉ mới". \newline 3. US-07 chọn loại nghỉ (ví dụ: Nghỉ phép, Nghỉ ốm...). \newline 4. US-07 chọn Ngày/Giờ bắt đầu và Ngày/Giờ kết thúc muốn xin nghỉ. \newline 5. US-07 nhập Lý do xin nghỉ. \newline 6. US-07 gửi yêu cầu. \newline 7. Hệ thống tạo bản ghi yêu cầu nghỉ với trạng thái "Chờ phê duyệt" và thông báo cho Quản lý. \newline 8. Hệ thống báo cho US-07 đã gửi yêu cầu thành công. \\
\hline
Alternative Flow & \textbf{Basic Flow (Yêu cầu Đổi ca):} \newline    1. US-07 vào "Lịch của tôi", chọn ca làm việc muốn đổi. \newline    2. US-07 chọn hành động "Yêu cầu đổi ca". \newline    3. Hệ thống có thể hiển thị danh sách các nhân viên khác cùng vai trò, cùng thời điểm có thể đổi. \newline    4. (Tùy chọn) US-07 chọn một nhân viên muốn đề xuất đổi cùng (hoặc để ca đó thành "Ca mở" cho người khác nhận). \newline    5. US-07 nhập lý do đổi ca (nếu cần). \newline    6. US-07 gửi yêu cầu. \newline    7. Hệ thống tạo yêu cầu đổi ca, thông báo cho Quản lý (và có thể cả nhân viên được đề xuất đổi). \\
\hline
Exception Flow & \textbf{6a. Lỗi gửi yêu cầu:} \newline    1. Hệ thống gặp lỗi khi tạo hoặc gửi yêu cầu. \newline    2. Hệ thống báo lỗi cho US-07. \\
\hline
\multicolumn{2}{|c|}{\textbf{2.3. Thông tin bổ sung (Additional Information)}} \\
\hline
\textbf{Mục} & \textbf{Nội dung} \\
\hline
Business Rule & - \textbf{BR-UC1.16-1:} Quy trình và điều kiện cho việc đổi ca/xin nghỉ (ví dụ: phải báo trước bao lâu, giới hạn số lần) cần được định nghĩa rõ ràng bởi nhà hàng. \newline - \textbf{BR-UC1.16-2:} Yêu cầu phải được Quản lý phê duyệt (UC-MD01-17) thì mới có hiệu lực. \\
\hline
Non-Functional Requirement & - \textbf{NFR-UC1.16-1 (Usability):} Giao diện gửi yêu cầu phải đơn giản, dễ sử dụng. \newline - \textbf{NFR-UC1.16-2 (Workflow):} Cần có luồng thông báo rõ ràng giữa nhân viên và quản lý về các yêu cầu này. \\
\hline
\end{longtable}

\subsubsection{Use Case UC-MD01-17: Phê duyệt/Từ chối Yêu cầu Đổi ca/Xin nghỉ}

\begin{longtable}{|m{4cm}|p{11cm}|}
\caption{Đặc tả Use Case UC-MD01-17: Phê duyệt/Từ chối Yêu cầu Đổi ca/Xin nghỉ} \label{tab:uc_md01_17_revised} \\
\hline
\multicolumn{2}{|c|}{\textbf{2.1. Tóm tắt (Summary)}} \\
\hline
\textbf{Mục} & \textbf{Nội dung} \\
\hline
\endhead % Header cho các trang tiếp theo
\hline
\endfoot % Footer cho bảng
\hline
\endlastfoot % Footer cho trang cuối cùng
Use Case Name & Phê duyệt/Từ chối Yêu cầu Đổi ca/Xin nghỉ \\
\hline
Use Case ID & UC-MD01-17 \\
\hline
Use Case Description & Cho phép Quản lý nhà hàng (US-01) xem xét các yêu cầu đổi ca hoặc xin nghỉ do nhân viên gửi lên (từ UC-MD01-16) và đưa ra quyết định Phê duyệt (Approve) hoặc Từ chối (Refuse) các yêu cầu đó. \\
\hline
Actor & US-01 (Quản lý nhà hàng) \\
\hline
Priority & Nice to Have (Nếu có UC-MD01-16) \\
\hline
Trigger & Có một hoặc nhiều yêu cầu đổi ca/xin nghỉ từ nhân viên đang ở trạng thái "Chờ phê duyệt". Hệ thống có thể gửi thông báo cho Quản lý. \\
\hline
Pre-Condition & - US-01 đã đăng nhập vào hệ thống với quyền quản lý lịch trình và phê duyệt yêu cầu. \newline - Có các yêu cầu đổi ca/xin nghỉ đang chờ xử lý. \\
\hline
Post-Condition & - Trạng thái của yêu cầu được cập nhật thành "Đã phê duyệt" hoặc "Đã từ chối". \newline - Nếu yêu cầu được phê duyệt: \newline    - \textit{Xin nghỉ:} Khoảng thời gian đó có thể được tự động đánh dấu là không sẵn sàng cho nhân viên, hoặc Quản lý cần điều chỉnh lịch thủ công. \newline    - \textit{Đổi ca:} Lịch làm việc được cập nhật (nhân viên được đổi ca, ca được gán cho người mới). \newline - Nhân viên gửi yêu cầu nhận được thông báo về quyết định của Quản lý. \\
\hline
\multicolumn{2}{|c|}{\textbf{2.2. Luồng thực thi (Flow)}} \\
\hline
\textbf{Mục} & \textbf{Nội dung} \\
\hline
Basic Flow & 1. US-01 nhận được thông báo về yêu cầu mới hoặc truy cập vào mục quản lý Yêu cầu Nhân viên trong module Planning/HR. \newline 2. Hệ thống hiển thị danh sách các yêu cầu đang chờ phê duyệt, bao gồm thông tin người yêu cầu, loại yêu cầu, thời gian, lý do. \newline 3. US-01 chọn một yêu cầu để xem xét chi tiết. \newline 4. US-01 đánh giá yêu cầu (ví dụ: xem xét lý do, kiểm tra ảnh hưởng đến lịch trình chung, tình trạng nhân sự). \newline 5. US-01 chọn hành động "Phê duyệt" (Approve) hoặc "Từ chối" (Refuse). \newline 6. (Tùy chọn) US-01 có thể nhập lý do cho quyết định của mình (đặc biệt khi từ chối). \newline 7. Hệ thống cập nhật trạng thái của yêu cầu. \newline 8. Nếu Phê duyệt yêu cầu Xin nghỉ: Hệ thống có thể tự động tạo một bản ghi không sẵn sàng cho nhân viên đó (UC-MD01-15) hoặc nhắc nhở Quản lý cập nhật lịch. \newline 9. Nếu Phê duyệt yêu cầu Đổi ca: Hệ thống tự động cập nhật lịch làm việc (ví dụ: gỡ nhân viên A khỏi ca X, gán nhân viên B vào ca X, hoặc mở ca X cho người khác nhận). \newline 10. Hệ thống gửi thông báo kết quả (Phê duyệt/Từ chối kèm lý do nếu có) cho nhân viên đã gửi yêu cầu. \\
\hline
Alternative Flow & Không có luồng thay thế đáng kể. \\
\hline
Exception Flow & \textbf{7a. Lỗi cập nhật trạng thái yêu cầu/lịch trình:} \newline    1. Hệ thống gặp lỗi khi cố gắng cập nhật trạng thái yêu cầu hoặc khi tự động điều chỉnh lịch làm việc. \newline    2. Hệ thống báo lỗi. Quyết định có thể chưa được áp dụng hoàn toàn. \\
\hline
\multicolumn{2}{|c|}{\textbf{2.3. Thông tin bổ sung (Additional Information)}} \\
\hline
\textbf{Mục} & \textbf{Nội dung} \\
\hline
Business Rule & - \textbf{BR-UC1.17-1:} Quản lý phải xem xét cẩn thận các yêu cầu để đảm bảo hoạt động của nhà hàng không bị ảnh hưởng tiêu cực. \newline - \textbf{BR-UC1.17-2:} Quyết định của quản lý là cuối cùng. \newline - \textbf{BR-UC1.17-3:} Nếu yêu cầu được phê duyệt và ảnh hưởng đến lịch đã xuất bản, cần có cơ chế thông báo cho các nhân viên liên quan về sự thay đổi. \\
\hline
Non-Functional Requirement & - \textbf{NFR-UC1.17-1 (Usability):} Giao diện quản lý yêu cầu phải dễ dàng cho quản lý xem xét và ra quyết định. \newline - \textbf{NFR-UC1.17-2 (Workflow):} Luồng thông báo phê duyệt/từ chối phải kịp thời. \\
\hline
\end{longtable}


