% \subsection{Phạm vi đề tài}
% Phạm vi của đề tài được xác định rõ ràng để tập trung vào việc xây dựng và triển khai một hệ thống quản lý đặt món ăn trong nhà hàng. Hệ thống sẽ bao gồm các chức năng cốt lõi, được phát triển nhằm đáp ứng nhu cầu của các nhóm người dùng chính: khách hàng, các nhân viên và quản lý. Cụ thể, phạm vi đề tài được giới hạn như sau:
% \begin{itemize}

%     \item Đối tượng sử dụng:
%         \begin{itemize} 
%             \item Hệ thống được thiết kế cho các nhà hàng có nhu cầu cải tiến quy trình quản lý đặt món, phục vụ khách hàng, và vận hành nội bộ.
%             \item Các nhóm người dùng bao gồm: khách hàng, các nhân viên và quản lý.
%         \end{itemize}
        
%     \item Tính năng hệ thống:
%         \begin{itemize}
%             \item Xây dựng các chức năng cơ bản cho khách hàng như xem thực đơn, đặt món, đặt bàn, thanh toán và quản lý tài khoản cá nhân.
%             \item Cung cấp các công cụ hỗ trợ nhân viên nhà hàng trong việc quản lý đơn hàng, quản lý trạng thái món ăn, quản lý thanh toán và xử lý các yêu cầu từ khách hàng.
%             \item Phát triển các chức năng quản lý, bao gồm quản lý sản phẩm, doanh thu, nhân sự và vận hành.
%         \end{itemize}
        
%     \item Phạm vi triển khai:
%         \begin{itemize}
%             \item Hệ thống sẽ được thử nghiệm và triển khai trong môi trường mô phỏng, với khả năng mở rộng áp dụng cho các nhà hàng thực tế.
%             \item Phạm vi triển khai giới hạn ở các nhà hàng có quy mô nhỏ đến trung bình, với khả năng hỗ trợ tối đa một số lượng nhất định về đơn hàng và người dùng đồng thời.
%         \end{itemize}
        
%     \item \textbf{Giới hạn kỹ thuật:}
%         \begin{itemize}
%             \item Hệ thống được xây dựng dưới dạng nền tảng web với giao diện di động cơ bản, ưu tiên tính dễ sử dụng và hiệu quả, chưa tích hợp các công nghệ tiên tiến như trí tuệ nhân tạo hoặc phân tích dữ liệu lớn.
%             \item Hỗ trợ các phương thức thanh toán phổ biến như tiền mặt và thanh toán qua ngân hàng trực tuyến (e-banking), nhưng chưa mở rộng đến các phương thức yêu cầu thiết bị phần cứng như máy quẹt thẻ, nhằm đơn giản hóa việc triển khai và bảo trì.
%         \end{itemize}

% \end{itemize}

\subsection{Phạm vi đề tài}

Để đảm bảo dự án khả thi và tập trung vào việc giải quyết các mục tiêu cốt lõi trong khung thời gian và nguồn lực hạn chế, phạm vi của đề tài được xác định rõ ràng như sau:

\subsubsection{Phạm vi Chức năng}

Hệ thống sẽ tập trung vào các chức năng thiết yếu cho quy trình đặt món và quản lý cơ bản tại một nhà hàng điển hình. Các chức năng chính bao gồm:
        \begin{itemize}
            \item Quản lý Thực đơn: Thêm, sửa, xóa, ẩn/hiện món ăn, danh mục món ăn, quản lý giá bán và mô tả cơ bản.
            \item Quản lý Bàn/Khu vực: Thiết lập sơ đồ bàn, quản lý trạng thái bàn (trống, đang phục vụ, đã đặt, cần dọn).
            \item Quy trình Đặt món tại bàn: Nhân viên phục vụ nhận đơn qua thiết bị di động (máy tính bảng/điện thoại), gửi đơn tự động đến bếp/bar. Bao gồm khả năng chọn món, tùy chỉnh số lượng, ghi chú đơn giản.
            \item Hiển thị Đơn hàng tại Bếp (Kitchen Display System - KDS): Màn hình hiển thị danh sách các món cần chế biến, trạng thái món.
            \item Quản lý Đơn hàng cơ bản: Xem danh sách đơn hàng, trạng thái đơn hàng.
            \item Quy trình Thanh toán: Hỗ trợ tạo hóa đơn tạm, áp dụng giảm giá đơn giản (theo phần trăm hoặc số tiền cố định), ghi nhận thanh toán bằng tiền mặt và tích hợp thanh toán cơ bản qua ví điện tử.
            \item Báo cáo cơ bản: Báo cáo doanh thu theo ngày, báo cáo các món bán chạy trong ngày.
            \item Quản lý Tài khoản người dùng: Tạo và quản lý tài khoản cho các vai trò (Quản lý, Phục vụ, Bếp, Thu ngân, Quản lý) với phân quyền truy cập chức năng tương ứng.
        \end{itemize}

    % \item \textbf{Các chức năng không bao gồm (Out of Scope):}
    %     \begin{itemize}
    %         \item Quản lý Kho hàng chi tiết (Inventory Management): Không theo dõi tồn kho nguyên vật liệu. Chỉ có thể quản lý trạng thái hết món thủ công.
    %         \item Quản lý Chấm công và Lịch làm việc nhân viên.
    %         \item Hệ thống Quản lý Quan hệ Khách hàng (CRM): Không quản lý thông tin chi tiết khách hàng, lịch sử đặt món cá nhân, (vẫn có chương trình khách hàng thân thiết phức tạp).
    %         \item Tích hợp sâu với các nền tảng giao hàng inhouse của bên thứ ba (ShipDay).
    %         \item Phân tích và Báo cáo nâng cao (ví dụ: phân tích lợi nhuận, phân tích theo giờ cao điểm chi tiết, dự báo).
    %         \item Quản lý đa chi nhánh phức tạp (đồng bộ dữ liệu, báo cáo tổng hợp toàn chuỗi). Hệ thống ban đầu tập trung vào vận hành cho đa chi nhánh đơn giản.
    %     \end{itemize}

\subsubsection{Phạm vi Kỹ thuật}

Các giới hạn và lựa chọn về mặt công nghệ được xác định như sau:

\begin{itemize}
    \item \textbf{Kiến trúc hệ thống:} Dự kiến xây dựng theo kiến trúc ứng dụng kiến trúc Client-Server cơ bản để đơn giản hóa việc phát triển và triển khai trong giai đoạn đầu.
    \item \textbf{Công nghệ phát triển (Dự kiến):}
        \begin{itemize}
            \item \textit{Frontend (Giao diện người dùng):} Sử dụng một framework JavaScript hiện đại như React cùng với ShadcnUI và TailwindCSS để xây dựng giao diện tương tác và responsive.
            \item \textit{Backend (Xử lý logic):} Sử dụng một nền tảng phía máy chủ phổ biến Java (với Spring Boot) dựa trên kinh nghiệm của thành viên trong nhóm và các đặc điểm của nền tảng.
            \item \textit{Cơ sở dữ liệu:} Sử dụng hệ quản trị cơ sở dữ liệu quan hệ (SQL) PostgreSQL để đảm bảo tính nhất quán dữ liệu.
            \item \textit{Giao tiếp Real-time (nếu cần cho KDS/cập nhật trạng thái):} Có thể sử dụng WebSockets hoặc các kỹ thuật tương tự.
        \end{itemize}
    \item \textbf{Nền tảng triển khai:} Hệ thống chủ yếu hoạt động trên trình duyệt Web trên các thiết bị như máy tính, máy POS, máy tính bảng. Chưa bao gồm việc phát triển ứng dụng di động gốc (Native Mobile App) cho iOS hay Android trong phạm vi này.
    \item \textbf{Bảo mật:} Áp dụng các biện pháp bảo mật cơ bản như mã hóa mật khẩu, bảo vệ chống lại các lỗ hổng web phổ biến, phân quyền dựa trên vai trò. Không bao gồm các biện pháp kiểm thử xâm nhập (penetration testing) chuyên sâu.
    % \item \textbf{Hiệu năng và Khả năng mở rộng:} Hệ thống được thiết kế để đáp ứng hiệu năng cho quy mô một nhà hàng vừa và nhỏ. Các giải pháp tối ưu hóa hiệu năng sâu hoặc kiến trúc cho khả năng mở rộng quy mô lớn (ví dụ: microservices) nằm ngoài phạm vi ban đầu.
\end{itemize}

% Việc xác định rõ phạm vi này giúp đội ngũ tập trung nguồn lực vào việc xây dựng các chức năng cốt lõi, đảm bảo chất lượng và tiến độ của dự án. Các chức năng nằm ngoài phạm vi có thể được xem xét phát triển trong các giai đoạn hoặc phiên bản tiếp theo.
