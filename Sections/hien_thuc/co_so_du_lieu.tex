\subsection{Thiết kế cơ sở dữ liệu}
ER diagram là công cụ hữu hiệu giúp kiểm tra tính nhất quán và hoàn thiện trong thiết kế cơ sở dữ liệu. Nếu có sai sót hoặc mâu thuẫn trong hiểu biết ban đầu về hệ thống, chúng có thể được phát hiện và sửa chữa kịp thời thông qua ER diagram. Điều này góp phần loại bỏ các lỗi thiết kế và tối ưu hóa cơ sở dữ liệu.

\subsubsection{Bảng mô tả các schema}
Dưới đây là các bảng mô tả thuộc tính của các schema có trong lược đồ cơ sở dữ liệu quan hệ:

\begin{table}[H]
\centering
\renewcommand{\arraystretch}{1.5}
\begin{tabular}{|l|p{7cm}|p{4cm}|}
\hline
\textbf{Thuộc tính mô tả} & \textbf{Mô tả} & \textbf{Kiểu dữ liệu} \\
\hline
user\_id & ID của người dùng & bigint \\
\hline
username & Username của người dùng & varchar(255) \\
\hline
email & Email của người dùng & varchar(255) \\
\hline
password & Mật khẩu của người dùng đã được hash & varchar(255) \\
\hline
avatar\_url & URL dẫn đến avatar của người dùng & varchar(255) \\
\hline
user\_type & Loại người dùng (recruiter, candidate) & enum \\
\hline
\end{tabular}
\caption{Users Schema}
\end{table}

