\subsection{So sánh với hệ thống của nhóm}

Chúng em sẽ tiến hành so sánh hệ thống của mình với các hệ thống khác trong ngành, tập trung vào các chức năng mà họ cung cấp cho khách hàng. Cụ thể, chúng em sẽ đánh giá các nội dung trên trang web của họ, cách thức trình bày các thông tin đó và mức độ tương tác giữa người dùng và hệ thống.

Chúng em sẽ kiểm tra xem các hệ thống có cung cấp 10 nội dung sau đây không: Giới thiệu, Thực đơn, Đặt món trực tuyến, Đặt bàn, Đánh giá và bình luận, Khuyến mãi, Thông tin liên hệ, Đăng nhập/Tài khoản cá nhân, Tương thích di động, và Hỗ trợ khách hàng. Chúng em sẽ đánh giá cách thức trình bày và mức độ hoàn thiện của từng nội dung theo các cấp độ phân loại chi tiết dưới đây:

        \begin{enumerate}
            \item Giới thiệu
                \begin{itemize}
                    \item Cấp độ 1: Cung cấp thông tin cơ bản về nhà hàng (tên, địa chỉ).
                    \item Cấp độ 2: Bao gồm thông tin cơ bản cùng với lịch sử, sứ mệnh và tầm nhìn của nhà hàng.
                    \item Cấp độ 3: Ngoài các thông tin trên, còn có câu chuyện thương hiệu, giới thiệu đội ngũ quản lý và đầu bếp, cùng các giải thưởng đã đạt được.
                \end{itemize}
            \item Thực đơn
                \begin{itemize}
                    \item Cấp độ 1: Danh sách các món ăn và đồ uống cơ bản.
                    \item Cấp độ 2: Thực đơn kèm theo mô tả chi tiết và giá cả cho từng món.
                    \item Cấp độ 3: Thực đơn bao gồm hình ảnh chất lượng cao cho mỗi món, thông tin về nguyên liệu và giá trị dinh dưỡng.
                \end{itemize}
            \item Đặt món trực tuyến
                \begin{itemize}
                    \item Cấp độ 1: Thực đơn sẽ có phần riêng để khách hàng thêm món ăn vào đơn hàng của mình.
                    \item Cấp độ 2: Có hệ thống giỏ hàng cho phép khách hàng có thể coi lại những món ăn trước khi tạo đơn.
                    \item Cấp độ 3: Cho phép thanh toán trực tuyến thông qua các hình thức thanh toán online.
                \end{itemize}
            \item Đặt bàn
                \begin{itemize}
                    \item Cấp độ 1: Cho phép đặt bàn thông qua form cơ bản.
                    \item Cấp độ 2: Hiển thị trạng thái bàn và cập nhật trạng thái theo thời gian thực.
                    \item Cấp độ 3: Cung cấp sơ đồ chi tiết của nhà hàng, cho phép người dùng xem tổng quan các vị trí bàn còn trống.
                \end{itemize}
            \item Đánh giá và bình luận
                \begin{itemize}
                    \item Cấp độ 1: Cho phép nhìn thấy những bình luận do nhà hàng thu thập được, nhưng không cho phép bình luận trực tiếp trên hệ thống.
                    \item Cấp độ 2: Cho phép khách hàng đánh giá, bình luận trực tiếp và nhận phản hồi từ quản lý nhà hàng.
                \end{itemize}
            \item Khuyến mãi
                \begin{itemize}
                    \item Cấp độ 1: Giảm giá cho một món ăn cụ thể với mức giảm cố định.
                    \item Cấp độ 2: Cho phép sử dụng mã giảm giá hoặc voucher.
                    \item Cấp độ 3: Cung cấp chương trình tích điểm dành cho khách hàng thân thiết.
                \end{itemize}
            \item Thông tin liên hệ
                \begin{itemize}
                    \item Cấp độ 1: Chỉ cung cấp địa chỉ và số điện thoại.
                    \item Cấp độ 2: Thêm email liên hệ và bản đồ chỉ đường.
                    \item Cấp độ 3: Có form liên hệ trực tuyến, liên kết với mạng xã hội và hỗ trợ chat trực tiếp.
                \end{itemize}
            \item Đăng nhập/Tài khoản cá nhân
                \begin{itemize}
                    \item Cấp độ 1: Cho phép tạo tài khoản, không dùng mạng xã hôi.
                    \item Cấp độ 2: Cho phép dùng mạng xã hôi để đăng nhập.
                    \item Cấp độ 3: Tài khoản cá nhân cho phép xem lịch sử đặt hàng, nhận ưu đãi dành riêng và quản lý thông tin cá nhân.
                \end{itemize}
            \item Tương thích di động
                \begin{itemize}
                    \item Cấp độ 1: Hiển thị trên di động nhưng chưa tối ưu giao diện và chức năng.
                    \item Cấp độ 2: Tương thích hoàn toàn với thiết bị di động, giao diện và chức năng được tối ưu hóa.
                \end{itemize}
            \item Hỗ trợ khách hàng
                \begin{itemize}
                    \item Cấp độ 1: Cung cấp số điện thoại hỗ trợ chỉ trong giờ hành chính (ví dụ: 8h-17h). Khách hàng có thể gọi khi cần hỗ trợ về các vấn đề cơ bản như đặt món, thắc mắc về dịch vụ.
                    \item Cấp độ 2: Cung cấp email hỗ trợ, khách hàng có thể gửi yêu cầu qua email về các vấn đề cần giải đáp, và nhận phản hồi trong vòng 24 giờ. Hỗ trợ các vấn đề liên quan đến đơn hàng, thanh toán hoặc yêu cầu thông tin thêm về dịch vụ.
                    \item Cấp độ 3: Hỗ trợ khách hàng qua nhiều kênh (chat trực tuyến, email, điện thoại, form liên hệ) với phản hồi nhanh chóng 24/7. Khách hàng có thể liên hệ bất cứ lúc nào để giải quyết các vấn đề khẩn cấp như yêu cầu thay đổi đơn hàng, khiếu nại, hay cần trợ giúp về dịch vụ trong suốt quá trình sử dụng.
                \end{itemize}
        \end{enumerate}

        \begin{longtable}{|p{2cm}|p{1.5cm}|p{1.5cm}|p{1.5cm}|p{1.5cm}|p{1.5cm}|p{1.5cm}|p{1.5cm}|}
        \hline
        \textbf{Chức năng} & \textbf{Menu+ (Our System)} & \textbf{Cracco} & \textbf{KFC} & \textbf{Haidilao} & \textbf{Yoshinoya} & \textbf{Cơm Niêu Sài Gòn} & \textbf{Thanh's Deli}\\ 
        \hline
        \endfirsthead
        \hline
        \textbf{Chức năng} & \textbf{Menu+ (Our System)} & \textbf{Cracco} & \textbf{KFC} & \textbf{Haidilao} & \textbf{Yoshinoya} & \textbf{Cơm Niêu Sài Gòn} & \textbf{Thanh's Deli} \\
        \endhead
        \hline
        % \multicolumn{8}{|r|}{\small\slshape Còn tiếp} \\ \hline
        \endfoot
        \hline
        \endlastfoot
        Giới thiệu & 3 & 2 & 3 & 3 & 3 & 3 & 3\\ 
        \hline
        Thực đơn & 3 & 2 & 3 & 2 & 3 & 2 & 2\\ 
        \hline
        Đặt món trực tuyến & 3 & 0 & 3 & 3 & 3 & 3 & 3\\ 
        \hline
        Đặt bàn & 3 & 1 & 1 & 1 & 2 & 1 & 3\\ 
        \hline
        Đánh giá và bình luận & 2 & 0 & 0 & 0 & 2 & 0 & 2\\ 
        \hline
        Khuyến mãi & 2 & 2 & 3 & 3 & 2 & 2 & 2\\ 
        \hline
        Thông tin liên hệ & 3 & 1 & 3 & 2 & 3 & 3 & 2\\ 
        \hline
        Đăng nhập/Tài khoản cá nhân & 3 & 3 & 3 & 3 & 1 & 3 & 1\\ 
        \hline
        Tương thích di động & 2 & 2 & 2 & 2 & 2 & 2 & 2\\ 
        \hline
        Hỗ trợ khách hàng & 2 & 2 & 3 & 2 & 1 & 2 & 2\\ 
        \hline
        \caption{Bảng so sánh các nhà hàng}\\
        \end{longtable}

\subsection{Kết luận}
Việc so sánh hệ thống của nhóm với các hệ thống hiện có trong ngành nhà hàng đã giúp làm rõ những ưu điểm và nhược điểm của sản phẩm mà nhóm phát triển. Hệ thống của nhóm nổi bật với một số điểm mạnh quan trọng khi so sánh với các hệ thống liên quan.

Đầu tiên, hệ thống của chúng em được cải tiến với khả năng tích hợp các tính năng như tự gọi món qua ứng dụng và thanh toán trực tuyến. Mặc dù các hệ thống như Cơm Niêu Sài Gòn, KFC và Haidilao cũng đã triển khai các tính năng đặt món trực tuyến và thanh toán, nhưng hệ thống của nhóm mang đến một trải nghiệm người dùng dễ dàng và nhanh chóng hơn nhờ vào giao diện thân thiện và tiện lợi. Hơn nữa, hệ thống của nhóm cung cấp phản hồi thời gian thực và khả năng theo dõi đơn hàng một cách thuận tiện, điều mà một số hệ thống khác còn thiếu sót.

Mặc dù hệ thống của chúng em có những ưu điểm về việc tối ưu trải nghiệm người dùng và quy trình, song so với các hệ thống như Yoshinoya hay KFC, hệ thống của nhóm vẫn còn hạn chế về tính linh hoạt trong việc mở rộng các tính năng phức tạp hơn, như tích hợp chương trình khách hàng thân thiết hay quản lý các chương trình khuyến mãi phức tạp. Các tính năng như phân tích hành vi mua sắm và khuyến mãi dựa trên tần suất vẫn là những lĩnh vực cần cải tiến.

Tóm lại, hệ thống của chúng em thể hiện sự vượt trội trong việc tối ưu hóa quy trình đặt món và thanh toán, giúp nâng cao hiệu quả công việc và cải thiện trải nghiệm khách hàng. Tuy nhiên, để có thể phát triển mạnh mẽ hơn, hệ thống cần học hỏi và cải tiến thêm ở những lĩnh vực mở rộng tính năng và quản lý khách hàng. Việc so sánh này đã giúp nhóm nhận diện rõ các điểm mạnh hiện tại cũng như những khu vực cần cải thiện, làm cơ sở để phát triển hệ thống trong tương lai.




